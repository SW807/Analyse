%  A simple AAU report template.
%  2015-05-08 v. 1.2.0
%  Copyright 2010-2015 by Jesper Kjær Nielsen <jkn@es.aau.dk>
%
%  This is free software: you can redistribute it and/or modify
%  it under the terms of the GNU General Public License as published by
%  the Free Software Foundation, either version 3 of the License, or
%  (at your option) any later version.
%
%  This is distributed in the hope that it will be useful,
%  but WITHOUT ANY WARRANTY; without even the implied warranty of
%  MERCHANTABILITY or FITNESS FOR A PARTICULAR PURPOSE.  See the
%  GNU General Public License for more details.
%
%  You can find the GNU General Public License at <http://www.gnu.org/licenses/>.
%
\input{setup/preamble.tex}% package inclusion and set up of the document
\input{setup/hyphenations.tex}% 
\input{setup/macros.tex}% my new macros


\begin{document}
%frontmatter
\pagestyle{empty} %disable headers and footers
\pagenumbering{roman} %use roman page numbering in the frontmatter
%\hspace*{-1cm}\parbox[b][\textheight][t]{\textwidth}
{

\begin{center}
	\includegraphics[height=5.2cm]{aau_logo_da}\\
	\vspace{0.25cm}
	%Student Report
\end{center} 

\vspace{1cm}
\begin{center}

\textbf{\Huge {PsyLog - En Modulær Mobil Platform med Fokus på Affektive Lidelser}} \\ \vspace{0.5cm}
\textbf{\Large P8 Projekt af SW807F15 og SW808F15}\\ \vspace{0.5cm}
\textbf{\large 2. februar 2015 til 27. maj 2015}\\
\end{center}



\vspace{0.25cm}
\begin{center}
\item {\textbf{Deltagere:}} \\
\textbf{SW807F15:}\\
Mikael Elkiær Christensen\\
Mikkel Sandø Larsen\\
Stefan Marstrand Getreuer Micheelsen\\
Bruno Thalmann\\[0.2cm]
%list of group members
\textbf{SW808F15:}\\
Søren Skibsted Als\\
Lars Andersen\\
Lasse Vang Gravesen\\
Mathias Winde Pedersen

\end{center}

\thispagestyle{empty}

\newpage
\thispagestyle{empty}
\mbox{}
}
%\input{sections/colophon.tex}
\listoftodos
\pdfbookmark[0]{Titelblad}{label:titlepage_da}
\aautitlepage{%
  \danishprojectinfo{
    Rapportens titel %title
  }{%
    Semestertema %theme
  }{%
    Forårssemestret 2015 %project period
  }{%
    SW807 og SW808 % project group
  }{%
    %list of group members
    \textbf{SW807:}\\
    Mikael Elkiær\\
    Forfatter 2\\
    Forfatter 3
  }{%
    %list of supervisors
    Vejleder 1\\
    Vejleder 2
  }{%
    1 % number of printed copies
  }{%
    \today % date of completion
  }%
}{%department and address
  \textbf{Elektronik og IT}\\
  Aalborg Universitet\\
  \href{http://www.aau.dk}{http://www.aau.dk}
}{% the abstract
  Her er resuméet
}

\cleardoublepage

\section*{Forord}
Denne rapport er første del i et projekt om psykiske lidelser, som del af Software studie på Aalborg Universitet i P8 projektet mellem 2. Februar 2015 og 27. Maj 2015. 
Den er udarbejdet i et samarbejde mellem grupperne SW807F15 og SW808F15.
Projektet er vejledt af Ivan Aaen, hvis vejledning er meget værdsat.
Derudover er projektet foreslået af Morten Aagaard som gennem projektet agerede som bindeled mellem projektgruppen og specialister indenfor psykologi. Dette resulterede blandt andet i møder med psykolog Janne Vedel Rasmussen, psykiater Jørgen Aagaard, og et fokusgruppe interview med patienter med affektive lidelser. Hans hjælp, råd samt indsigt i faget var en stor hjælp for projektet.
\cleardoublepage

\pdfbookmark[0]{Indholdsfortegnelse}{label:contents}
\pagestyle{fancy} %enable headers and footers again
\setcounter{tocdepth}{1}
\tableofcontents

%\input{sections/preface.tex}
\cleardoublepage
%mainmatter
\pagenumbering{arabic} %use arabic page numbering in the mainmatter

\chapter*{Introduktion}
\addcontentsline{toc}{chapter}{Introduktion}
% Psykiske lidelser, hvor mange er påvirket?
% Forskellige slags, nogle er normalt fungerende

% Normalt fungerende = normale mennesker = forskellige behov
% Platform der kan ramme bredt
% Usikkert område, som stensikkert er her i lang tid = mulighed for at tilpasse

% Mulighed for at bruge mobilteknologi (initierende problem)
% Undersøge muligheder ift. sensorer/logning

Ifølge Psykiatrifonden er det sådan at \textit{''I Danmark får hver 3. af os på et tidspunkt en psykisk sygdom – og mange flere er pårørende. Nu og her oplever næsten 700.000 danskere psykiske problemer, hvoraf stress, angst og depression udgør den største kilde til alt fra dårlig trivsel til svær sygdom.''}\cite{psykiatrifonden}




\chapter{Problemanalyse}
Dette afsnit analyserer og afgrænser problemdomænet.
Det indeholder en gennemgang af nøglepunkter fra et møde med psykolog, Janne Rasmussen, og en professor indenfor psykiatri, Jørgen Aagaard.
Desuden er der foretaget et fokusgruppeinterview med patienter der har affektive lidelser, som bliver analyseret og de vigtigste punkter fremhævet.
Der er en gennemgang af vigtige begreber og eksisterende systemer.
Til sidst vælges en platform på baggrund af ovenstående.

\section{Sygdomme}
Vi skelner i vores arbejde mellem tre kategorier af sygdomme, disse værende somatisk, psykisk og affektiv.

Somatiske sygdomme er kendetegnende ved at være sygdomme der grunder i det kropslige, eksempelvis Parkinsons syge.
Hvis man modsat er psykotisk betyder det at på grund af sygdom har man nedsat realitetstestning\citep{}.
%Disposition
%Affektiv
%pyskotisk
%somatisk

\section{Affektive Lidelser}\label{sec:affektivelidelser}
Dette afsnit er primært baseret på \citet{misc:affektivelidelser, misc:netpsykdepression, misc:netpsykmani}.

Affektive lidelser omfatter en række sygdomme hvor stemningslejet afviger fra det habituelle.
\stefan{vi har så vidt jeg kan se ikke nogen kilde på en definition af ``habituelt stemningsleje''}
Kendetegnende ved sindslidelserne er at de er hyppige og forekommer periodisk.
Dette varierer meget, da nogle har enkeltstående episoder, mens andre har tilbagevendende episoder.
Der er en stor dødelighed blandt patienter med sygdommen, hovedsagligt grundet selvmord.

Stemningslejet kan afvige fra det habituelle på forskellig vis.
Man skelner mellem mani og depression.
Unipolare patienter lider af depression og oplever depressions perioder.
Bipolare patienter derimod har haft minimum én mani periode og nul eller flere depressions perioder.
Derudover findes der en blandingstilstand hvor man både har depressive og maniske symptomer på samme tid.


Stemningsleje er et begreb der dækker over humør og lyst.
Ved depression er stemningslejet sænket, mens det ved mani er løftet.

Der findes forskellige årsager til affektive lidelser. 
Disse inkluderer miljømæssige og arvemæssige forhold, især ved bipolare patienter.
Det skal forstås således at de miljømæssige og arvemæssige forhold gør patienten mere sårbar overfor de affektive lidelser.

Derudover kan somatiske sygdomme såsom blodprop, hjerneblødning og Parkinsons sygdom forårsage depression og mani.
Ud over dette kan misbrug, samt bestemte former for medicinbrug, forårsage mani og depression.

\subsection{Depression}
Risikoen for at udvikle en depression for kvinder er på ca. 8\%, mens den for mænd er på ca. 4\% \citep{misc:affektivelidelser}.
Debutalderen er almindeligvis 40-50 år \citep{misc:affektivelidelser}.
En ubehandlet sygdom varer almindeligvis seks til tolv måneder.
Kroniske depressioner med op til flere års varighed er ikke ualmindelige, især hos ældre \citep{misc:affektivelidelser}.
Derudover er det ca. 15\% af patienterne der kun har en enkeltstående hændelse af depression, hvilket understreger at sygdommen ofte er tilbagevendende.

\Citet{misc:netpsykdepression} klassificerer en depressiv enkeltepisode ud fra en række kriterier, der nævnes herefter.
De følgende kriterier er taget direkte ud fra \citet{misc:netpsykdepression}.

\begin{mdframed}
\subsubsection{Depression kriterier}
For at opfylde kriterierne for at have en depressiv enkeltepisode i lettere grad skal man:
\begin{itemize}
	\item Have haft depressionen i mere end 2 uger
	\item Ikke tidligere have haft en mani eller en let mani, en såkaldt hypomani
	\item Ikke have en fysisk lidelse som kan forklare symptomerne
\end{itemize}
Man skal have mindst 2 af følgende kernesymptomer:
\begin{itemize}
	\item Man er i dårligt humør og er nedtrykt og trist
	\item Man har nedsat lyst til at foretage sig noget, og man har mere eller mindre mistet interessen for ting, man plejer at interessere sig for
	\item Man bliver hurtigt træt og har ikke så meget energi som man plejer
\end{itemize}
Desuden skal man have mindst 2 af følgende ledsagesymptomer:
\begin{itemize}
	\item Man har nedsat selvtillid eller selvværdsfølelse
	\item Man lider af skyldfølelse og bebrejder sig selv urimeligt
	\item Man har tanker om at det ville være bedre, hvis man var død, eller man tænker på at begå selvmord
	\item Man har svært ved at koncentrere sig eller oplever at man ikke kan tænke klart
	\item Man er enten urolig og hvileløs, eller også er ens bevægelser nærmest gået i stå
	\item Man sover enten mere eller mindre, end man plejer
	\item Man har mistet appetitten og har tabt sig, eller man er begyndt at trøstespise og har taget på
\end{itemize}
\end{mdframed}

\subsubsection{Grader af depression}
Man skelner mellem forskellige grader af depression.
Depression i lettere grad vil sige at man er i stand til at bibeholde sine sædvanlige aktiviteter, på trods af at man har nedsat stemningsleje.
Ved depression i moderat grad har man 4 ledsagesymptomer og som resultat af dette har man svært ved at forsætte med sin sædvanlige aktiviteter.
Har man en svær depression har man alle 3 kernesymptomer og 5 ledsagesymptomer.
Ved svær depression er man ikke i stand til at fortsætte med sine normale aktiviteter.

\subsubsection{Individuelle Symptomer}
Det er vigtigt for patienter der lider af depression, at de er opmærksomme på hvad deres symptomer er.
Disse symptomer vil for individet ofte starte hver gang.
Eksempler på dette kan være en overfladisk søvn, forøget bekymringer om bagateller eller forlænget søvn.

Hvis disse symptomer opfanges, kan behandling af depressionen påbegyndes tidligere i forløbet og i bedste tilfælde kan svære tilfælde forhindres.
\subsubsection{Selvbehandling}
Man kan nedsætte risikoen for en depression hvis man har en sund livsstil.
Dette inkluderer at spise sund mad, at dyrke motion og undgå at indtage rusmidler.
Desuden informerede kontaktpersonen \citet{misc:janne-rasmussen} at til behandling af depression opfordres patienten til at foretage en mængde af lystbetonede aktiviteter.

\subsection{Mani}
Risikoen for at udvikle en mani, og derved en bipolar sygdom, er ca. 1-2\%, og er således lige hyppig blandt mænd og kvinder.
Dog er det ofte at sygdommen er tilbagevendende, da der er en risiko på ca. 90\% for at få en ny episode på et senere tidspunkt.
Debutalderen for sygdommen er almindeligvis før 30-års alderen, og ved omkring halvdelen af patienterne forekommer sygdommen før man er 20 år.
Varigheden af sygdommen er i ubehandlede tilfælde typisk mellem to og otte måneder, men der findes her også kroniske tilfælde,

\citet{misc:netpsykmani} klassificerer en manisk enkeltepisode uden psykotiske symptomer som at man i mere end en uge skal opfylde følgende:
\begin{mdframed}
\begin{itemize}
	\item Have været opstemt, eksalteret og irritabel.
	\item Hvis man er opstemt eller eksalteret have mindst tre af følgende symptomer. Hvis man især er irritabel skal man op på mindst fire symptomer:
	\begin{itemize}
		\item Man er hyperaktiv, rastløs og urolig
		\item Man føler et indre pres for at tale uafbrudt
		\item Man har tankeflugt, hvor tankerne springer fra emne til emne
		\item Man har en hæmningsløs adfærd, hvor ens normale hæmninger er væk
		\item Man har nedsat behov for søvn
		\item Man har forhøjet selvfølelse, grandiositet
		\item Man er usamlet eller bliver konstant distraheret
		\item Man handler hensynsløst og uansvarligt
		\item Man har større seksualdrift end normalt
	\end{itemize}
	\item Ikke have haft hallucinationer eller vrangforestillinger
	\item Symptomerne må ikke skyldes en fysisk sygdom
\end{itemize}
\end{mdframed}
Hvis man derimod har en mani med psykotiske symptomer svarer den til symptomerne for en manisk enkeltepisode uden psykotiske symptomer, men hvor man har haft hallucinationer, eller vrangforestillinger. Dog ikke bizarre vrangforestillinger såsom ved skizofreni.

Derudover kan man også have hypomani der er en lettere grad for mani, hvor man blot i mere end fire dage skal have en manisk episode.

\subsubsection{Individuelle Symptomer}
Det er individuelt hvilke symptomer de enkelte patienter har på en begyndende mani periode.
Dog starter perioden ofte på samme måde som tidligere perioder.
Eksempelvis kan man være meget aktiv, rastløs, have mindre brug for søvn, være ekstatisk etc.

Ligesom depression er det vigtigt at opdage episoderne i de begyndende stadier, da man også her på den måde ville kunne mindske eller helt undgå episoden.
Derudover kan en depression følge efter en mani periode, og man kan derfor også mindske risikoen for disse episoder \citep{misc:bipolarsundhed}.

\subsubsection{Selvbehandling}
Man skal undgå at drikke alkohol når man er i en manisk periode, da det kan forværre perioden \citep{misc:netpsykmani}.
Derudover gælder det om at begrænse mængden af stimuli, og generelt forsøge at tage det mere med ro \citep{misc:janne-rasmussen}.

\subsection{Bipolar lidelse}
En bipolar lidelse er kendetegnende ved minimum en manisk episode og evt. depressions episoder.
Et eksempel på dette med en stemningsleje graf over tid kan ses i \cref{fig:stemningslejegrafeksempel}.

\begin{figure}
	\centering
	\includegraphics[scale=0.5]{affektivstemningsleje}
	\caption{Graf over stemningsleje for en patient med en mani og to depressions perioder.}\label{fig:stemningslejegrafeksempel}
\end{figure}

Det oftest sete er en mani episode og flere depressions episoder, men andre scenarier kan også forekomme.
Det gælder således om at have patienten på det habituelle niveau og hurtigt opdage når en patient afviger fra dette.

\input{indhold/moedemedjanne2}
\section{Møde med Psykiatri Professor Jørgen Aagaard}\label{sec:moede-med-joergen}
Dette afsnit opsummerer et møde med psykiatri professor Jørgen Aagaard på Psykiatrien ved Sygehus Syd i Aalborg \citep{misc:jorgen-aagaard}. 
Jørgen er psykiater, samt almindelig somatisk læge, hvor hans fokus område er primært forskning ind i psykiske lidelser.
Relevante emner fra mødet opsummeres i dette afsnit.

For at kunne detektere optrapning eller nedtrapning af depression/mani er ændring i adfærd centralt, dette kan eksempelvis gøres ved at se på døgnrytme og generelt aktivitetsniveau. 
Det kan også måles på kropslig stress, hvilket kan være at patienten begynder at svede eller pulsen ændrer sig uden at lave noget aktivt. 
Det er selvfølgelig vigtigt for alle personer at de har en regulær døgnrytme eller et stabilt aktivitetsniveau, men det er mere vigtigt for deprimerede.

\paragraph{Social Adfærd og Døgnrytme}
Social adfærd ændrer sig ved depression og mani.
Ved depression ændrer døgnrytmen sig, den naturlige sammenhæng mellem følelser og hvad der siges mindskes, og aktivitetsniveauet nedsættes.

Ved mani er det lidt anderledes, de sover mindre, er mere urolige, har forskellig seksualitets niveau, spiser mindre og den sociale kontakt gennem mobil kan være spam-lignende. 

Hvis døgnrytme kan detekteres ved hjælp af en mobil ville det være brugbart at se ændringer i denne.
Dette er vigtigt da hvis døgnrytmen ændrer sig er der større chance for at lidelsen er ved at ændres.

Såfremt det er muligt at måle social eller fysisk aktivitet på smartphonen, kan dette bruges til at måle ændringer hvilket kan indikere en ændring i lidelsen.

Hvis man kan måle kropslig stress kan dette også indikere en forværring af lidelsen, dette kunne f.eks. være måling af ufrivillige rystelser, sved eller puls. 

Det der anses som de mest værdifulde er døgnrytme, altså hvornår man går i seng og hvornår man står op, og på samme tid også se på om man kan registrere forstyrret søvn, altså usammenhængende søvn. 
Andre vigtige faktorer er social aktivitet og kropslig stress.

\paragraph{Ændringer}
Jørgen lagde meget stor vægt på at der ikke kan dannes et generelt billede af hvordan adfærd ser ud for en rask person, som man derefter kan sammenligne med for at finde ud af om en person har enten depression eller mani.
Det er derfor nødvendigt at finde en baseline (en normaladfærd) for individet og derefter kigge på ændringer i forhold til denne baseline.

\paragraph{Kognitive Ændringer}
Hvis man er stresset vil man få det værre. 
Kognitive egenskaber som korttidshukommelse og løsning af matematiske eller logiske problemer vil være hæmmet.
Man kan derfor forestille sig at patienten skal udføre en test der afprøver patientens evner i disse områder.

\paragraph{Visuel Repræsentation}
Hvis man skal præsentere tilstanden for patient skal den være enkel og være på patientens egne præmisser. Det er en god idé at præsentere hvordan patienten har det ud fra den registrering applikationen har gjort. Dette skal så kunne bruges som et hjælpemiddel der kan give objektiv information til patienten om deres tilstand.

Det komplette referat af mødet med Jørgen Aagaard kan ses i \cref{app:moede-med-joergen-referat}.

\paragraph{}
For at supplere viden om den aktuelle målgruppe; patienter med affektive lidelser, specifikt uni- og bi-polar affektiv lidelse, blev der arrangeret et fokusgruppeinterview med indlagte patienter.

\newcommand{\pa}{A}
\newcommand{\pb}{B}
\newcommand{\pc}{C}
\newcommand{\pd}{D}
\newcommand{\pe}{E}

\section{Fokusgruppe Interview}
Dette fokusgruppe interview blev udført 20/4-2015, på Aalborg Psykiatrisk Afdeling, hvoraf de deltagende var 5 unipolare patienter.
De 5 patienter vil herefter blive benævnt henholdsvis \pa, \pb, \pc, \pd~og \pe.

To af gruppens medlemmer deltog og fungerede som interviewere, samt en tredje der skrev referat og optog interviewet.
Først blev projektet introduceret, ved at den overordnede idé blev præsenteret.
Derudover blev der begrænset, da der var meget lidt produkt at vise, hvilket resulterede i en ca. halvering af varigheden.

Herefter følger en kronologisk gennemgang af de stillede spørgsmål.

\subsection{Spørgsmål}

\paragraph{Hvad er et kendetegn for forværring af jeres situation?}
\pa~fortæller at søvn er en vigtig faktor; før \pa s sidste periode gik \pa~søvnløs i 4 dage.
Derudover trækker \pa~sig socialt; \pa~skulle have besøgt noget familie, men så sig nødsaget til at vælge det fra.
(\pb~forlader rummet).
\pc~er enig med \pa, i at man stopper med at sove og at man trækker sig socialt.
\pd~har svært ved at håndtere når der sker for meget uvist, hvilket gør det svært at rumme/overskue det, og dette afhænger også meget af søvn.
\pe~fortæller at det er de samme ting der går igen; dårlig appetit, dårligt humør og dårlig søvn.
\pe~skal virkelig tage sig sammen for at udføre noget som helst.
Når \pe~har en periode, er morgener en stor plage, hvor aftener er bedre.
(Der er en mindre diskussion omkring hvordan det er at have en periode).

\paragraph{Hvilken information ville I ønske I havde adgang til vedrørende jeres adfærd?}
\pe~forklarer at man ikke er i tvivl om at man har en depression og mener ikke at man vil orke at slå noget op i applikationen når først man er i en periode.
\pa~mener at det kunne være brugbart med en graf over eksempelvis søvnmønster, så man kunne danne sig et overblik.
Der er stod enighed om at når man først er i en periode, er man ikke i tvivl om dette.
\pe~lægger ikke umiddelbart mærke til noget op til perioder, men mener at de kommer brat.
\pa~fortæller at han har lagt mærke til at hans søvn er blevet forringet op til perioden.
\pa~indskyder at det ville være fordelagtigt at fange perioden tidligt.


\section{Vigtige begreber}
Dette afsnit definerer vigtige begreber inden for psykiatrien, som patient empowerment, trends, cykler og interventioner.

\subsection{Patient Empowerment}\label{sec:patientempowerment}
Patient empowerment omhandler en inddragelse af patienten i egen behandling.
Dette inkluderer at give redskaber til patienten, der gør det nemmere for patienten at vurdere sin egen sygdomssituation og foretage informerede behandlingsvalg ud fra den vurderede sygdomssituation.
Det er en tanke der er gøres en indsats for at implementere af sygehusvæsenet \citep{misc:patientpowerhovedstaden}.
Dog er det vigtigt at patient empowerment udføres ordenligt.
Steder hvor strategien ikke er at foretrække er hvis patienten føler sig utryg ved at skulle have et sådant ansvar for egen behandling, hvor patienten helst ikke vil indrages yderligere i behandlingen.
Derudover kræver det at patienten bliver velinformeret om hvordan han skal monitorere egen sygdom og handle ud fra det.

Til at understøtte denne informering af patienten om egen sygdom kan diverse empiri kilder være fordelagtige, såsom søvnændringer og aktivitetsændringer\citep{misc:jorgen-aagaard}, og er derfor undersøgt nærmere i dette projekt.
Måder man kan understøtte denne selvbehandling er ved at se på diverse mønstre og handlinger, hvilket inkluderer trends, cykler og interventioner.

\subsection{Trends}
Trends beskriver tendenser i en persons adfærd, eksempelvis at man går i seng kl. 22 og står op kl. 7 eller går en tur hver aften.
Det interessante med trends er at se på ændringer, da disse kan antyde en ændret sindstilstand.
En ændring kunne være at personen begynder at gå i seng kl 4 eller sover til kl 12 i en periode.

\subsection{Cykler}
Cykler omhandler fænomenet at en persons sindstilstand går i cykler.
Dette er interessant da ens adfærd bliver påvirket af ens sindstilstand.
Ved at kunne detektere den nuværende sindstilstand og har kortlagt cyklus for den pågældende person, kan dette bruges til at forudsige det næste stadie, der kunne være en depression.

\subsection{Interventioner}
Interventioner omhandler handlinger der kan afbryde en sindstilstand eller påbegyndende sindstilstand.
Eksempelvis ved en påbegyndende depression kan man foretage sig lystbetonede aktiviteter.
Dette kan være farligt at overlade til et program at diktere, da forkerte vurderinger kan have voldsomme konsekvenser.
En måde dette kunne være et problem er ved en manisk patient der bliver vurderet depressiv, der ville en lystbetonet aktivitet, såsom at gå en tur, ikke gavne og kunne gøre situationen værre.


\section{Eksisterende systemer}
Bla bla bla introduktion.

\subsection{Ginger.io}
Ginger.io er en iPhone/Android app, beregnet til at assistere patienter med diverse psykiskse lidelser.
Følgende afsnit er baseret på \cite{ginger_dot_io}, \cite{gingerio_mit} og \cite{gingerio_dailymail}.


App'en er lavet til at assistere i behandlingen for følgende lidelser: depression, angst, bipolær affektiv lidelse og skizofreni.

Efter at app'en er installeret, vil den indsamle data, hvilket kan sige noget om mobil-brugerens sindsmæssige tilstand.
Der bliver overordnet set indsamlet to typer data; aktiv og passiv.
Den aktive data er forespørgsler fra app'en, hvor mobil-brugeren selv skal svare.
Den passive data vil blive indsamlet i baggrunden, og består af interaktions- og lokations-data, hvor interaktions-data er mobil-brugerens opkald- og SMS-vaner og lokations-data er opfanget via GPS og accelerometer, for at sige noget om hvor og hvordan mobil-brugeren bevæger og opholder sig.


\section{Valg af Platform}\label{sec:valg_af_android}
Platformen for projektet skal vælges, her ses der på hvilke fordele og ulemper de forskellige platforme har og endeligt laves der en beslutning.

Der er forskellige platforme der kan vælges, specifikt Android, iPhone og Windows Phone.
Der kunne også vælges en hybrid platform som for eksempel Apache Cordova der tillader udvikling af smartphone applikationer ved hjælp af HTML, CSS og JavaScript \citep{misc:apachecordova}, eller kryds platform udvikling som for eksempel Xamarin der tillader udvikling i C\# som kan kompileres til mange forskellige operativ systemet såsom Android, iOS og andre \citep{misc:xamarin}.

Fordelene ved Android er som følgende:
\begin{itemize}
\item Det er en meget åben platform som giver adgang til meget sensor data
\item Android har flere brugere end iPhone og Windows Phone.
\item Der er ingen begrænsninger på hvad der kan udvikles
\item Der er mange udviklings ressourcer tilgængelige såsom guides, tutorials.
\item Android applikationer udvikles som regel i Java, hvilket er et udbredt programmeringssprog.
\end{itemize}

Android har dog også ulemper, såsom mange forskellige typer smartphones med forskellig hardware og forskellige versioner af operativ systemet.

Fordelene ved iPhone er primært at der er nemmere at udvikle til da det ikke er særlig fragmenteret i forhold til Android. 
Brugerfladen er meget standardiseret hvilket gør det nemmere at udvikle den del af applikationen. 
iPhone har ulemper, da det er et meget mere lukket system hvilket gør at data kan være umulige at få fat i. iPhone udvikling foregår kun på OS X, og kræver en licens. iPhone udvikling foregår i et sprog som ikke bruges bredt, Objective-C.

Fordelene ved Windows Phone er et meget modent/godt udviklingsmiljø og programmeringssprog, og at brugerflade design er meget nemt. 
Ulemper ved Windows Phone er så at der ikke er særlig mange udviklings ressourcer tilgængelig da Windows Phone markedet ikke er særlig bredt. Windows Phone er også en lukket platform.

Fordelene ved en hybrid platform som for eksempel Apache Cordova er at ens applikation kommer ud til den bredest mulige målgruppe, da man her har adgang til alle de forskellige smartphone enheder.
Det muliggøre at skrive en applikation i HTML, CSS og JavaScript og gør det muligt at lave en applikation uden at bruge platformens native codelanguage.
Et problem med cross platform er komplicerer designprocessen, da hver smartphone platform har forskellige design guidelines.
Derudover giver det også en væsentlig uoverensstemmelse mellem hvad man kan forvente af de underliggende datalag, hvilket kan gå ud over kvaliteten af det udviklede software.

Vores projekt er meget afhængig af åben adgang til datakilder, hvilket Android giver bedre adgang til.
Vores situation er også at udvikling på Android virker nemmere da det ikke påkræver et specielt operativ system da ingen på udviklingsholdet har OS X computere, og at udvikling i Java er kendt blandt mange på udviklingsholdet.
Desuden fravælger vi også hybrid- og kryds platform, idet at brug af disse vil have den konsekvens at lav-niveau kontrol vil blive tabt og at verificering af systemet vil kræve flere smartphones at teste på.



\chapter{Essence}
Essence er en softwareudviklingsteknik der er under udvikling på Aalborg universitet og som beskrives i \citet{art:essence}
Følgende afsnit beskriver benyttelsen af diverse emner indenfor Essence til vores udviklings proces.
% Systematisk beskrivelse af konfigurations tabel.
\section{Konfigurationstabel}\label{konfigurationstabel}
En konfiguration af et system er en kombination af beslutninger relateret til systemets opbygning, dette inkluderer beslutninger om målgruppe, systemkoncept, komponenter og mere.
En konfigurationstabel er så en tabel med en sådan konfiguration, hvor konfigurationen er inddelt i de fire kategorier Paradigm, Product, Project og Process \citep{art:essence}.
I \cref{tab:konfigurationsTabel} præsenteres konfigurationstabellen for projektet, og indholdet vil i de følgende afsnit blive forklaret.

\begin{figure}
\includegraphics[scale = 0.65,trim = 1cm 3cm 6cm 2cm, angle = 90, clip]{KonfigurationTabel}
\caption{Konfigurations tabellen for systemet.}
\label{tab:konfigurationsTabel}
\end{figure}

\section{Vision}
For at få et 'Vision' af projektet overvejes der følgende koncepter fra \citet{art:essence}, derudover er der enkelte oversatte citater derfra.
\textit{Metafor}, \textit{Ikon}, \textit{Prototype} og \textit{Proposition} er de forskellige muligheder som bogen præsenterer som kan bruges til at repræsenterer en vision.

Ifølge bogen er Metafor \textit{et ord eller sætning der repræsenterer eller figurativt beskriver en løsning uden af være bogstavelig anvendelig.}
Bogen beskriver Ikon som \textit{et symbol der eksemplificerer en eller flere nøgle kvaliteter i en idé.} Det går på de visuelle, hvor øvelsen er hvordan man konkretiserer en eller flere nøglekvaliteter af en idé. Dette er eksempelvis hvor den teknologiske del er nemt udførlig, men hvordan det æstetiske i en løsning ikke er umiddelbar.
Derudover beskrives Prototype som \textit{en kompleks og konkret repræsentation af et vision. Prototyper er ufærdiggjort software som traditionelt bruges til at bekræfte udviklernes opfattelse af krav.} Denne bruges til at få feedback på en ufuldstændig software løsning, og er en fysisk repræsentation af løsning. 
Til sidst Proposition er \textit{er et udsagn som er velbegrundet og har positive egenskaber.} Denne svarer basalt set til en traditionel problemformulering. 

Vores vision for projektet repræsenteres ved hjælp af metafor repræsentationen, specifikt formuleres det ved hjælp af tre metaforer.
Vi bruger metafor på baggrund af at det giver et løst overblik over selve løsningen som giver plads til at ændre på løsningen uden at miste overblikket. 

De tre metaforer er \textit{Objektiv dagbog}, \textit{Fitness tracker} og \textit{F16 fly}.
At formulere vores vision som en metafor lader os koncist specificere hvad der er nøgleaspekterne af produktet.

Den objektive dagbog danner tanken om en dagbog baseret på objektive datakilder, hvilket svarer til sensor data, brugsdata etc.
Alt sammen data der kan indsamles uden brugerinteraktion.

Fitness trackeren som metafor planter tanken om en applikation der løbende evaluerer ens præstationsevne, hvilket kan oversættes til mentalt helbred.

F16 fly metaforen henvender sig til platforms designet, der er tiltænkt at være en meget modulær og kraftig platform, ligesom det er tilfældet med F16 flyet hvor man kan hægte en lang række komponenter på alt efter hvad der er brug for i den pågældende situation.
%\section{Udløsning af Projektet}
% Essence chapter 4.
Det er vigtigt at vide hvad der udløste projektet, altså hvad motiverede projektet. 
Det kan f.eks. være at der var et bruger behov for det, at en teknologisk mulighed opdages, at der var en løsnings mulighed hvor man 'genbruger' en udviklet løsning i et andet domæne og konkurrence stress hvor man vil prøve at få en konkurrencemæssig fordel. 

Det der udløste vores projekt var `bruger behov'.
Grunden til det kan klassificeres som brugerbehov er, at en person der arbejder med problemområdet kom med ideen til dette projekt. 
Det behov der er i problemområdet ligger i at personer med affektive lidelser har brug for en mere moderne tilgang til tilstands overvågning end et ugentlig møde med et psykolog der i løbet af denne periode skal finde ud af om de har det godt.

Det skal dog også siges at projektet har et kraftigt grundlag i 'teknologisk mulighed', da ideen til projektet er baseret på en teknologisk mulighed med smartphone som platform.
Denne mulighed er, at smartphones kan indsamle mange varierende datatyper, der muligvis kan bruges til at give et indblik i en persons sindstilstand.

\section{Kriterier}
Her vil de vigtigste kriterier, som skal opfyldes for at kunne kalde projektet en succes, blive præsenteret.
Disse kriterier dækker over to overordnede områder: i forhold til bruger, og i forhold til platformen i sig selv.

\subsection{Bruger}
\begin{description}[style=nextline]
	\item[Sikkerhed] 
		Eftersom systemet kan komme til at arbejde med person-følsomme data, er det vigtigt at denne data opbevares sikkert, så applikationer der ikke har lov til at se ens data ikke har mulighed for dette.
	\item[Stabilitet]
		For at sikre der ikke opstår for mange huller i den indsamlede data, skal systemet køre stabilt, så der kontinuert kan indsamles data.
	Hvis der opstår for mange huller, kan dette give et ufuldstændigt billede af patientens sindstilstand, som måske kan fejl-fortolkes.
	\item[Præcision]
		Ligesom ved huller i indsamlingen af data over tid, er det lige så vigtigt at data der kommer ind er præcis.
	Dette grundes at analyse på upræcis data vil give et upræcist billede af patientens sindstilstand, hvilket kan føre til fejlagtige vurderinger af denne.	
\end{description}

\subsection{Platform}
\begin{description}[style=nextline]
	\item[Fleksibilitet]
	Det skal være nemt at modificere funktionalitet til platformen, da platformen skal kunne tilpasses til forskellige individer.
	\item[Mulighed for Udvidelse]
	Det skal være muligt at udvide platformens funktionalitet, uden at skulle ændre på selve platformen.
	Det skal gøres på sådan en måde at personer der ikke er en del af systemet, kan lave og tilføje egne dele til platformen.
	Dette giver naturligvis yderligere overvejelser ift. sikkerhed.
\end{description}

\subsection{Evaluering af Kriterier}
For at evaluere de forskellige kriterier skal


\mikael{Hvordan skal vi evaluere på de kriterier? Hvad kunne være godt at gøre? Hvad har vi egentlig tænkt os at gøre?}
\bruno{Igen +1 til Mikael - damn it..}
\lasse{Mikael skriver dette, det sagde han at han ville d. 2015-05-19 kl 12:45}
% A description of use context and selected use scenarios (see Essence-book Chapter 13).


\section{Paradigm view}
I dette view undersøges hvordan \textit{the Challenge} bliver set fra brugerens perspektiv.
Teknikker til denne undersøgelse inkluderer at definere applikationens problem domæne ved hjælp af en \textit{Use context} og \textit{Use Scenarios}.

\paragraph{Stakeholders}
I dette projekt er patienterne den vigtigste \textit{stakeholder}, da det er patienterne der skal bruge applikationen i hverdagen.
Applikationen skal derfor udvikles på patienternes præmisser.
Sikkerheden skal være så patienterne er trygge ved at gemme sine personline oplysninger i applikationen, og brugerfladen skal udvikles så patienter med forskellige grader af lidelser kan anvende den.

På den anden side er behandlerne og læger som skal kunne få fat i det data der er nødvendigt for at kunne behandle patienten og hjælpe med patientens sygdomsforløb.
Data skal altså være muligt at trække ud på en måde som behandlere og læger kan arbejde videre med.\als{Er dette et krav? eller bare en option.}

\paragraph{Use context}
Konteksten som applikationen skal kunne bruges i er meget bred, da det omfatter hele patientens hverdag.
Der skal derfor tages højde for at GPS signal ikke altid er tilgængeligt, og at data forbindelse ikke nødvendigvis er tilgængelig hele tiden.
Da applikationen skal logge data om patientens færden skal der håndteres at telefonen kan være i lommen, i hånden og ikke mindst på et bord eller i en jakkelomme der hænger i entreen.


\paragraph{Technologies}
Teknologier der benyttes inkluderer en smartphone og wearables der kan bruges sammen med en smartphone.
Disse kan være ure, armbånd og ørestykker.

\paragraph{Problems and needs}
\stefan{taget fra s. 83. Ved ikke om vi skal skrive noget her - fra essence - bare intern kommentar}

\paragraph{Use Scenarios}
\textit{Use scenarios} bruges til at udforske ideer og muligheder i forhold til brugerens brug af systemet.

Scenarier:
\begin{itemize}
	\item Patienten bevæger sig rundt i sin hverdag med telefonen i lommen. 
	Data logges i systemet om gemmes til at kunne blive analyseret.
	\item Patienten får en notifikation af systemet der beder patienten svare på et spørgsmål omkring patientens søvn.
	Patienten svarer på spørgsmålet og fortsætter sin aktivitet.
	\item Patienten får en notifikation af systemet der beder patienten svare på et spørgsmål omkring patientens søvn.
	Patienten har ikke lyst til at svare på spørgsmålet nu, og udsætter det til senere.
	\item Patienten vil gerne have applikationen til at fortælle hvordan den vurderer hans tilstand.
	Applikationen viser at patienten udviser normal adfærd.
	\item Patienten vil gerne have applikationen til at fortælle hvordan den vurderer hans tilstand.
	Applikationen viser at patientens sindstilstand er lavere end normalen.
	Patienten konsulterer sin liste af lystbetonede aktiviteter og udfører en af disse.
\end{itemize}
\stefan{Skal kigges på i fællesskab}

% A discussion of how to implement support for key use scenarios (see Essence-book Chapter 14).

\stefan{A discussion of how to implement support for key use scenarios (see Essence-book Chapter 14)}

\chapter{Indsamling af data}
Dette afsnit detaljerer hvilke kilder af data som kan bruges til dette projekt, hvilke devices der kan give adgang til de kilder, hvordan man logger på en mobil, og hvordan kilder af data kan bruges.
\chapter{Indsamling af data}
Dette afsnit detaljerer hvilke kilder af data som kan bruges til dette projekt, hvilke devices der kan give adgang til de kilder, hvordan man logger på en mobil, og hvordan kilder af data kan bruges.
\section{Sensorer}
some awesome intro

\subsection{Kamera}
\section{Enheder}\label{sec:kilder-til-sensorer}
Dette afsnit beskriver de forskellige typer af enheder udstyret med sensorer, således disse enheder kan agere kilder til dataindsamling.

De overordnede typer af enheder er \textit{smartphones}, \textit{smart wristband} og \textit{smartwatches}. 

\paragraph{Smartphone}
er en generel platform til et meget bredt brug og har derfor typisk et væld af sensorer, som fx. GPS, accelerometer og gyroskop.
Disse bruges til forskellige funktioner i telefonen, både internt og eksternt, fx. kan gyroskopet bruges til at bestemme orienteringen af smartphonen, hvor GPS kan bruges til navigation.

\paragraph{Smart wristband} % Skal det hedde dette?
er et stykke udstyr som bruges til aktivitetssporing, primært af fitness og helbredsgrunde, hvor de bl.a. bruges til at måle aktivitetsniveau og søvnmønster.
Denne aktivitetssporing påkræver en mængde sensorer, disse kan være de sensorer der findes i smartphones, men også sensorer der måler direkte på kroppen, fx. sensorer der kan måle puls eller galvanisk hud respons. 
Dog varierer det meget hvilke sensorer der findes i de forskellige smart wristbands, afhængigt af prisklasse.
Et eksempel på dette kunne være en JawBone UP3 hvor man finder sensorer der måler temperaturen af omgivelserne og kroppen, puls og galvanisk hud respons \citep{misc:jawboneup3sensors}. 
Dette er i kontrast til andre smart wristbands hvor der f.eks. ikke kan findes en sensor til at måle galvanisk hud respons.

\paragraph{Smartwatch}
er en intelligent version af normale armbåndsure, i den forstand at et smartwatch giver ekstra funktionalitet som minder om det der findes i en smartphone \citep{msic:smartwatchstate}. 
Som eksempel på dette kan man spille spil på dem, læse SMS og bruge den som medie-fjernbetjening, dog er smartwatches mindre kraftfulde end smartphones. 
Smartwatches og wristbands har fordelen at det er noget man går rundt med og derfor har direkte kontakt til kroppen det meste af tiden.
Endvidere har smartwatches mange af de samme sensorer som en smartphone og et smart wristband, ofte har de ikke sensorer til kropslige målinger, hvor de så primært fungerer som en hybrid mellem en smartphone og et smart wristbands.

\subsection{Opsummering}
Smartphones er alsidige platforme der kan bruges til at køre applikationer.
De indbyggede sensorer gør dem til en attraktiv platform at udvikle på, men smart wristbands og smartwatches har mere specialiserede sensorer som bedre understøtter logning af relevant helbredsdata.
På baggrund af dette ses det at smartphones kan bruges til meget, men hvis man vil have kropslige målinger er det en god idé at enten bruge et smart wristband eller et smartwatch med de relevante sensorer.

I \cref{tab:sensorsInDevices} er der et overblik over hvilke sensorer der findes i de forskellige slags udstyr. * indikerer at de kan findes i den slags udstyr, men det er ikke særlig tit at man finder det.

\begin{table}[h]
\centering
\begin{tabular}{|c|c|c|c|}
\hline  			 & Smartphone 	& Smart wristbands 	& Smartwatch	 	\\ 
\hline Accelerometer &  \checkmark 	& \checkmark		& \checkmark  		\\ 
\hline Gyroskop		 &	\checkmark	& \checkmark		& \checkmark		\\
\hline Kompas		 &  \checkmark	&					& \checkmark		\\
\hline GPS			 &	\checkmark	&					& \checkmark*		\\
\hline Barometer	 &	\checkmark	&					& \checkmark		\\
\hline Rumtemperatur &				& \checkmark*		&					\\
\hline Hudtemperatur &				& \checkmark*		& \checkmark		\\
\hline Lyd			 &	\checkmark	&					& \checkmark		\\
\hline Lys			 &	\checkmark	& \checkmark*		&					\\
\hline Kamera		 &	\checkmark	&					& \checkmark*		\\
\hline Puls			 &				& \checkmark		& \checkmark*		\\
\hline GHR			 &				& \checkmark*		& \checkmark*		\\ \hline
\end{tabular}
\caption{Overblik over sensorer i forskellige platforme.}\label{tab:sensorsInDevices}
\end{table}

\section{Mobil logning}
Dette afsnit fortæller om de forskellige ting der er mulige at logge på telefonen, som er interessante for projektet.

\paragraph{Opkald}
Opkaldsoversigten i Android giver adgang til opkalds historikken.
Applikationer skal dog have adgang til dette.
Et opkald består af tidspunkt, modtager, afsender og varighed.

\paragraph{Applikationer}
Det er muligt at logge hvilke applikationer der bliver brugt ved at se hvornår applikationerne er aktive.

\paragraph{Skærm}
Android har events for hvornår en skærm tændes og slukkes.

\paragraph{Tastatur}
Logning af tastaturet er ikke muligt, da man ikke kan få adgang til det gennem en applikation.

\paragraph{SMS}
Det er muligt at få adgang til SMS beskeder, det kræver dog tilladelse fra brugeren til applikationen.
\section{Idéer}
Dette afsnit bruger symptomer/kriterier for mani og depression beskrevet i \cref{} og ser på hvilke sensorer eller hvilke mobile lognings metoder der kunne bruges til at løse problemet.

\subsection{Depression}

\subsubsection{Kernesymptomer}
Kernesymptomerne kunne løses på følgende måder:

\paragraph{Man er i dårligt humør og er nedtrykt og trist}
Her er det en mulighed at bruge kameraet til at tage billede af patientien. Disse analyseres derefter og der kigges på hvilket humør en person er i. Denne idé er præsenteret i brochuren i \cref{app:brochure}.
\\
\\
Under behandlingen bruges der et skema der hedder stemningsregistering(\cref{sec:moede-med-psykolog}). Dette skema kunne bruges til at interagere med patienten, for på den måde at finde ud af hvordan vedkommende har det. Disse data kan dog manipuleres af patienten. Dog kan en hyppigere eller mindre brug af stemningsregistrerings funktion indikere mani eller depression.
\paragraph{Man har nedsat lyst til at foretage sig noget, og man har mere eller mindre mistet interessen for ting man plejer at interessere sig for}
Hvis ting man plejer at have lyst til har en anden lokation end ens hjem, kan man vha. GPS'en se om man er der. Der kunne indkodes nogle nøglelokationer, hvor nøglelokationer er lokationer hvor man foretager sig ting man har lyst til.
Hvis disse nøglelokationer så ikke bliver besøgt kan dette være et tegn på en depression.
\\
\\
Hvis man normalt laver ting man har lyst til i hjemmet er der en anden måde at afsløre en adfærdsændring. Man kunne evt. identificere forskellige aktiviteter i hjemmet vha. accelerometeret og gyrosensoren. For hvis der er en ændring i data, kunne dette vise en adfærdsændring. Det kræver dog sandsynligvis meget data.
\\
\\
For at hjælpe behandlingen kunne man implementere en mobil løsning for huskekortet som Janne nævnte under mødet. Man kan bruge notifikationer til at huske patienten på at vedkommende skal huske at lave nogle ting de engang have lyst til.

\paragraph{Man bliver hurtigt træt og har ikke så meget energi som man plejer}


\chapter{Platform}
For at kunne udvikle et system der spænder bredt og har rig mulighed for senere udvidelser, er det nødvendigt at etablere en platform hvorpå denne udvikling kan foregå.
En sådan beskrivelse af platformen kan, for fremtidige udviklere, fungere som en slags opskrift i forbindelse med eventuelle udvidelser.
Det følgende kapitel giver en beskrivelse af den udviklede platform; herunder arkitekturen og dens komponenter.
\section{Motivation} %Als' ide - et slags bindeled mellem konfigurationstabel og platform
%Udfordringer
%   Generalitet
%   Fleksibiletet
        %videre udvikling
	    % enkelt komponenter
	    % samspil

%fleksibilitet
Da vi har fundet at symptomerne på affektive lidelser varierer meget fra person til person er vi nødt til at have en platform der understøtter denne tankegang.
Som nævnt i konfigurationstabellen har vi som vision at platformen skal være modulær og fleksibel som et F-16 fly, hvor man kan hægte moduler på så de passer til den enkelte patient.
Vi sigter efter en fleksibel platform i flere henseender.
Vi ønsker en åben platform der muliggøre at udviklere nemt kan udvikle moduler der kan benyttes af platformen.

Fleksibilitet tænkes også i den sammenhæng der er udskiftelighed, fusionering af flere moduler. Forstået på den måde at et givent modul kan komme med et estimat for eksempelvis søvn, så kan et andet modul også gøre dette, og platformen understøtter at et givent modul kan vælge og vrage hvilken kilde ønskes brugt.
Det er ikke begrænset til søvn, men flere kilder der kan være tilgængelig til samme datatype.

Platformen skal desuden understøtte kommunikation mellem moduler så data fra et modul nemt og sikkert kan deles med andre moduler.

%generalitet
Derudover har vi begrænset tid, men der er en stor mulighed for at lave et system der understøtter registrering af en lang række forskelligartede symptomer.
Af den grund er det helt centralt at platformen udvikles til at være så generel som mulig, for ellers risikerer vi at udelukke fremtidig udvikling og brug af systemet med hensigt i andre symptom registreringer.

Det første centrale krav til platformen er altså at platformen skal være fleksibel i forhold til moduler der kan hægtes på platformen.
Det andet centrale krav til platformen er at platformen skal være så generel at vi ikke begrænser fremtidig udvikling.
\section{Eksisterende Mønstre}
Ved implementeringen af platformen vil det være fordelagtigt at anvende et eksisterende mønster til at beskrive systemets arkitektur.
Ved at anvende et eksisterende mønster vil mange strukturelle udfordringer kunne undgås.
Herunder beskrives eksisterende arkitekturmønstre og der foretages en vurdering af hvorvidt mønstrene kan anvendes, givet behovene beskrevet i \cref{arkitekturkrav}.
\mikkel{Hvor har vi netop de her mønstre fra og hvordan ved vi at der ikke findes andre mønstre der passer bedre?}

\subsection{Client-Server}
En af standard arkitekturerne i mobil udvikling er \textbf{Client-Server}, hvilket kunne være hvor web sider kan bruges til at f.eks. visualisere data, men dette har et problem idet at sensor data indsamling vil nødvendigvis påkræve implementering adskilt da det er data som er interessant.
Denne arkitektur har dette problem at det ikke er nemt for udefrakommende udviklere at tilføje funktionalitet, det kan desværre også være svært at implementere dele som indsamler, analysere og viser data idet at alt data ville ligge lokalt og skal nødvendigvis sendes til serveren før den kan bruge denne data til noget. 
Det har dog fordelen at det kan køre på Android gennem en almindelig webbrowser eller en applikation som bruger WebViews.
Baseret på disse ville denne arkitektur nok ikke fungere.
\mikkel{Vi mangler en beskrivelse af selve arkitekturen.}
\stefan{kilde?}
\subsection{Standalone}
En anden arkitektur ville være en \textbf{standalone} applikation som indeholder alt sensor indsamling, analyse og visualisering i samme applikation.
Dette har fordelen at det er nemmere at programmere end 15 forskellige applikationer, men kommer med ulempen at applikationen kunne ende med at blive meget stor.
Det er selvfølgelig muligt for udefrakommende udviklere at implementere ny funktionalitet til denne hvis de har fri adgang til kildekoden, men dette vil nok danne med problemer med Google Play Store idet at hvis 30 forskellige udviklere tilføjer ny funktionalitet til applikationen ville dette påkræve at hvis denne skal bruges af flere ville der være flere versioner af den samme applikation uploadet af forskellige udviklere og dette gør det på samme tid svære at udvikle videre på andres arbejde da alting er samlet i en applikation og ikke over flere.
På grund af dette ville denne arkitektur ikke gøre det let at implementere og bruge ny funktionalitet.
Denne arkitektur gør det muligt at implementere forskellige dele til applikationen som indsamler, analysere og viser data.
Da det ikke er let at tilføje ny funktionalitet afvises denne arkitektur. 
\mikkel{Det her afsnit skal skrives om - det er noget rod. Har vi en kilde på det eller er det bare fri fantasi? :P}

Baseret på disse antages det at arkitekturen der er blevet valgt er den mest åbenbare og at hvis der er andre arkitekturer kender vi ikke til dem.
\mikkel{Det giver ingen mening. Arkitekturen er den åbenbare?
Og 'hvis der er andre arkitekturer kender vi ikke til dem'?? Det svarer til 'Hvis det vi har lavet er skidt er det fordi vi er uvidende.'}
\stefan{kilde?}

For at kunne imødekomme et bredt spektrum af lidelser samt et bredt udvalg af forskellige, Android baserede, smartphones kræves en modulær arkitektur hvori moduler kan fjernes, tilføjes og ændres uafhængigt af det overordnede system.

Moduler inddeles i tre konceptuelle lag: \textit{sensor}-, \textit{analyse}- og \textit{visnings}moduler.
Denne inddeling vil tillade adskilt udvikling af enkelte moduler uanset hvilket lag de tilhører.
Ved at kombinere de forskellige moduler på tværs af lagene, opnås et fungerende system.

Eksempelvis vil et accelerometer \textit{sensor}modul kunne anvendes af et søvn \textit{analyse}modul der viser resultatet i et graf \textit{visnings}modul.
Skulle man her ønske en anden fortolkning af accelerometerets data kan et andet analysemodul anvendes.
Sammensætningen af analyse- og visningsmoduler sker ved en beskrivelse af deres grænseflade.
Moduler med fælles grænseflade vil kunne kombineres.
Det vil sige at der kan anvendes flere forskellige visningsmoduler til det samme analysemodul, givet at de er kompatible.
Tilsvarende kan det samme visningsmoduler anvendes til flere forskellige analysemoduler.
\textbf{Det har ikke været muligt at finde et umiddelbart arkitektur-mønster for dette.}
\stefan{vi bør nok udvide dette, det virker lidt tyndt at skrive}
\ivan{Enig. Check lige
	Gamma et al om
	I har overset noget.
	Samme om Fowler.
	HVIS der ikke
	findes mønstre, så
	tag nogle, der
	næsten gør, og
	forklar hvorfor de
	ikke kan bruges
	alligevel.}

\section*{Opbygning}
Den overordnede arkitektur er opbygget af fire komponenter: \textit{manager}, \textit{moduler}, \textit{DB access} og \textit{DB}.\ivan{evt. to ord om hver komponents formål}
En skitse af arkitekturen kan ses på \cref{arkitektur_udkast_1}.
\begin{figure}[h]
	\includegraphics[width=\textwidth]{architecture_draft}
	\caption{Første udkast til arkitektur.}
  \label{arkitektur_udkast_1}
\end{figure}
Som nævnt herover er styrken ved denne arkitektur, den modulære opbygning af sensorer, analyser og visninger.
Disse lag er derfor alle indholdt i den overordnede komponent \textit{moduler}.
Derudover eksisterer et system, bestående af de resterende tre komponenter, der anvender de moduler der er installeret på den enkelte telefon. \ivan{kompakt og lidt klodset formulering}
Systemet er altså opbygget uden \ivan{evt. to ord om hver komponents formål} til de enkelte moduler, hvilket igen tillader udviklingen af moduler sideløbende med det overordnede system.
\ivan{Information hiding? Lidt tyndt her}

Herunder gives en kort beskrivelse af hver af de fire komponenter.
Komponenterne beskrives i rækkefølge af deres indbyrdes afhængighed, således at forståelsen af hver komponent kun afhænger af det læste.

\ivan{Måske var det en ide at bruge en skabelon for gennemgangen af komponenterne f.eks. (1) Formål - (2)design, og - (3) Perspektiv for fremtidig udvikling - På den måde kunne I understrege generaliteten og fremtidsikringen i jeres design}
\subsection*{DB}
Denne komponent administrerer data for systemets forskellige moduler.
Data opbevares i en række tabeller i et relationelt database system.
Hvert modul har mulighed for at definere egne tabeller, der alle gemmes i \textit{DB} komponenten.

Til dette projekt er valgt en SQLite database da denne er standard i Android.

\subsection*{DB Access}
Denne komponent styrer adgangen til \textit{DB} komponenten så det sker på en ensrettet måde.
Derudover skal \textit{DB Access} også sørge at et komponent kun kan skrive til sine egne tabeller, men have mulighed for at læse fra de tabeller som den er afhængig af.
Dette er dog ikke muligt da \textit{DB Access} bruger en \textit{ContentProvider}, men her er det ikke muligt at se hvad for en anden applikation der tilgår \textit{DB Access}, hvilket gør at denne begrænsning ikke er mulig. \ivan{Klodet og redundant sprog}

Systemet anvender udelukkende en database placeret på selve mobiltelefonen og er dermed begrænset af de muligheder der er for lagring på den enkelte enhed.
I en fremtidig udvidelse af systemet kunne \textit{DB Access} komponenten tilgå et ekstern lager hvor dele af det administrerede data kan gemmes.
Denne abstraktion forventes at kunne påføres \textit{DB Access} uden at kræve ændringer i de resterende komponenter.

Muligheden for en sådan udvidelse af systemet er ikke undersøgt og der kan derfor naturligt være komplikationer herved, ligesom udvidelsen ikke med sikkerhed kan laves uden påvirkning af de resterende komponenter.

\subsection*{Moduler}
Denne komponent består af de tre forskellige modul-lag.
Til hvert enkelt modul hører en modulbeskrivelse (se \cref{modul_definition}) der beskriver modulets afhængigheder af andre moduler samt hvordan det skal administreres i systemet.
Denne administration består til dels i en definitioner af de tabeller modulet har behov for at få oprettet i \textit{DB} komponenten.

Bemærk at de tre lag i modul-komponenten består af udskiftelige moduler og at lagene derfor kun eksisterer konceptuelt. \ivan{Dette forstår jeg ikke helt}
Nedenstående beskriver derfor de forskellige moduler der \textit{kan} ligge i hvert af lagene.
Beskrivelserne er til dels intentioner for modulerne i hvert lag og omfatter altså ikke alle moduler der kan udvikles til systemet.

\paragraph{Sensor}
\textit{Sensor}laget indeholder moduler der indsamler data fra telefonens (eller tilbehør dertil) forskellige sensorer og applikationer.
Der påføres kun et minimum af behandling på indsamlede data (eksempelvis komprimering) således at data kan indsamles kontinuert uden stort energi-behov.
Et sensor modul bør ikke smide indsamlede data ud.
Det vil sige at komprimering bør være tabsfri og at eventuelle fejlmålinger bør markeres som sådan i sensorens data tabel.
\mikkel{Beskriv et eksempel på et sensor modul, når vi har flere detaljer om et.}
\ivan{Betyder det, at data lagres decentralt ved hver sensor? Hvorfor dete valg? evt. henvise til senere i kapitlet}

\paragraph{Analyse}
\textit{Analyse}laget indeholder moduler der bruger data fra et antal sensormoduler samt eventuelle andre analysemoduler.
Herefter udføres en analyse af det indsamlede data med \textit{''forståelig information''} som resultat.
I denne process vil der kunne forekomme tab af data.
Herved opnås en opsummering af det indsamlede data, der skaber værdifuld information for brugeren.
Som en del af et analysemoduls beskrivelse findes en beskrivelse af hvilken information man kan få fra modulet.
Denne information anvendes af visningsmodulerne.

Der kan ved analyse af sensor data anvendes flere ressourcer, da analysen typisk vil kunne udføres på større mængder indsamlede data få gange dagligt.
Eksempelvis kan analysen foretages om natten hvor telefonen kan sættes til opladning.
\mikkel{Beskriv et eksempel på et analyse modul, der anvender ovenstående sensor modul, når vi har flere detaljer om et.}

\paragraph{Visning}
\textit{Visnings}laget indeholder moduler der visualiserer analyserede data.
Som en del af et visningsmoduls beskrivelse findes en beskrivelse af hvilken information modulet accepterer.
Hvis denne beskrivelse stemmer overens med et analysemodul, kan den enkelte visning anvendes på den enkelte analyse.
\mikkel{Beskriv et eksempel på et visnings modul, der anvender ovenstående analyse modul, når vi har flere detaljer om et.}

\subsection*{Manager}\label{subsec:arkitektur-Manager}
Manager komponenten kan siges at være grænsefladen mellem bruger og moduler.
Den står for at administrere de installerede moduler ud fra de beskrivelser der er givet for de enkelte moduler.
Denne administration indebærer blandt andet oprettelse af de tabeller hvert modul har i sin beskrivelse, samt start og stop af sensor- og analysemoduler.
Sidstnævnte sker ud fra definitioner givet i beskrivelserne af de enkelte moduler.

Ved at sammenholde visninger og analyser kan manageren beskrive for brugeren hvilke data der kan fremvises og med hvilke visninger det kan ske.
Manageren har desuden en prædefineret brugerflade der anvender ovenstående kombinationer til at vise brugeren de relevante informationer.

Manager komponenten indeholder desuden et JSON skema for hver modul-type i moduler komponenten.
Disse definitioner beskriver formatet for eventuelle nye moduler man måtte ønske at føje til systemet.
\stefan{et eller flere skemaer? Muligvis har views anderledes skema, men analyse og sensorer har det samme}

\subsection{Diskussion af Valg}
Et af spørgsmålene der skal stilles er om dette valg af arkitektur dækkende og om der ikke er alternative som kunne have blive brugt.
En af grund idéerne til platformen er at det skal være let for udefrakommende udviklere at tilføje funktionalitet. 
Det skal være muligt at implementere dele til systemet som indsamler, analysere og viser data.
Baseret på Android's åbenhed, brugsniveau og dokumentation er det blevet valgt at platformen skal køre på Android.

Men er der alternativer? En af standard arkitekturerne i mobil udvikling er Client-Server, hvilket kunne være hvor web sider kan bruges til at f.eks. visualisere data men dette har et problem idet at sensor data indsamling ville nødvendigvis påkræve implementering adskilt da det er denne data som er interessant.
Denne arkitektur har dette problem at det ikke er nemt for udefrakommende udviklere at tilføje funktionalitet, det kan desværre også være svært at implementere dele som indsamler, analysere og viser data idet at alt data ville ligge lokalt og skal nødvendigvis sendes til serveren før den kan bruge denne data til noget. 
Det har dog fordelen at det kan køre på Android gennem en almindelig webbrowser eller en applikation som bruger WebViews.
Baseret på disse ville denne arkitektur nok ikke fungere.

En anden arkitektur ville være en standalone applikation som indeholder alt sensor indsamling, analyse og visning i samme applikation.
Dette har fordelen at det er nemmere at programmere end 15 forskellige applikationer, men kommer med ulempen at denne ene applikation ville være meget stor.
Det er selvfølgelig muligt for udefrakommende udviklere at implementere ny funktionalitet til denne hvis de har fri adgang til kildekoden, men dette vil nok danne med problemer med Google Play Store idet at hvis 30 forskellige udviklere tilføjer ny funktionalitet til applikationen ville dette påkræve at hvis denne skal bruges af flere ville der være flere versioner af den samme applikation uploadet af forskellige udviklere.
På grund af dette og af andre grunde ville denne arkitektur ikke gøre det LET at implementere og bruge ny funktionalitet.
I denne er det muligt at implementere forskellige dele til applikationen som indsamler, analysere og viser data.
Denne kan godt implementeres på Android, så til dette ville der ikke være noget problem.
Da det ikke er let at tilføje ny funktionalitet afvises denne arkitektur. 
\section{Komponenter}
Som beskrevet i \cref{arkitektur:opbygning} består systemet af fire komponenter.
Hvert af disse beskrives i yderligere detaljer i de følgende afsnit.
\subsection{Manager}
Manageren er den eneste komponent brugeren interagerer med.
Det er igennem manageren at brugeren vælger hvilke moduler, der skal køre.
Det er også den, som sørger for at de rigtige moduler kører på de rigtige tidspunkter.
Dette gøres ved hjælp af \texttt{TaskRunner}.
Derudover er det også manageren, der sørger for at skaffe moduldefinitionen fra alle de installerede moduler.
Ydermere, giver manageren et overblik over de visualiseringsmoduler, som er installeret samt have en måde at vise visualiseringsmodulerne på.
\subsubsection{Kørsel af Moduler}
Der er overordnet to måder hvorpå moduler køres; enten styrer de selv deres kørsel (kontinuerte kørende moduler) eller så administreres de af Managerens \texttt{TaskRunner}, som er en \texttt{Service} der startes når mobilen tændes.
Både sensor- og analyse-moduler kan køre på disse to forskellige måder.

\paragraph{Kontinuerligt kørende moduler}
Til moduler der skal køres kontinuert, startes deres \texttt{Service} blot lige så snart modulet aktiveres i indstillinger, hvorefter det selv administrerer hvornår og hvor ofte det udfører dets opgave.
Fordelen ved dette er at vi sparer kommunikations-overhead, da der ikke konstant skal kommunikeres mellem Manager og kontinuert-kørende moduler.

\paragraph{TaskRunner}
Derudover er der også moduler som kun skal køres med faste, større, intervaller, eller på bestemte tidspunkter.
Her er der ikke behov for at det enkelte modul har en kontinuert kørende \texttt{Service}, men kan derimod nøjes med at Manageren starter modulets opgave på det korrekte tidspunkt.
På denne måde spares der ressourcer, da modulet kun aktiveres i den tid hvor det skal udføre sin opgave. 

Når \texttt{TaskRunner}en startes, danner den en liste over de moduler der skal køres med interval eller på fast tidspunkt.
Derefter laver den en prioriteret kø, sorteret efter næste kørsels-tidspunkt.
\texttt{TaskRunner} tråden \texttt{sleep()}es så, indtil næste opgave skal udføres.
Efter opgaven er udført, udregnes næste kørsels-tidspunkt for den netop kørte opgave, hvorefter prioritets-køen sorteres.
Så \texttt{sleep()}es tråden igen, indtil næste kørsels-tidspunkt, og dette fortsætter så længe der er aktive moduler der skal køres på denne måde.

\subsubsection{Indstillinger}\label{sec:settings}
\chapter{Indstillinger Arbejdsblad}
%guidelines
For at få en velkendt og standardiseret brugergrænseflade fulgtes android design guidelines.
Disse guidelines angiver hvornår man skal bruger diverse knapper, actionbars, settings etc.
\winde{Kilde til androids guidelines}

%Prototypes
Ved at følge disse guidelines blev en række prototyper for indstillinger lavet.
Disse byggede på samme princip om at udarbejde en indstillingsmenu.
Der var diskussion om hvordan disse skulle være, men over flere iterationer valgtes der at gå fra en "wizard" tilgang til en regulær settings menu,
Billeder af diverse prototyper kan ses i \cref{fig:prototype-manager}

\begin{figure}
	\centering
	\begin{subfigure}[b]{0.45\textwidth}
			\includegraphics[scale=0.3, page=1, trim = 1cm 5.5cm 1cm 0cm, clip]{prototype.pdf}
			\caption{Forside}
	\end{subfigure}
	\begin{subfigure}[b]{0.45\textwidth}
			\includegraphics[scale=0.3, page=2, trim = 1cm 5.5cm 1cm 0cm, clip]{prototype.pdf}
			\caption{Indstillinger}
	\end{subfigure}
	\caption{Prototype af Manager}
	\label{fig:prototype-manager}
\end{figure}


%Actionbar
Ud fra prototypen kan en actionbar blandt andet ses, tanken er at følge et standard design hvor man har en actionbar i toppen.
Denne muliggør navigation til indstillinger, men også at gå tilbage til hovedmenuen, nuværende er der dog problemer med at den ikke vises under settings\als{Skal have lavet mere arbejde med den}

%Indstillinger, checkbox
Til at angive om et givent modul skal være aktiveret eller ej bruges checkboxes.
Dette skyldes at det er et simpelt ja/nej valg. 
Tanken er så at de moduler man har valgt er dem der kører på telefonen.

%indstillinger, dependencies og events
For at scanne mobilen for de moduler der er installeret bruges JSONParser der tager vare af dette. \als{referer til JSONParser}
Dette giver udslag i en række moduler der har afhængigheder af andre moduler og skal takles.
JSONParseren giver som resultat en liste af moduler. Disse scannes så igennem for at finde deres afhængigheder.
Disse afhængigheder bruges så til at konstruere events (OnChange) til at fortælle de moduler der skal have besked når et givent modul aktiveres/deaktiveres.
Ved at lave en sådan række er der implicit konstrueret en dependency graph.
Og som resultat af dette kan man forestille sig et hierarki hvor et modul på det lavest liggende niveau medfører en kæde af deaktiveringer af moduler der eksplicit og implicit afhænger af dette modul.

%indstillinger, onPause start og stop sensorer
Efter afhængighederne er enkodet i programmet mangler der at takle hvordan sensorer skal startes og stoppes fra indstillingsmenuen.
Til at takle dette bruger vi den allerede udviklede ServiceHelper\als{referer til denne}.
Der vælges så at starte og stoppe de fornødne sensorer i onPause, da det typisk er når man forlader en indstillingsmenu at man gerne vil have at indstillingerne træder i kraft, og sikrer også at man ikke skal klikke på ekstra knapper for at indstillingerne træder i kraft.
\subsubsection{JSON-parser}\label{subsub:JSONparser}\mikkel{Find på bedre navn}
JSON-parserens job er at finde JSON filen for hvert eneste modul installeret på smartphonen, hvorefter for hver modul bliver lavet et objekt ud fra den klasse lavet fra JSON-skemaet.
Måden dette er gjort på er ved brug af `jsonschema2pojo' \citep{jsonpojo}, der gør det muligt at få genereret klasserne som passer til JSON skemaerne, og som kan lave objekter baseret på de klasser og hvad der står i JSON filerne. 
\subsection{Data lag}
Dette afsnit detaljerer data laget i det overordnede system. 

\subsubsection{Database}
Det er nødvendigt at lagre data fra moduler så andre moduler kan bruge dataene.
I Android er det standard at data kun er tilrådighed til den specifikke applikation der har lagret det.
Applikationen gøre sin data tilrådighed for andre via en \texttt{ContentProvider}.

\subsubsection{Samlet i Manager eller i Hver Applikation}
Der er undersøgt to muligheder for gøre data tilgængelig.
Man kan samle data fra alle moduler i manageren da denne står for den overordnede kontrol af systemet. 
Manageren kan da definere et interface til databasen, som så moduler bruger til at læse eller skrive data. 
Den anden mulighed er at hvert modul direkte lagrer data i deres egen database, og så gør den tilgængelig for de moduler der vil bruge den.
Vi vælger at samle al data i manageren, da det stærkt simplificerer adgang til databasen, da hvis individuelle databaser var blevet brugt til hvert modul vil dette påkræve opsætning af databaser i hvert eneste modul.
Derudover simplificere det også backup og udvælgelse af data hvis det ligger samlet.

\subsubsection{Database Helper}
For at lave databasen i manageren skal der oprettes rettigheder til at kunne læse/skrive til databasen samt lave nye database tabeller.

\subsubsection{ContentProvider}
En content provider er en Android konstruktion der giver adgang til data for andre applikationer.
ContentProvider er en abstrakt klasse, hvor subklasser skal overskrive 5 metoder:  getType, query, insert, delete og update. 

Insert gør det muligt at skrive til databasen, query læser fra databasen, delete sletter og update opdaterer eksisterende data.

getType er en metode der benyttes til at angive MIME typen af data ContentProvideren giver adgang til.

I den nuværende implementering er det valgt kun at implementere insert og query da det er de eneste relevante funktioner i forhold til det behov vi har på nuværende tidspunkt, hvor vi kun er interesserede i at læse fra og skrive til databasen.

Delete kan blive relevant til brug ved oprydning af for gammel data. Da vi fokuserer på at analysere på mønstre i dataene kunne man eksempelvis slette sensor data der er ældre end 2 måneder. 
Alder på data til sletning kan afhænge af datatype, eksempelvis kunne man forestille sig at bevægelsesdata kan slettes efter kort tid mens sociale interaktionsdata skal leve længere.
\section{Modul-definition}\label{modul_definition}
\section{Modul Definition}

%Måde at lave eksterne modul-apps uden at ændre hoved-app
%Definere output (tabeller/kolonner), samt input (afhængigheder)
%Evt. konfigurationsmuligheder for moduler afhængige af det
%Fleksibel måde at modtage data fra andre moduler, uden at skulle opdatere app(s). Dvs. ikke design-mønstre: observer, mediator, men gennem content provider.
% modularisering i Android; multiple apps under samme package

For at gøre systemet mere fleksibelt, vil vi udtænke en modul-baseret arkitektur.
Det skal være muligt at tilføje eksterne moduler, uden at have behov for at lave ændringer i hoved-applikationen.
Dette kan gøre sig gældende når der kommer nye sensorer på markedet, eller hvis der skal laves nye former for visninger til det allerede indsamlede data.
Til dette er der valgt at bruge JavaScript Object Notation (JSON) samt JSON Schema \cite{json_schema}.
Eksemplerne der bruges herefter vil derfor være i henholdsvis JSON eller JSON Schema.

\subsection{Typer af moduler}
Der vil findes i alt tre typer moduler; \textit{sensor}, \textit{analyse} og \textit{view}.\footnote{Der bruger engelske begreber her for at holde det implementerings-nært.}
Sensor-modulerne repræsenterer de fysiske sensorer til stede i telefon eller tilsluttet wearable.
De leverer data som analysis eller view modulerne skal bruge til at henholdsvis behandle eller vise data.
Baseret på en eller flere sensor- eller analysis-moduler, kan et analyse modul levere behandlet data, til brug af andre analysis-moduler, samt view-moduler.
View-modulerne bruges til visning af den rå sensor-data eller den behandlet analysis-data.

Som minimum har et modul et navn og en version.
Sensor- og analysis-moduler vil have data, analysis- og view-moduler vil have afhængigheder.

\subsection{Data}
Sensor- og analysis-moduler skal gøre data tilgængeligt for andre analysis- og view-moduler.
For at specificere hvordan den data skal gemmes, samt hvad der er tilgængeligt for andre, skal dette defineres for hvert modul af førstnævnte typer.
For hvert modul skal der defineres en eller flere tabeller, med navn, da det er muligt for ét modul at levere mere end én slags data.
For hver tabel defineres en eller flere kolonner, med et beskrivende navn, samt datatyper og evt. en måleenhed.

\subsection{Afhængigheder}
Et analysis- eller view-modul kan være afhængigt af andre sensor- eller analysis-moduler, da det kan være de skal bruge en bestemt slags data for at være anvendelige.
Derfor skal det defineres for hvert modul hvilke andre moduler det er afhængigt af.
Dette kan gøre på to måder; hard- eller soft-dependency.
En hard-dependency er ét andet modul, samt version, som det pågældende modul ikke kan fungere uden.
En soft-dependency er en liste af andre moduler, hvor mindst ét af de listede moduler skal være til stede på enheden.

\subsection{JSON og JSON Schema}
For at have en modul-beskrivelse der er læselig for både mennesker og maskiner, er JSON valgt.
Det burde også være muligt for ikke-tekniske personer at læse, skrive og forstå et JSON dokument.
For at sikre validiteten af eksternt leveret modul-beskrivelser, udarbejdes der et JSON Schema, som JSON-dokumenter kan holdes op imod og derved verificeres.
Det anvendte JSON Schema kan findes i \cref{app:json_schema}.

\paragraph{Eksempel} på en modul-beskrivelse.
Al meta-data er præfikset med \_ (underscore).
\begin{lstlisting}
{
  "name": "accelerometer",
  "_version": 1.0,
  "tables": [
    { "name": "accelerations",
      "columns": [
        { "name": "accX",
          "dataType": "REAL",
          "_unit": "g" },
        { "name": "accY",
          "dataType": "REAL",
          "_unit": "g" },
        { "name": "accZ",
          "dataType": "REAL",
          "_unit": "g" }
      ]}]}
\end{lstlisting}

\subsection{Implementering}
Som nævnt i \cref{valg_af_android}, implementeres der til Android telefoner.
Dette sætter nogle begrænsninger ift. valg af løsninger.

\subsubsection{JSON}
JSON blev valgt over XML, da der er bedre native support for JSON på Android.
Dette er ikke normalt et problem, men da vi gerne ville have automatisk generering af Java klasser ud fra vores schema, og det ikke kunne lade sig gøre med XML, blev JSON i stedet valgt.
Der er ingen begrænsninger ved JSON, frem for XML, dog en anden syntaks.

\subsubsection{Moduler som apps}
For at det skal være muligt at installere moduler uden at opdatere hoved-applikationen, skal der installeres apps via Google Play Store.
Alle modul-apps, samt hoved-applikationen, deler \textit{package}-navn.
Hver modul-app har sin JSON beskrivelse som en eksternt tilgængelig \textit{resource}, som hoved-applikationen eller andre moduler har adgang til.

Kommunikation mellem apps foregår med \textit{services}, \textit{intents} eller \textit{content provider}.
Services og intents skal defineres i applikationens \textit{manifest}, og ville derfor skulle ændres hver gang der kom nye moduler til.
I stedet for benytter vi content provider til både at gemme data opsamlet af moduler, samt at gøre tilgængelig til andre moduler.

\subsubsection{Data typer}
De tilgængelige data typer tilgængelig for tabel-kolonner, er begrænset til dem som er tilgængelig i SQLite, som er den database der bruges på Android telefoner.
Der er 5 typer: \textit{NULL}, \textit{INTEGER}, \textit{REAL}, \textit{TEXT} og \textit{BLOB}.

\section{Visnings Modul}

Som udgangspunkt er \textit{visninger} forskellige visualiseringer af analyse modulernes output.

Det antages at analyse moduler kan og vil være meget forskellige, og at de enkelte visninger ikke nødvendigvis vil passe på alle analyser.
Derfor skal der være en måde at knytte de enkelte visninger med de passende analyser.
En måde at gøre dette på er at lave en baglæns afhængighed for en visning til en eller flere analyser, som beskriver hvilke analyser den pågældende visning kan repræsentere.
Problemet med dette er at denne liste, dvs. selve modul app'en, skal opdateres hver gang der oprettes nye analyser der passer sammen med visningen.
En anden måde er at have bindingen på analyse-modulerne, så der i stedet beskrives hvilke visninger der gør sig gældende for de enkelte analyser.
Her er problemet dog stadig det samme, at analyserne skal opdateres hver gang der kommer nye visninger.

En mere fleksibel måde at håndtere denne knytning mellem visninger og analyser er at deducere ud fra analysens tabel/kolonne signatur, hvilken slags data den outputter.
På denne måde kan der laves visninger der kan repræsentere bestemte signaturer, og der i stedet laves en implicit binding mellem visninger og analyser.

I mange tilfælde vil denne implicitte binding være nok, dog kan det forestilles at analyserne vil være forskellige og til tider komplekse, hvilket kan gøre det nødvendigt at nærmere specificere hvordan den tilgængelige data kan vises.
Her kunne der tilføjes noget meta-data på analysens tabeller/kolonner, der beskriver den på sådan en måde, at det ville kunne bruges til visninger der ikke nødvendigvis implicit kunne knyttes til det.

\subsection{Eksempel}
Her bruges light, som er en simpel sensor og analyse, der fra sensoren indhentes løbende lys-niveau (i lux).
Analysen tager den senest indsamlet data og giver et gennemsnit.
Det antages at alle tabeller har en tids-kolonne, der beskriver enten hvornår data blev indsat i databasen, eller i nogle tilfælde af analyser overføres tiden fra sensor-data.

\begin{lstlisting}
{
  "name": "lightAvg",
  "_version": 1.0,
  "tables": [
    {
      "name": "lightAvg",
      "columns":[
        { "name": "lightAvg", "dataType": "REAL", "_unit": "lux" }
      ]
    }
  ],
  "dependencies": [
    [{ name": "light" }]
  ]
}
\end{lstlisting}

Et eksempel på et visnigs-modul som passer på denne type analyse kunne være en simpel 2D graf, som viser den gennemsnitlige belysning over tid.

\begin{lstlisting}
{
  "name": "2dgraph",
  "_version": 1.0,
  "_type": "view",
  "view": {
    "layout": "2dgraph.xml",
    "data": [
      { "name": "x", "dataTypes": ["INTEGER"], "fromTimestamp": true },
      { "name": "y", "dataTypes": ["INTEGER", "REAL"] }
    ]
  }
}
\end{lstlisting}

Denne visning beskriver en 2D graf, som bruger layoutet i \texttt{2dgraph.xml}.
På x-aksen vises tiden (der står INTEGER fordi SQLite ikke har datorepræsentation ud over unix-time).
Her bruges det tidsstempel som forventes på alle tabeller.
Til y-aksen kan bruges alle analyser, som blot har en enkelt kolonne af heltal eller decimal-tal værdi.

\subsection{Administrering af visninger}
Ligesom sensorer og analyser, skal visninger også administreres af brugeren.

For at holde det så simpelt som muligt, kunne man, ligesom ved sensorer/analyser, have to niveauer af administration.
Som udgangspunkt vil alle installeret visninger automatisk knytte sig til de aktiverede analyser.
Derudover vil man i de avancerede indstillinger kunne aktivere/deaktivere visninger for de enkelte analyser.

På denne måde vil der automatisk blive genereret en liste af visning/analyse kombinationer, men med mulighed for at slå nogle af kombinationerne fra.
Det vurderes at muligheden for at slå visninger fra skal være under avancerede indstillinger, da ikke teknisk begavede brugere sagtens kunne forvirres af alle de potentielle visning/analyse kombinationer de ville præsenteret for.
Grunden til at det stadig er med som en mulig customization er, at der er stor fokus på at brugerne skal have mulighed for at styre hvad programmet skal gøre. 

\chapter{Eksperimenter}\label{eksperimenter}
For at få et indblik i hvordan platformen passer ind på en mobil platform, udføres der en række eksperimenter herefter for at undersøge hvordan pladsforbruger i applikationen er.
På mobile enheder er der begrænset lagerplads grundet enhedens størrelse, som eksempel har vores Samsung Galaxy S4 enheder kun 16 GB hvorimod mange computere nu om dage har mindst 1 TB (ca 64 gange mere).
Dette er nødvendigt at forholde sig til, da det sætter en begrænsning på hvad man kan tillade sig at gøre.
I vores tilfælde er det vi skal overveje hvor stor andel sensor data der kan lagres.
For at undersøge hvordan de forskellige sensorer bruger pladsen på telefonen opstilles to eksperimenter.
\ivan{Evt. omtale af backup af data til server}

Applikationen blev sat til at køre med følgende moduler tændt:
\begin{itemize}
	\item sound
	\item screen
	\item proximity
	\item location
	\item gyroscope
	\item accelerometer
\end{itemize}

Dette udvalg er de moduler der på eksperiment tidspunktet var kørbare.
Det er vigtigt at bemærke at de ikke er i deres endelige form og at listen er ikke er komplet.
Der vil i fremtiden komme flere sensorer og der vil komme analysemoduler der analyserer data og muligvis gemmer denne analyse.
Disse eksperimenter vil derfor kun kunne give et indblik i hvordan sensorer opfører sig og give et estimat for pladsforbruget.

Moduler som screen, proximity og location producerer kun data når der sker ændringer. 
Derfor blev det første eksperiment udført på en måde der forsøger at efterligne brugssituation. 
Telefonen blev bevæget og skærmen blev tændt og slukket flere gange under testperioden.
Det første eksperiment blev udført over 30 minutter.

Det andet eksperiment blev udført over en weekend, hvor telefonen lå stille.
Dette eksperiment burde vise et minimumsforbrug for sensorerne.

\paragraph{Forespørgselshastighed}
Pladsen som modulerne bruger vil naturligvis afhænge af hvor ofte modulerne forespørger sensorerne for data.

I android er der forskellige metoder til at angive forespørgselshastigheden på en sensor.
Der findes fire konstanter der kan benyttes til at angive hastigheder der passer i forskellige kontekster \cite{sensormonitor}.

\begin{itemize}
	\item SENSOR\_DELAY\_FASTEST forespørger hele tiden.
	\item SENSOR\_DELAY\_GAME forespørger hver 20. ms, og er beregnet til spil.
	\item SENSOR\_DELAY\_UI forespørger hver 60. ms og er tilstrækkeligt til brug i userinterfaces.
	\item SENSOR\_DELAY\_NORMAL forespørger hver 200. ms er tilstrækkeligt til at opfange ændringer i skærm orientering.
\end{itemize}
Det er muligt at angive opdateringen i ticks hvis man vil have en langsommere opdatering.
Det skal bemærkes at disse angivelser kun benyttes som et hint til telefonen, det kan ikke garanteres at sensoren bliver forespurgt med det angivne interval.
I dette eksperiment er SENSOR\_DELAY\_NORMAL benyttet.


\subsection{Resultater}
Til analyse af pladsforbruget blev programmet ``SQLite-analyzer'' benyttet \cite{sqliteanalyzer}.
SQLite-analyzer viser statistik for en SQLite database inklusive data for de enkelte tabeller.

\paragraph{30 minutters eksperiment}
Efter 30 minutter fyldte databasen 1156459 bytes ($\sim 1.15$ MB).
Under antagelsen at sensorerne vil fortsætte med at generere data i den samme hastighed vil det efter 24 timer fylde omkring 55 MB. 

De enkelte sensorers tabeller var fordelt på følgende måde

\begin{tabular}{|c|c|c|c|}
	\hline Modul 			& Plads forbrugt i bytes	& Antal datapunkter  & \% af databasen \\
	\hline Accelerometer 	& 609266 / $\sim 609$ KB	& 12434 			 & 51.1 \\ 
	\hline Gyroskop 		& 495018 / $\sim 495$ KB	& 10146 			 & 41.6\\ 
	\hline Lyd 				& 42648  / $\sim 42$ KB		& 1706 			 	 & 4.3 \\ 
	\hline Lokation 		& 4751 	 / $\sim 4$ KB		& 91 				 & 0.92 \\ 
	\hline Proximity 		& 3173 	 / $\sim 3$ KB		& 135 				 & 0.31 \\ 
	\hline Skærm 			& 1242 	 / $\sim 1$	KB		& 54				 & 0.31 \\ 
	\hline 
	\ivan{SKAL disse data lagres diskret form? Kan de aggregeres uden at miste vigtig viden}
\end{tabular} 

Som det ses af tabellen er de store pladssyndere accelerometret og gyroskopet.
Dette skyldes at de forespørger konstant og gemmer al data.

Lydmodulet var på eksperimenttidspunktet meget simpelt, og gemte kun amplituden af den optagne lyd i databasen. 
Af denne grund fylder lydmodulets data i denne test ikke mere end 4 \% af databasen.
Afhængig af hvad man vil analysere vil det være nødvendigt at gemme væsentligt mere end dette.

Lokation er fra android konstrueret så den kun forespørger når der er en tilstrækkelig ændring i lokationen.

Proximity sensoren ændrer sig kun når man sætter noget ind foran den, og der er derfor få datapunkter i dennes tabel.
Det samme gælder for skærmsensoren der kun registrerer en ændring når skærmen enten tændes eller slukkes.	

\paragraph{Weekend eksperiment}

Eksperimentet blev udført fra kl. 13:35 fredag eftermiddag til kl. 08:23 mandag morgen, hvilket giver en total varighed på 66 timer og 48 minutter.

I denne periode blev databasen fyldt med 136523139 bytes (136 MB) hvilket svarer til 1018829 bytes (1 MB) pr. halve time eller 49 MB i døgnet.

\begin{tabular}{|c|c|c|c|}
	\hline Modul 		 & Plads forbrugt i bytes	  	& Antal datapunkter & \% af databasen \\
	\hline Accelerometer & 65493057 / 65 MB   			& 1336593    		& 47.6 \\ 
	\hline Gyroskop 	 & 64930232 / 64 MB  			& 1335216   		& 47.6\\ 
	\hline Lyd 		  	 & 5476680  / 5  MB  			& 227991     		& 4.4 \\ 
	\hline Lokation 	 & 622494 	/ 622 KB			& 11971      		& 0.45 \\ 
	\hline Proximity 	 & 3173 	/ 3 KB				& 135 				& 0.003 \\ 
	\hline Skærm 		 & 138           				& 6          		& 0.003 \\ 
	\hline 
\end{tabular} 

Denne tabel ligner ikke overraskende den tabel der blev produceret af eksperimentet på 30 minutter.
Det er igen accelerometret og gyroskopet der fylder databasen efterfulgt af lydmodulet der bruger væsentligt mindre, og til sidst de resterende moduler der næsten ikke bruger noget plads.

\subsection{Muligheder for begrænsning af pladsforbrug}
For ikke at fylde mobil telefonen op med data fra modulerne kan det være en god idé at begrænse pladsforbruget, hvilket kan gøres på forskellige måder, nogle af måderne er beskrevet herefter.

\subsubsection{Client server}
Ved at sætte en server til rådighed for at holde på data kan man holde det aktuelle forbrug på selve telefonen nede.
Det vil kræve at man sætter en synkronisering op der sørger for at kun data der allerede er analyseret bliver slettet fra telefonen. 
Frekvensen af en sådan synkronisering afhænger derfor af hvor lang tids data analysemodulerne er afhængige af.
Det vil selvfølgelig altid være muligt at hente data tilbage fra serveren, eller hvis pladsen bliver et større problem, udføre analyser på serversiden.

\subsubsection{Oprydning i data}
Skærm tændt modulet er opbygget således at det kun indsættes data når skærmen bliver tændt eller slukket.
En udvidelse af de andre moduler kan gøre at de virker ligedan. 
For eksempel er accelerometerdata kun interessante når der sker en vis svingning i acceleration.
Der kan da sættes en tærskel for hvor store svingningerne i acceleration der skal ske for at dataen gemmes i databasen.
Her kunne algoritmer som Douglas-Peucker algoritme eller Sliding Window blive brugt. \ivan{Forklar gerne utralkort}

\subsubsection{Opdateringshastighed}
Som nævnt under opsætningen til forsøget bruges android konstanter til at angive hvor hurtigt en sensor skal opdatere.
Disse konstanter er \textbf{indbygget} og beregnet til opdateringshastigheder der er hurtige nok til at være responsive ved brug i applikationer.\ivan{RØD RETTELSE}
Vores kontekst er at logge brugerens færden, og 5 gange i sekundet er ikke nødvendigt.
Der kan derfor spares en del plads ved enten at opdatere langsommere generelt, eller ved at ændre opdateringshastigheden løbende i takt med at der kommer mere relevant data.
Det skal dog overvejes om den nedsatte præcision kommer til at have en effekt på de analyser der skal bruge dataen.
\ivan{Er det en ide at diskutere opdateringshastighed i forhold til forskellige typer data?}


\subsection{Konklusion}
Når telefonen ikke er i brug vil det valgte udsnit af moduler generere 49 MB data i døgnet mens aktiv brug af telefonen vil få dette forbrug op på 55 MB.

De store syndere er accelerometret og gyroskopet der begge er sat til at gemme data 5 gange i sekundet.
Det vil derfor være nødvendigt at udvikle en strategi for begrænsning af datamængde.
Her er foreslået enten oprydning af data som går ud på kun at gemme det data der er interessant for analyserne.
En anden mulighed er at nedsætte opdateringshastigheden så der ikke gemmes så ofte.
\ivan{En tredje mulighed er at aggregere (tælle \# skrift over tid istedet for diskrete værider)}

Implementation af enten den ene eller begge af disse strategier vil kunne reducere den gemte data, men det vil højst sandsynligt ikke stadig være for meget til at dataen kan holdes på telefonen i længere perioder.\ivan{OK. denne kombi af negativ/positiv kan jeg ikke parse. Jeg giver op :-)}
Derfor vil en udvidelse med server klient arkitektur være nødvendig, men der er ikke nok ressourcer eller tid til at implementere dette så derfor vil dette ikke blive implementeret.\ivan{Unødvendigt defensivt. Dette kan tilføjes en senere version}

\chapter{Refleksion}
I udviklingsforløbet af platformen, var der forskellige beslutninger, som blev taget i henhold til hvad der skulle implementeres, samt hvad der kunne udvikles videre på og generelle overvejelser om platformen.
At reflektere over disse beslutninger og overvejelser skaber overblik og forståelse af forløbet, og er beskrevet herefter.

%\section{Valg af data kilder}
%Hvilke data kilder man bruger i for eksempelvis et analyse modul kan være vigtige, idet hvis man for eksempel har data fra et smartwatch og fra en accelerometer på smartphonen og en af disse viser ingen bevægelse ville det være en god idé at hvis modulet selv kunne evaluere hvilke kilde den skal tage data fra. 
%Men dette blev aldrig lavet, idet det ikke var tænkt særlig vigtigt og at ressourcerne var begrænsede og skulle bruges på andre opgaver i stedet for.

\section{Kontekst}
Konteksten for brugen af applikationen er ikke blevet arbejdet med.
Man kunne forstille sig at der kunne være indbyggede moduler til at registrere hvilken kontekst smartphonen befinder sig i.
Dette kunne være at finde ud af om smartphonen befinder sig på en person eller om den er i en jakkelommer, på et natbord eller lignende.
Hvis det er muligt at identificere konteksten, kan det være yderst brugbar information for analysemoduler, til at hjælpe dem med at afgøre hvordan forskellige former for data skal behandles.
Her kunne man for eksempel forestille sig at lyd var yderst relevant hvis enheden var på en person, men irrelevant hvis den lå i en jakkelomme.
Der er også den mulighed at platformen ud fra konteksten vælger hvilke datakilder, der stadig er relevante at se på, men en sådan måde at beslutte på vil reducere platformens modularitet, så det vil være bedre at overlade det til det enkelte modul at beslutte hvad der er relevant for den givne kontekst.

Dog vides det ikke om en sådan klassificering kan opnå den fornødne præcision, men undersøgelse af det vil bestemt være værd at overveje.

\section{Brugerinteraktion}
En række refleksioner går på brugerinteraktion.
Dette inkluderer refleksion af visualiseringer, interaktive moduler, notifikationer og huskekort, hvilket er beskrevet herunder.

\subsection{Visualiseringer}
Data indsamlet og analyseret af forskellige moduler, skulle originalt kunne visualiseres, med det formål at give brugere af platformen et overblik over deres sindstilstand.
I \cref{modul_definition}, blev der diskuteret hvordan håndteringen af visualiserings-modulerne skal foretages.
Dette er et emne, der bør udforskes ved videre arbejde, men er ikke foretaget på nuværende tidspunkt, da det blev dømt for tidskrævende.
Ved videre arbejde bør datavisualiserings-teknikker og -teori undersøges nærmere.

\subsubsection{Hvordan skal brugeren vælge visualiserings moduler?}
Udover diskussion om hvilke krav der skal være til visualiserings-moduler, beskrevet i \cref{modul_definition}, skal man også overveje hvordan brugere skal vælge hvilke visualiserings-moduler de gerne vil kunne se, og hvordan de skal vises i manageren.

En mulighed er at have en administrationsside til visualiseringer, der minder om indstillingssiden til valg af dataindsamlings- og analysemoduler.
Eksempelvis kan man forestille sig en kobling mellem disse moduler og visualiseringerne, sådan at for hvert modul kan man angive hvilke visualiseringer man vil benytte.
Koblingen ville her gå på visualiseringer, der passer til de enkelte datasæt som modulerne tilbyder.

En anden mulighed er at visualiserings-modulerne, der er installeret, er dem der bliver vist i manageren, og specificerer selv hvilke moduler de virker til.
Dette er en mere enkel løsning at implementere, til gengæld tilbyder den knap så stor fleksibilitet i forhold til udviklingen af nye analysemoduler, da man så er nødsaget til at skulle opdatere specifikationen for de pågældende visualiserings-moduler.

\subsubsection{Hvordan skal visualiseringer vises i manageren?}
En anden diskussion går på hvorledes visualiseringerne skal vises i manageren.
En enkel måde at gøre dette på er som en liste, der bliver præsenteret på forsiden af manageren.
Der vil med denne løsning være mulighed for at ændre rækkefølgen af modulerne ved at trække visualiseringerne til en anden position i listen.
En ulempe ved denne løsning er at den ikke tilbyder kategorisering af visualiseringer.

En inddeling af visualiseringerne i forskellige kategorier ved hjælp af faner er et alternativ, der kan benyttes.
Dette sikrer at hvis man har mange visualiseringer bliver det nemmere at abstrahere over disse, da man kan nøjes med at se på en kategori ad gangen, eksempelvis søvn eller social aktivitet.

\subsubsection{Hvordan skal visualiseringer gøres forståelige?}
Endvidere er der en overvejelse, som går på at udvikle visualiseringer, der er forståelige og gennemskuelige for målgruppen, men dette betragtes som et ansvar for den enkelte visualiserings-udvikler og er dermed ikke noget platformen bør takle.
Platformen benyttes af udvikleren, men hvordan de enkelte visualiseringer skal se ud er op til den enkelte udvikler.

De præsenterede muligheder er ikke fastlagte, og opfordres til at blive diskuteret yderligere ved videre arbejde, så den mest hensigtsmæssige løsning opnås.

\subsection{Interaktive moduler}
En idé, som ikke blev undersøgt var at udvikle interaktive moduler.
Eksempelvis kunne patienter blive sat til at løse matematiske problemer for at teste deres kognitive evner eller spille `balance the ball' for at teste deres reaktionsevne.
Dette kunne være en god idé, da man ved affektive lidelser nogle gange ser den slags symptomer, for eksempel ved depression kan man tit opleve at man har svært ved at huske detaljer, tage beslutninger eller koncentrere sig og hvis spil kunne laves for at teste disse kunne man få et indblik i hvordan patienterne har det. 
En yderlig grund til at undersøge og implementere dette er, at det vil gøre systemet mere interessant for brugerne og give dem en grund til at bruge det hver dag. 
Et sådant modul ville klassificeres som en mellemting mellem et visualiserings modul og et sensor modul.
Visualisering da modulet skal vise noget til brugeren og sensor da den giver data, der kan bruges af analysemoduler til at få et indblik i folks tilstand.

\subsection{Notifikationer}
Noget, som også blev diskuteret tidligere i projektforløbet var at give notifikationer til brugeren, herunder det, som tidligere er benævnt som interventioner.
Dette blev dog tilsidesat, da det blev vurderet at der var for mange faldgruber ift. at forstyrre en evt. patient med psykiske problemer.
Dog erfarede vi senere, i forbindelse med styregruppemødet, at patienterne er meget interesseret i at få at vide så tidligt som muligt om de er under forværring i deres tilstand.

Der kunne eksempelvis gives to slags informationer; interventionen, som direkte siger at der foregår en forværring eller at der har været en drastisk stigning/fald ift. sensor/analyse.
Den anden slags information kan være noget opsummerende og mere neutralt, så der ikke på samme måde skal evalueres om hvorvidt der er ved at ske en forværring, derimod blot præsentere noget af det seneste data fra en sensor/analyse.
Det andet eksempel på en notifikation kunne få brugeren til at reflektere over sin situation, hvis der for eksempel kunne ses et tydeligt fald i søvnkvalitet eller antal sociale interaktioner, men uden direkte at sige noget om en evt. forværring.

\subsubsection{Huskekort}
I interviewet med psykolog Janne Vedel Rasmussen blev huskekort nævnt, se \cref{sec:moede-med-psykolog}.
Disse dækker over nogle fysiske huskekort som patienten har på sig, hvorpå der står en form for instruktioner, der kan assistere i svære situationer; hvor dette eksempelvis kan være at ringe til en ven før en svær beslutning skal træffes eller at droppe den nuværende opgave og i stedet foretage en lystbetonet aktivitet.

Disse kan på samme måde som benævnt ovenover også implementeres som notifikationer.
På denne måde kan der foreslås en lystbetonet aktivitet hvis analyser antyder en stressende dag.
Dette er især interessant ved fysisk sensor/analyse hvor stress antageligvis kan opdages i øjeblikket og et passende huskekort kan præsenteres.

Disse huskekort skal dog nødvendigvis udarbejdes sammen med patientens behandler, da antal, type og indhold af huskekort varierer.

\section{Platform}
Andre refleksioner går på platformen.
Disse refleksioner omhandler hvordan man kan mindske plads- og strøm-forbrug, alternativer til moduldefinitions-deling, samt hvordan en højere granularitet kan opnås hvad angår databaserettigheder, og er beskrevet herunder.
Ydermere reflekteres der også om platformens styrker og svagheder.

\subsection{Dataindsamling}
Som nævnt i \cref{eksperimenter}, blev der udført eksperimenter for at få en idé over hvor meget plads der kræves for at kunne benytte platformen.
Ydermere siger den at mængden af data der indsamles kan blive reduceret.
Der nævnes tre forskellige måder dette kan gøres på, hvilket er ved brug af skyen, komprimering af data eller ved at ændre på opdateringshastigheden for de forskellige sensorer.
I den sammenhæng er det vigtigt at vægte fordele og ulemper for metoderne, samt om det er muligt at kombinere de forskellige metoder.
En kombinering man kunne forstille sig er at lagre gammelt data i skyen, hvorimod ny data lagres lokalt, men hvor al data komprimeres.
Hvordan dette skal implementeres er et åbent spørgsmål, men bør udforskes videre.
Som udgangspunkt har vi lavet et interface, som modulerne kan benytte, i form af DBAccess, så disse ændringer ville være backend ændringer og dermed ville de ikke komme til at påvirke de enkelte moduler.


En anden tankegang til at reducere plads er ved at ændre arkitekturen så man kan kommunikere modul til modul uden nødvendigvis at lagre i en database.
En fordel ved dette kan være at moduler kan sende deres opsamlede data til andre moduler, så man kan undgå helt at gemme data i databasen for visse mellemleds-moduler.
På den måde vil man kunne spare en masse plads, men det vil komme på bekostning af at det ikke længere er muligt at rekonstruere en analyse. En måde dette kunne implementeres på, uden at gå på kompromis med modulariteten, er at benytte sig af et observer designmønster som beskrevet i \cref{subsec:DBACCESS}.
Dog er det en stor ændring i forhold til den nuværende arkitektur, men er et alternativ man med fordel kunne benytte sig af til at gøre datadeling mellem moduler mindre pladskrævende.

Når en eller flere af disse er implementeret, er det en god idé at gentage eksperimenterne for at verificere at løsningerne har den ønskede effekt, og på samme tid vil det være fordelagtigt at udvide eksperimenterne til at inkludere andre faktorer såsom batteriforbrug og netværksforbrug.

\subsection{Analyse i skyen}
For at aflaste smartphonen, er det allerede blevet forslået at bruge skyen til at opbevare data, men på samme tid kan analyser fordelagtigt også blive udført på skyen. 
Dette vil have den effekt at smartphonen ikke behøver gøre særlig meget udover at registrere sensor data.
Dette er også en idé, der bør undersøges videre, hvor man også skal have problematikker såsom sikkerhed og kommunikationstid i mente.

\subsection{Alternativ moduldefinitions-deling}
Med den nuværende implementering skal platformen hente ressourcefilerne for de enkelte moduler, for at læse deres moduldefinitioner.
Med denne løsning har andre applikationer dog også rettighed til at læse ressourcefilerne, hvilket normalt ikke er muligt med Android applikationer og det er på samme tid heller ikke den ønskede måde at dele filer på i Android.

Til at undgå denne metode af deling, bør der ved videre arbejde blive set på andre metoder til at dele filer.
En mulig metode var at bruge en fileprovider.
En file provider kan gøre det muligt at dele en bestemt fil og er den officielle måde at dele filer på i Android.
Vi nåede dog frem til at brugen af fileprovider havde en væsentlig ulempe.
Det var nødvendigt for platformen at vide hvilke fileproviders der skulle bruges, hvilket gør platformen væsentlig mindre modulær, da tilføjelse af et modul også vil kræve ændring i platformen.

Vi tager det forbehold at andre alternativer kan benyttes vi ikke har kendskab til, og hvis en sådan løsning findes bør denne benyttes da den nuværende løsning ikke er idéel, idet det ikke er sådan man skal dele filer mellem applikationer på Android.

\subsection{Databaserettigheder}\label{databaserettigheder}
Med den nuværende implementering af DBAccess har hvert modul adgang til alt data som ligger i databasen.
Dette er unødvendigt åbent og giver for bred adgang til de forskellige modulers data.
For at gøre løsningen mere sikker, bør der med videre arbejde implementeres en måde til at begrænse hvilke moduler, der har adgang til hvilket data.

En ønsket mulighed er at gøre det internt i platformen, idet man kan definere tilladelser i moduldefinitionen og baseret på disse give tilladelse til tabeller. 
Dette har ikke været muligt at implementere, da man fra DBAccess' vinkel ikke kan blive orienteret om hvilket modul benytter den, dette er en begrænsning fra Androids side.

En måde at løse dette problem på er ved en nøgleudveksling mellem manager og modul når et modul installeres, hvor hvert modul får en unik nøgle man sender med, så manageren kan kende forskel på moduler.

\subsection{Styrker og svagheder}
Platformen har forskellige styrker og svagheder, disse bliver dokumenteret herunder. 

Platformen er meget modulær, idet det er simpelt at tilføje nye moduler til den. % Styrke
Dette har dog den betydning at brugeren potentielt skal installere mange applikationer, hvilket selvfølgelig kan være forvirrende hvis de ikke ved hvad de skal bruge. %svaghed
Man kunne måske her forstille sig en fremtidig løsning hvor systemet anbefalede hvilke applikationer, der skulle hentes for at lave bestemte analyser, og måske endda at det var muligt at installere dem gennem platformen.

Platformen har det problem at størrelsen af data den gemmer kan give problemer og lige nu er der ikke blevet implementeret noget, som kan håndtere dette på en hensigtsmæssig måde. % Svaghed

Denne mængde data kan tage lang tid at bearbejde og analysere, hvorfor nogle moduler har måder, der gør det nemmere at håndtere meget data såsom at tage det lidt af gangen.
Idet platformen tillader dette er det en styrke. % Styrke

Platformen er meget åben, idet den deler adgang til databasen og ressourcefiler, hvilket gør det meget nemt for modulerne at få adgang til de datasæt der skal bruges. 
Der kan dog gøres et argument for at det er for åbent, idet alt for meget data deles og ikke bare hvad modulerne skal bruge. 

\lasse{Hvis der er nogen som har nogle andre styrker og svagheder må de godt skrive dem her.}

\chapter{Konklusion}
Der kan konkluderes at en platform er blevet udviklet, der understøtter udviklingen af dataindsamlings- og analyse-moduler.
Et bevis derpå kan findes i \citet{misc:soevnrapp} og \citet{misc:surveyrapp}, hvor udviklingen af søvnestimeringsmoduler, aktivitetsmoduler, surveymoduler, opkald og sms/mms moduler er beskrevet.
Dette hænger fint i tråd med F-16 flyet som en metaforisk vision.
Tankegangen om at have en række moduler man kan hægte på platformen fungerer i praksis, men der er stadig en række problemstillingerne der bør arbejdes videre på som reflekteret ovenfor.

Derudover lød udfordringen på at udvikle en applikation til at overvåge adfærd for unipolare og bipolare patienter og gøre dem opmærksomme på adfærdsændringer.
Denne udfordring er ikke løst, men der er påbegyndt en løsning af problemet.
Fokus har gået på dataindsamlings- og analyse-aspektet af platformen, hvor det er blevet gjort muligt at udvikle moduler dertil, jævnfør tidligere nævnte udviklede moduler.
Vi kan konkludere at platformen har været tilstrækkelig modular og fleksibel til de udviklede moduler, men det er muligt at forbedringer kan foretages,
Det skulle dog være nemt at udvide platformen uden at skulle ændre på implementeringen af de enkelte moduler, da et facade designmønster er benyttet.
Platformen danner dermed grundlaget for videre arbejde, da med platformen udviklet bør der ved videre arbejde fokuseres på moduler til den efterfølgende registrering af ændring i adfærd og visualisering af analyser, baseret på analyseresulaterne af tidligere nævnte udviklede moduler.

Da fokus har været at facilitere dataindsamlings- og analyseaspektet af platformen mangler der at blive udforsket hvorledes visualiserings-moduler skal faciliteres.
Vi er klar over denne mangel, men det blev bedømt vigtigere at fokusere på at gøre det nemt at indsamle relevant data først, inden at der blev lagt fokus på hvordan man skulle visualisere data.

\als{Der skal skrives noget konklusion i forhold til kriterier - hvordan overholder platformen kriterierne?}


% Platform facilliterer udviklingen af moduler der kan benyttes
% Mangler stadig udvikling, såsom visualiserings perspektivet
% på trods af det svarer den til vision som metafor
% evaluering - integrationstest
% når stefan og mikael har taget kriterier, skriv om hvorvidt platformen holder det


\bibliographystyle{unsrtnat}
\bibliography{bib}
\label{bib:mybiblio}

\appendix
\include{appendix/janne}
\chapter{Referat af møde med Jørgen Aagaard}\label{app:moede-med-joergen-referat}
Dette er et referat af møde med Jørgen Aagaard der foregik 20-02-2015. Til stede var begge projektgrupper, vejleder Ivan Aaen, ekstern kontakt Morten Aagaard samt psykiatriprofessor Jørgen Aagaard.

\section{Referat}
Møde med Jørgen Aagaard. Er psykiater, men er også somatisk læge. Er vant til at lave forskning.

Præsentation af emne, derefter skal der laves fokus på hvad der skal snakkes om.

Introduktion af personer.

Ivan er lektor, arbejder med software innovation. Er interesseret i hvordan målinger kan bruges til at afhjælpe psykiske eller somatiske problemer. Tanken er at der skal være et samarbejde mellem software og psykiatri, hvor vi skal se hvilke data der kan trækkes frem og hvad de kan bruges til.

Deprimerede har døgnrytme problemer, kropsligt stress, det er ændringer i deres aktivitet. Der er klart nogle ting som er karakteriske for deprimerede, da det kan være optakt til at de får det værre. Klinisk beslutningtagen bliver skiftet til at patienten bestemmer mere, selvom det traditionelt har været paternalistisk. I det kommer der ny teknologi der kan hjælpe.

Kan der skelnes mellem stress og svulst i hjernen?
Svulst i hjernen vil blive gradvist værre uden tegn på forbedringer, hvorimod stress ville kunne se periodiske forbedringer.

Aktiviter, døgnrytme er mere vigtige for deprimerede/affektive selvom det påvirker alle mennesker. 

Døgnrytme kan detekteres fra en mobil. Døgnrytme ændrer sig hvis depressionen ændrer sig. 
Telefonisk aktivitet: Hvem de ringer til, hvor lang tid de snakker, hvad de snakker om i SMSer. Her er det vigtigt at se på adfærdsændring. Da man har et stabilt aktivitetsmønster, og hvis den ændrer sig kan det være en indikation for at depression ændrer sig. Det er subjektets adfærd der ændrer sig, det er ikke noget objektivt. 
Social aktivitet, fysisk aktivitet. Hvis den ændrer sig for patienten kan det indikere ændring i sindstilstanden. 
Fysisk aktivitet kan måles ved accelerometer. Hvis det ændrer sig, igen er det vigtigt. Man skal helst bevæge sig. Man kan f.eks. bruge skridttæller til at måle fysisk aktivitet, dog ændrer det sig fra ugedage til weekend. Skridttæller kan bruges til at indikere om de f.eks. forsøger at tabe vægt, som de er blevet fortalt at de skal. Hvis depressionen bliver værre får man mere kropslig stress: Rystelser, sved, puls. Man skal måske skelne mellem makro bevægelser(skridt tæller, hvor mange kilometer man går) og mikro bevægelser(rystelser). Der er ikke noget der er rigtigt, det er ændringer for subjektet.

Stemme: Man kan samle flere oplysninger f.eks. toneleje, hvor hurtig man snakker, hvor længe man snakker, man kan oversætte det til tekst og bruge sentiment analyse. Når man bliver deprimeret, kan stemmemodulation blive mindre. Man kan snakke trist, kedelig og uden motivation.

Tekst: Her kan man se på hvilke ord der bliver brugt, så kan man måske se om de ændrer ord de bruger. 

Tastatur: Man kan se på rettelser, fortrydelser. Persondataloven er nok vigtig her.

Kognitive: Hvis man er stresset, kan de få det meget værre. Det er sværre at huske, tider, kort tidshukommelse. Man kunne forestille sig at lave kort tids hukommelses test. Her kan man se om huske applikationer bliver brugt mere ofte, f.eks. kalendere. Som sagt, ved højere kropslig stress, er det meget mere udtalt ved affektive patienter.

OCD: Symptomer: Tvangsritual, som at vaske hænder 25 gange og hvis man ikke gør det føler man tvang. Der er en rimelig stabil tilstand, som langsomt bliver værre el. bedre. Har ikke tænkt over det, men man kan nok opdage om det bliver bedre. De får det bedre når de ikke er trynet af egne tvangstanker/handlinger. Det er lidt sværre at gå til. OCD kan også være et symptom af f.eks. depression. Her kan man nok etablere en baseline, hvor patienten udfører tvangstanken og så opdage om det bliver mindre hyppigt. Tvangstanker er noget alle oplever, men dem med OCD er det meget mere optrædt, som f.eks. en normal person tjekker 1-2 gange om de har låst mens en med OCD vil gøre det 10 gange. 
Ivan viser en opmærksomhedstest.

Kan man bruge målinger til at se forskel på psykiske eller somatiske problemer. Hvis man nu tager depressions området og dem som er over 50 år, så har de en større hyppighed for fysiske problemer, f.eks. kognitive forstyrrelser og det hænger sammen med depression. En yngre deprimeret, som f.eks. 25 år så er der ingen sammenhæng.

Hvis der nu er forkalkning i hjernen, og dette tripper depression så vil der være kognitive problemer, men det vil være mere stabilt end op og nedadgående. Hvis man kan måle at kognitive problemer bliver værre, kan det være at det ikke er et stress relateret problem. F.eks. regneopgaver som bliver adminstreret hver dag kan bruges til at måle om det bliver værre, man kan også gentage uafhængige ord, man kan også gentage en ordrække der ikke giver mening, om man kan huske talrækker efter et kort stykke tid. Det skal være meget simple tests, og så se om performance bliver værre. Man kan også måle om man retter svaret, om hvor lang tid man tager ved at svare som supplerende information. MMSE(vigtigt!). Man kan prøve at gentage telefonnummer bagfra, 12 34 56 78 -> 87 65 43 21. Man skal have tilladelse til at kunne have denne data, og at det er patientens data, der kræves nogle tilladelser for at have med patienter at gøre. Det kan være et hjælpemiddel til at stille den rigtige diagnose, for at åbne for en diagnostisk revurdering. Her kan man gå efter et væld af sygdomme, de er meget præget af at adfærds forandring er sygdoms forandring. Kropslig stress. 

Ved depression og mani, ændrer man sin sociale adfærd? Ja det bliver værre, f.eks. hvis der er et par så kan konen se at manden bliver værre f.eks at døgnrytmen ændrer sig, at der ikke er en naturlig sammenhæng mellem følelser og hvad der siges. Aktiviteter nedsættes, døgnrytme ændres, vil ikke selv sige at det sker. Ved mani, lidt anerledes: Sover mindre, urolighed, anden seksualitet, andre ser det først, spiser mindre, social kontakt spam lignende. Det kan nok detekteres teknologisk, her kan man se om man SMSer mindre eller om man snakker med andre folk. 

Sygdomsangst skal nok undgås. 

Hvor mange patienter med affektive lidelser har 'familie'? Dem med affektive lidelser har mindre stabil samliv med familie eller venner. 
Man kan nok tænde mikrofonen i 10 sekunder, og så se om man sidder alene for at finde ud af om der overhovedet er nogen lyd.

Hvis du skulle se på patienter med affektiver lidelser, hvilke er så mest værdifuld? Hvad er lavt hængende frugter, som kan realiseres i den tid der nu er, hvilke har størst værdi tror du? Jørgen vil give hvordan man kan vide det om patienter først, så hvad hans synes.
Patienter: Man kunne lave en fokusgruppeinterview af patienter og spørge dem hvad de vil have, den som administrere det skal have prøvet det først.
Læger: Døgnrytme registering. Hvornår går man i seng og hvordan man står op. Man kan også have forstyrret søvn, og om man vågner i REM søvn.
Socialaktivitets registrering.
Biologisk stress registrering(Ingen motion, sveder, ryster, pupil ændringer - Der skal bare være et udtryk for det). Pupilændringer, kan også bruges men det er bare et udtryk for biologisk stress.

Ustabil medicinering, påvirker biologisk stress. 

Patienter skal have fuld tilsagn, og om data går andre steder hen end deres egen mobil. Vi skal nok tænke på hvad der er nyttigt, og ikke tænke på etiske problemer. Det er mere fokuseret på udvikling af teknologi som kan afhjælpe behandling.

Et fokusgruppeinterview kan nok godt organiseres, Jørgen tilbyder det. Her skal man finde ud af hvordan det skal afholdes, interview teknikken er vigtig og det er ikke særlig svært at lære.

De kognitive aspekter gemmes til en anden gang. 

Patienten har et krisekort princip, de skal sige hvordan de har det og hvordan de skal reagere(Eskaleres op til at de skal ringe til Psykiatrien). Patienter har en udskrivningsaftale, som siger hvem de skal ringe til. 

Hvis man skulle præsentere personens tilstand for personen, hvordan skal man gøre det? Hvis du laver udfra indsamlet data, er en præsentation af hvordan de har det nyttig. Præsentationen, skal være enkel og på patientens egne præmisser. VAS Skala, om tilstanden fra 0-100 går op eller ned og så sætter man bare krydser. Det skal være en meget enkel visualisering af det. 
Man kan komme til at vise nogen data som er meget nedslående, hvad skal vi gøre ved det? Det elektroniske hjælpe middel skal knytte brugerens tilstand, og andre skal ikke umiddelbart have adgang til det. Den enkelte patient har et krisekort hvor der er nogle ting de skal gøre ved forværring.

Det med ryst på hånden gennem et spil er nok en enkel måde at læse uro eller stress. Man kan nok få ekstra niveauer af information ud fra den slags teknologi, f.eks. hvordan man udfører opgaven og om man ryster. 
\include{appendix/fokusgruppemoede}
\chapter{JSON schemas}\label{app:json_schema}

\begin{lstlisting}[caption=Module]
{
  "$schema": "http://json-schema.org/draft-04/schema#",
  "title": "Module",
  "type": "object",
  "javaType": "dk.aau.cs.psylog.data_access_layer.generated.Module",
  "properties": {
    "name": { "type": "string" },
    "_userfriendlyname": { "type": "string" },
    "_version": {
      "type": "number",
      "minimum": 0.0,
      "exclusiveMinimum": true
    },
    "_description":{ "type": "string" }
  },
  "required": [ "name", "_version" ],
  "additionalProperties": false
}
\end{lstlisting}

\begin{lstlisting}[caption=DataModule]
{
  "$schema": "http://json-schema.org/draft-04/schema#",
  "title": "DataModule",
  "type": "object",
  "javaType": "dk.aau.cs.psylog.data_access_layer.generated.DataModule",
  "extends": { "$ref": "module.schema.json" },
  "properties": {
    "tables": {
      "type": "array",
      "items": { "$ref": "table.schema.json" },
      "minItems": 1,
      "uniqueItems": true
    },
    "task": { "$ref": "task.schema.json", "description": "If no task is supplied, defaults to continuous" }
  },
  "required": ["name", "_version"],
  "additionalProperties": false
}
\end{lstlisting}

\begin{lstlisting}[caption=SensorModule]
{
  "$schema": "http://json-schema.org/draft-04/schema#",
  "title": "SensorModule",
  "type": "object",
  "javaType": "dk.aau.cs.psylog.data_access_layer.generated.SensorModule",
  "extends": { "$ref": "datamodule.schema.json" },
  "additionalProperties": false
}
\end{lstlisting}

\begin{lstlisting}[caption=AnalysisModule]
{
  "$schema": "http://json-schema.org/draft-04/schema#",
  "title": "AnalysisModule",
  "type": "object",
  "javaType": "dk.aau.cs.psylog.data_access_layer.generated.AnalysisModule",
  "extends": { "$ref": "datamodule.schema.json" },
  "properties": {
    "dependencies": {
      "type": "array",
      "items": { "type": "array", "items": { "$ref": "#/definitions/dependency" }, "minItems": 1, "uniqueItems": true }
    }
  },
  "additionalProperties": false,
  "definitions": {
    "dependency": {
      "properties": {
        "name": { "type": "string" },
        "version": { "type": "number", "minimum": 0.0, "exclusiveMinimum": true }
      },
      "additionalProperties": false,
      "required": [ "name", "version", "task" ]
    }
  }
}
\end{lstlisting}

\begin{lstlisting}[caption=ViewModule]
{
  "$schema": "http://json-schema.org/draft-04/schema#",
  "title": "ViewModule",
  "type": "object",
  "javaType": "dk.aau.cs.psylog.data_access_layer.generated.ViewModule",
  "extends": { "$ref": "module.schema.json" },
  "properties": {
    "view": {
      "type": "object",
      "properties": {
        "layout": { "type": "string", "pattern": "^[a-zA-Z0-9]+(.xml)" },
        "viewData": {
          "type": "array",
          "items": { "$ref": "#/definitions/viewData" },
          "minItems": 1,
          "uniqueItems": true
        }
      },
      "additionalProperties": false
    }
  },
  "additionalProperties": false,
  "required": [ "name", "_version" ],
  "definitions": {
    "viewData": {
      "properties": {
        "name": { "type": "string" },
        "dataTypes": { "type": "array", "items": { "enum": ["INTEGER", "REAL", "TEXT", "BLOB"] }, "uniqueItems":true, "minItems": 1 },
        "fromTimestamp": { "type": "boolean" }
      },
      "additionalProperties": false,
      "required": ["name", "dataTypes"]
    }
  }
}
\end{lstlisting}

\begin{lstlisting}[caption=Task]
{
  "$schema": "http://json-schema.org/draft-04/schema#",
  "title": "Task",
  "type": "object",
  "javaType": "dk.aau.cs.psylog.data_access_layer.generated.Task",
  "properties": {
    "type": { "enum": ["interval", "scheduled"] },
    "value": { "type": "string", "description": "Interval in minutes, scheduled in HH:MM format" }
  },
  "additionalProperties": false,
  "required": ["type", "value"]
}
\end{lstlisting}

\begin{lstlisting}[caption=Table]
{
  "$schema": "http://json-schema.org/draft-04/schema#",
  "title": "table",
  "type": "object",
  "javaType": "dk.aau.cs.psylog.data_access_layer.generated.Table",
  "properties": {
    "name": { "type": "string" },
    "columns": {
      "type": "array",
      "items": { "$ref": "column.schema.json" },
      "minItems": 1,
      "uniqueItems": true
    }
  },
  "additionalProperties": false,
  "required": [ "name", "columns" ]
}
\end{lstlisting}

\begin{lstlisting}[caption=Column]
{
  "$schema": "http://json-schema.org/draft-04/schema#",
  "title": "column",
  "type": "object",
  "javaType": "dk.aau.cs.psylog.data_access_layer.generated.Column",
  "properties": {
    "name": { "type": "string" },
    "dataType": { "enum": [ "INTEGER", "REAL", "TEXT", "BLOB", "BOOLEAN" ]},
    "nullable": { "type": "boolean" },
    "_unit": { "type": "string" }
  },
  "additionalProperties": false,
  "required": [ "name", "dataType"]
}
\end{lstlisting}
 % Out of date.
\chapter{Brochure Med Idéer Til Brug Af Sensorer}\label{app:brochure}
\includepdf[pages=-, nup=1x2]{appendix/brochurefil.pdf} 

\chapter{Stemningsregistrering}\label{app:stemningsregistrering}
\includepdf[pages=-]{appendix/stemningsregistrering.pdf}
\end{document}
