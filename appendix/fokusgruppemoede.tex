\chapter{Referat af fokusgruppe interview}\label{app:fokusgruppe-interview-referat}
\newcommand{\pa}{A}
\newcommand{\pb}{B}
\newcommand{\pc}{C}
\newcommand{\pd}{D}
\newcommand{\pe}{E}

Interviewet blev udført 20/4-2015, på Aalborg Psykiatrisk Afdeling, hvoraf de deltagende var 5 patienter med, åbenbart, unipolar depression.
For at anonymisere fokusgruppe mødet vil de 5 patienter vil herefter blive benævnt henholdsvis \pa, \pb, \pc, \pd~og \pe.

To af projekt gruppernes medlemmer deltog og fungerede som interviewere(Mikkel og Søren), samt en tredje der skrev referat og optog interviewet(Lasse). En rådgiver(Morten) i projektet deltog også.
Først blev projektet introduceret, ved at den overordnede idé blev præsenteret.
Derudover blev mødet forkortet da der var meget lidt produkt at vise, hvilket resulterede i en ca. halvering af varigheden.

\section{Spørgsmål}
Dette afsnit indeholder en kronologisk gennemgang af de stillede spørgsmål og hvad der blev svaret af patienterne.

\paragraph{Hvad er et kendetegn for forværring af jeres situation?}
\pa~fortæller at søvn er en vigtig faktor; før \pa s sidste periode gik \pa~søvnløs i 4 dage.
Derudover trækker \pa~sig socialt; \pa~skulle have besøgt familie, men så sig nødsaget til at vælge det fra.
(\pb~forlader rummet).
\pc~er enig med \pa, i at man stopper med at sove og at man trækker sig socialt.
\pd~har svært ved at håndtere når der sker for meget uvist, hvilket gør det svært at rumme/overskue det, og dette afhænger også meget af søvn.
\pe~fortæller at det er de samme ting der går igen; dårlig appetit, dårligt humør og dårlig søvn.
\pe~skal virkelig tage sig sammen for at udføre noget som helst.
Når \pe~har en periode, er morgener en stor plage, hvor aftener er bedre.
(Der er en mindre diskussion omkring hvordan det er at have en periode).

\paragraph{Hvilken information ville I ønske I havde adgang til vedrørende jeres adfærd?}
\pe~forklarer at man ikke er i tvivl om at man har en depression og mener ikke at man vil orke at slå noget op i applikationen når først man er i en periode.
\pa~mener at det kunne være brugbart med en graf over eksempelvis søvnmønster, så man kunne danne sig et overblik.
Der er stor enighed om at når man først er i en periode, er man ikke i tvivl om dette.
\pe~lægger ikke umiddelbart mærke til noget op til perioder, men mener at de kommer brat.
\pa~fortæller at \pa~har lagt mærke til at \pa s søvn er blevet forringet op til perioden.
\pa~indskyder at det ville være fordelagtigt at fange perioden tidligt.

\paragraph{Hvad er jeres holdning til at blive overvåget af et mobilt system som kan give jer indsigt i jeres situation?}
\pc~har tidligere brugt en søvn-applikation, og det var dette der fik \pc~til at indse at der var et problem.
Indtil dette var \pc~i benægtelse omkring sin lidelse.
(Der er lidt diskussion omkring opbevaring og deling af privat data.)
(Morten indskyder med sin egen applikation til registrering af stress via sms-brug.)
\pa~og \pe~er enige om at denne type applikation er brugbar, da det vigtige er at blive tidligt gjort opmærksom.
\pa~uddyber at i \pa s tilfælde kunne det have været meget nyttigt, da i retrospekt kunne \pa~godt se at sin tilstand var i forværring, grundet den ringere søvn.

\paragraph{Hvordan har I det med at skrive dagbog hver aften om hvordan man har det?}
\pd~mener at det er noget man skal have overskud til, da der skal samles kræfter for at kunne gøre det.
\pe~siger at det ville kunne lade sig gøre før en periode, men absolut ikke under en periode.
\pe~tilføjer at det kunne være en god hjælp, men mener stadigvæk ikke at have lagt mærke til nogle symptomer ved begyndende periode.
(Mikkel forklarer målet med at vise den ændrede adfærd over periode.)
\pa~nævner igen at problemet for \pa~er manglende søvn.
\pe~tilføjer at manglende søvn ikke alene er et symptom, men afbrudt og urolig søvn er også vigtigt.
\pe~og \pd~er enige om at den forringede søvn er en god indikator på begyndende periode.
\pa~fortæller om hvad \pa~forsøger at gøre for at falde i søvn, blandt andet ting som at se fjernsyn, læse og drikke te.
(Mikkel foreslår en knap i applikationen, som man trykker på for at indikere at man forsøger at falde i søvn, da det er relativt nemt at detektere søvn, men ikke at detektere (fejlede) forsøg på at falde i søvn.)
\pa~synes det lyder som en god idé.
(Mikkel pointerer at dette også kunne hjælpe med at se noget positivt i ens adfærd.)

\paragraph{Åben diskussion}

\subparagraph{Er der yderligere kommentarer om grundtanken?}
\pa~nævner at det er vigtigt ikke kun at fokusere på længden af søvnen, men også tidspunkt og kvalitet, såsom uroligheder.
(Morten indskyder at der er gjort et stort indtryk på ham.
Han mener at det skal være et 'early warning' system.
Han fortæller om sin nabo der fik en blodprop og som følge fik at vide han skulle tabe sig og at det ikke hjalp og at idéen her skal være at tidligt give information til brugeren.
Morten stiller spørgsmålet \subparagraph{Hvilke handlinger kan man gøre for at forebygge?}
\pa~synes det er en god idé med 'early warning', da \pa~ser det som en god ting hvis applikationen kunne komme med en advarsel så \pa~kan kontakte egen læge før forværring eller evt. periode.
(\pd~fortæller om hvordan det er at være indlagt, \pe~tilføjer.)
(Morten spørger igen ind til forebyggende handlinger.)
\pa~forklarer at man vil kontakte læge, da man ikke selv kan ordinere medicin, \pe~er enig.
\pa~tilføjer at man også kan kontakte pårørende eller anden bekendt som man kan snakke med, da det sociale også kan hjælpe.
\pe~fortæller at man nok er på medicin i forvejen, ved ikke-førstegangs depressioner, så kontakt til læge er vigtig.
(Morten fortæller igen om sig selv, og nævner ting som motion og struktur i hverdagen til at forebygge, for på denne måde at skabe forudsigelighed.)
\pe~er enig, da ved indlæggelse at strukturere hverdagen er en vigtig del.
\pa~indskyder at \pa~har en struktureret hverdag og træner ca. fem dage om ugen normalt, og alligevel gik det galt og endte i periode.
\pa~er ikke i tvivl om at motion virker, men \pe~mener ikke at motion alene er nok til at forebygge perioder.
\pc~siger at når man først er kommet i periode, er der ingen vej tilbage ud over kontakt til læge.
Der er igen enighed om at det skal opdages tidligt.
\pa~tilføjer at det er svært, da det er meget forskelligt hvilke symptomer der er og hvilken behandling er nødvendig (medicinsk og andet).

\subparagraph{Giver produktet mening?}
\pa~og \pe~synes det virker brugbart, \pc~indskyder at kunne hjælpe til at opdage ved den lette depression, fremfor den svære, og man derved kunne søge hjælp hurtigere.

\subparagraph{Har I nogle bekymringer om produktet?}
\pa~og \pe~siger nej.
(Morten indskyder: hvad hvis applikationen fortæller at man har en 50 \% chance for at få en depression indenfor den næste måned, ville dette være forfærdeligt?)
\pa~siger klart nej, og mener det ville være godt, \pa~ville være glad for dette, da \pa~hellere vil have beskeden end at få depressionen.
\pa~fortsætter med, når \pa~har det godt kan \pa~handle, men når først en periode er indtruffet kan \pa~ikke handle.
(Morten spørger om det kunne blive selv-opfyldende på en dårlig måde.)
\pe~mener nej, så længe man kan regne med beskeden.
Derved kunne man forberede sig på det som kommer, og kan gøre ting som man ved er godt; ved at dyrke mere motion og tilføje mere struktur.
\pa~indskyder at man heller ikke skal blive overstruktureret.
\pe~mener at hvis man får besked om begyndende depression, vil det forværre, \pd~er enig.
\pa~er uenig.
\pd~fortsætter, hvis man får denne type besked vil man begynde at tænke negativt.
\pe~tilføjer, hvis man har haft en depression, er der større risiko for at få efterfølgende depressioner.
\pe~mener ikke at man tænker så meget over det mellem episoderne.
(Der er en mindre snak om depressioner igen.)
Der er igen enighed om at man ikke tænker over depression, når man er mellem perioder.
\pa~mener det ville være en god sikkerhed, men at der skal lægges vægt på formulering af beskeder.
Man bør ikke kalde det for depression, for så bliver det måske selv-opfyldende.
\pd~indskyder at netværk er vigtigt, hvor stor opbakning man får fra hjem og venner.
(Mikkel spørger ind til det med netværk.)
\pd~bekræfter at der bliver mindre kontakt ved begyndende episode.

\subparagraph{Har I idéer som I synes ville være idéelle at bruge, men mangler?}
\pa~forklarer at der er en indre uro, ikke nødvendigvis fysisk, men man kan godt ryste.
(Mikkel benævner applikation til selv-erklæring af humør, spørger ind til hvad der syntes om denne idé.)
\pa~er uenig, da det kun er den indre uro, ikke hvordan man føler at dagen er gået.
(Mikkel spørger videre, eksempelvis at man selv kan formulere spørgsmålene.)
\pa~vil foretrække en skala.
\pa~synes det er en god idé, så man kunne beskrive flere af de symptomer man har.
På denne måde kunne man få en besked om at de symptomer er ved at være opnået.
(Mikkel spørger ind til om det er ok applikationen er intrusive.)
\pa~sidestiller med valg af kanaler, og at man på samme måde skal bestemme sine symptomer.
(Afsluttende ord.)