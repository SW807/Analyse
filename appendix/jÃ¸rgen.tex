\chapter{Møde med Jørgen Aagaard Referat}\label{app:moede-med-janne-referat}

Dette er et referat af møde med \citet{misc:jorgen-aagaard} der foregik 20-02-2015

Møde med Jørgen Aagaard. Er psykiater, men er også somatisk læge. Er vant til at lave forskning.

Præsentation af emne, derefter skal der laves fokus på hvad der skal snakkes om.

Introduktion af personer.

Ivan er lektor, arbejder med software innovation. Er interesseret i hvordan målinger kan bruges til at afhjælpe psykiske eller somatiske problemer. Tanken er at der skal være et samarbejde mellem software og psykiatri, hvor vi skal se hvilke data der kan trækkes frem og hvad de kan bruges til.

Deprimerede har døgnrytme problemer, kropsligt stress, det er ændringer i deres aktivitet. Der er klart nogle ting som er karakteriske for deprimerede, da det kan være optakt til at de får det værre. Klinisk beslutningtagen bliver skiftet til at patienten bestemmer mere, selvom det traditionelt har været paternalistisk. I det kommer der ny teknologi der kan hjælpe.

Kan der skelnes mellem stress og svulst i hjernen?
Svulst i hjernen vil blive gradvist værre uden tegn på forbedringer, hvorimod stress ville kunne se periodiske forbedringer.

Aktiviter, døgnrytme er mere vigtige for deprimerede/affektive selvom det påvirker alle mennesker. 

Døgnrytme kan detekteres fra en mobil. Døgnrytme ændrer sig hvis depressionen ændrer sig. 
Telefonisk aktivitet: Hvem de ringer til, hvor lang tid de snakker, hvad de snakker om i SMSer. Her er det vigtigt at se på adfærdsændring. Da man har et stabilt aktivitetsmønster, og hvis den ændrer sig kan det være en indikation for at depression ændrer sig. Det er subjektets adfærd der ændrer sig, det er ikke noget objektivt. 
Social aktivitet, fysisk aktivitet. Hvis den ændrer sig for patienten kan det indikere ændring i sindstilstanden. 
Fysisk aktivitet kan måles ved accelerometer. Hvis det ændrer sig, igen er det vigtigt. Man skal helst bevæge sig. Man kan f.eks. bruge skridttæller til at måle fysisk aktivitet, dog ændrer det sig fra ugedage til weekend. Skridttæller kan bruges til at indikere om de f.eks. forsøger at tabe vægt, som de er blevet fortalt at de skal. Hvis depressionen bliver værre får man mere kropslig stress: Rystelser, sved, puls. Man skal måske skelne mellem makro bevægelser(skridt tæller, hvor mange kilometer man går) og mikro bevægelser(rystelser). Der er ikke noget der er rigtigt, det er ændringer for subjektet.

Stemme: Man kan samle flere oplysninger f.eks. toneleje, hvor hurtig man snakker, hvor længe man snakker, man kan oversætte det til tekst og bruge sentiment analyse. Når man bliver deprimeret, kan stemmemodulation blive mindre. Man kan snakke trist, kedelig og uden motivation.

Tekst: Her kan man se på hvilke ord der bliver brugt, så kan man måske se om de ændrer ord de bruger. 

Tastatur: Man kan se på rettelser, fortrydelser. Persondataloven er nok vigtig her.

Kognitive: Hvis man er stresset, kan de få det meget værre. Det er sværre at huske, tider, kort tidshukommelse. Man kunne forestille sig at lave kort tids hukommelses test. Her kan man se om huske applikationer bliver brugt mere ofte, f.eks. kalendere. Som sagt, ved højere kropslig stress, er det meget mere udtalt ved affektive patienter.

OCD: Symptomer: Tvangsritual, som at vaske hænder 25 gange og hvis man ikke gør det føler man tvang. Der er en rimelig stabil tilstand, som langsomt bliver værre el. bedre. Har ikke tænkt over det, men man kan nok opdage om det bliver bedre. De får det bedre når de ikke er trynet af egne tvangstanker/handlinger. Det er lidt sværre at gå til. OCD kan også være et symptom af f.eks. depression. Her kan man nok etablere en baseline, hvor patienten udfører tvangstanken og så opdage om det bliver mindre hyppigt. Tvangstanker er noget alle oplever, men dem med OCD er det meget mere optrædt, som f.eks. en normal person tjekker 1-2 gange om de har låst mens en med OCD vil gøre det 10 gange. 
Ivan viser en opmærksomhedstest.

Kan man bruge målinger til at se forskel på psykiske eller somatiske problemer. Hvis man nu tager depressions området og dem som er over 50 år, så har de en større hyppighed for fysiske problemer, f.eks. kognitive forstyrrelser og det hænger sammen med depression. En yngre deprimeret, som f.eks. 25 år så er der ingen sammenhæng.

Hvis der nu er forkalkning i hjernen, og dette tripper depression så vil der være kognitive problemer, men det vil være mere stabilt end op og nedadgående. Hvis man kan måle at kognitive problemer bliver værre, kan det være at det ikke er et stress relateret problem. F.eks. regneopgaver som bliver adminstreret hver dag kan bruges til at måle om det bliver værre, man kan også gentage uafhængige ord, man kan også gentage en ordrække der ikke giver mening, om man kan huske talrækker efter et kort stykke tid. Det skal være meget simple tests, og så se om performance bliver værre. Man kan også måle om man retter svaret, om hvor lang tid man tager ved at svare som supplerende information. MMSE(vigtigt!). Man kan prøve at gentage telefon nummer bagfra, 12 34 56 78 -> 87 65 43 21. Man skal have tilladelse til at kunne have denne data, og at det er patientens data, der kræves nogle tilladelser for at have med patienter at gøre. Det kan være et hjælpemiddel til at stille den rigtige diagnose, for at åbne for en diagnostisk revurdering. Her kan man gå efter et væld af sygdomme, de er meget præget af at adfærds forandring er sygdoms forandring. Kropslig stress. 

Ved depression og mani, ændrer man sin sociale adfærd? Ja det bliver værre, f.eks. hvis der er et par så kan konen se at manden bliver værre f.eks at døgnrytmen ændrer sig, at der en naturlig sammenhæng mellem følelser og hvad der siges. Aktiviteter nedsættes, døgnrytme ændres, vil ikke selv sige at det sker. Ved mani, lidt anerledes: Sover mindre, urolighed, anden seksualitet, andre ser det først, spiser mindre, social kontakt spam lignende. Det kan nok detekteres teknologisk, her kan man se om man SMSer mindre eller om man snakker med andre folk. 

Sygdomsangst skal nok undgås. 

Hvor mange patienter med affektive lidelser har 'familie'? Dem med affektive lidelser har mindre stabil samliv med familie eller venner. 
Man kan nok tænde mikrofonen i 10 sekunder, og så se om man sidder alene for at finde ud af om der overhovedet er nogen lyd.

Hvis du skulle se på patienter med affektiver lidelser, hvilke er så mest værdifuld? Hvad er lavt hængende frugter, som kan realiseres i den tid der nu er, hvilke har størst værdi tror du? Jørgen vil give hvordan man kan vide det om patienter først, så hvad hans synes.
Patienter: Man kunne lave en fokus gruppe af patienter og spørge dem hvad de vil have, den som administrere det skal have prøvet det først.
Læger: Døgnrytme registering. Hvornår går man i seng og hvordan man står op. Man kan også have forstyrret søvn, og om man vågner i REM søvn.
Socialaktivitets registrering.
Biologisk stress registrering(Ingen motion, sveder, ryster, pupil ændringer - Der skal bare være et udtryk for det). Pupilændringer, kan også bruges men det er bare et udtryk for biologisk stress.

Ustabil medicinering, påvirker biologisk stress. 

Patienter skal have fuld tilsagn, og om data går andre steder hen end deres egen mobil. Vi skal nok tænke på hvad der er nyttigt, og ikke tænke på etiske problemer. Det er mere fokuseret på udvikling af teknologi som kan afhjælpe behandling.

Fokus gruppe kan nok godt organiseres, Jørgen tilbyder det. Her skal man finde ud af hvordan det skal afholdes, interview teknikken er vigtig og det er ikke særlig svært at lære.

De kognitive aspekter gemmes til en anden gang. 

Patienten har et krisekort princip, de skal sige hvordan de har det og hvordan de skal reagere(Eskaleres op til at de skal ringe til Psykiatrien). Patienter har en udskrivningsaftale, som siger hvem de skal ringe til. 

Hvis man skulle præsentere personens tilstand for personen, hvordan skal man gøre det? Hvis du laver udfra indsamlet data, er en præsentation af hvordan de har det nyttig. Præsentationen, skal være enkel og på patientens egne præmisser. VAS Skala, om tilstanden fra 0-100 går op eller ned og så sætter man bare krydser. Det skal være en meget enkel visualisering af det. 
Man kan komme til at vise nogen data som er meget nedslående, hvad skal vi gøre ved det? Det elektroniske hjælpe middel skal knytte brugerens tilstand, og andre skal ikke umiddelbart have adgang til det. Den enkelte patient har et krisekort hvor der er nogle ting de skal gøre ved forværring.

Det med ryst på hånden gennem et spil er nok en enkel måde at læse uro eller stress. Man kan nok få ekstra niveauer af information ud fra den slags teknologi, f.eks. hvordan man udfører opgaven og om man ryster. 