\section{Møde med Psykiatri Professor Jørgen Aagaard}\label{sec:moede-med-joergen}
Denne sektion opsummerer et møde med psykiatri professor Jørgen Aagaard på Psykiatrien ved Sygehus Syd i Aalborg \citep{misc:jorgen-aagaard}. 
Jørgen er psykiater, samt almindelig somatisk læge, hvor hans fokus område er primært forskning ind i psykiske lidelser.
Relevante emner fra mødet opsummeres i denne sektion.

For at kunne detektere optrapning eller nedtrapning af depression/mani er ændring i adfærd centralt, dette kan eksempelvis gøres ved at se på døgnrytme og generelt aktivitetsniveau. 
Det kan også måles på kropslig stress, hvilket kan være at patienten begynder at svede eller pulsen ændrer sig uden at lave noget aktivt. 
Det er selvfølgelig vigtigt for alle personer at de har en regulær døgnrytme eller et stabilt aktivitetsniveau, men det er mere vigtigt for deprimerede.

\paragraph{Social Adfærd og Døgnrytme}
Social adfærd ændrer sig ved depression og mani.
Ved depression ændrer døgnrytmen sig, den naturlige sammenhæng mellem følelser og hvad der siges mindskes, og aktivitetsniveauet nedsættes.

Ved mani er det lidt anderledes, de sover mindre, er mere urolige, har forskellig seksualitets niveau, spiser mindre og den sociale kontakt gennem mobil kan være spam-lignende. 

Hvis døgnrytme kan detekteres ved hjælp af en mobil ville det være brugbart at se ændringer i denne.
Dette er vigtigt da hvis døgnrytmen ændrer sig er der større chance for at lidelsen er ved at ændres.

Såfremt det er muligt at måle social eller fysisk aktivitet på smartphonen, kan dette bruges til at måle ændringer hvilket kan indikere en ændring i lidelsen.

Hvis man kan måle kropslig stress kan dette også indikere en forværring af lidelsen, dette kunne f.eks. være måling af ufrivillige rystelser, sved eller puls. 

Det der anses som de mest værdifulde er døgnrytme, altså hvornår man går i seng og hvornår man står op, og på samme tid også se på om man kan registrere forstyrret søvn, altså usammenhængende søvn. 
Andre vigtige faktorer er social aktivitet og kropslig stress.

\paragraph{Ændringer}
Jørgen lagde meget stor vægt på at der ikke kan dannes et generelt billede af hvordan adfærd ser ud for en rask person, som man derefter kan sammenligne med for at finde ud af om en person har enten depression eller mani.
Det er derfor nødvendigt at finde en baseline (en normaladfærd) for individet og derefter kigge på ændringer i forhold til denne baseline.

\paragraph{Kognitive Ændringer}
Hvis man er stresset vil man få det værre. 
Kognitive egenskaber som korttidshukommelse og løsning af matematiske eller logiske problemer vil være hæmmet.
Man kan derfor forestille sig at patienten skal udføre en test der afprøver patientens evner i disse områder.

\paragraph{Visuel Repræsentation}
Hvis man skal præsentere tilstanden for patient skal den være enkel og være på patientens egne præmisser. Det er en god idé at præsentere hvordan patienten har det ud fra den registrering applikationen har gjort. Dette skal så kunne bruges som et hjælpemiddel der kan give objektiv information til patienten om deres tilstand.

Det komplette referat af mødet med Jørgen Aagaard kan ses i \cref{app:moede-med-joergen-referat}.

\paragraph{}
For at supplere viden om den aktuelle målgruppe; patienter med affektive lidelser, specifikt uni- og bi-polar affektiv lidelse, blev der arrangeret et fokusgruppeinterview med indlagte patienter.
