\subsection{Møde med Psykiatri Professor Jørgen Aagaard}\label{sec:moede-med-joergen}
Denne sektion opsummerer et møde med psykiatri professor Jørgen Aagaard på Psykiatrien ved Sygehus Syd i Aalborg \citep{misc:jorgen-aagaard}. 
Jørgen er psykiater, men også almindelig somatisk læge, men hans fokus område er primært forskning ind i psykiatriske sygedomme.
De detaljer fra mødet der er vigtige vil blive inkluderet i denne sektion

Det mest vigtige for at kunne detektere optrapning eller nedtrapning af depression/mani er forandring i forskellige opførelser som f.eks. døgnrytme og generelt aktivitetsniveau. Det kan også måles på kropslig stress, som f.eks. patienten begynder at svede eller pulsen ændrer sig uden at lave noget aktivt. 
Det er selvfølgelig vigtigt for alle personer at de har en regulær døgnrytme eller et stabilt aktivitetsniveau, men det er mere vigtigt for deprimerede.

Social adfærd ændrer sig ved depression og mani.
Ved depression ændrer døgnrytmen sig, at den naturlige sammenhæng mellem følelser og hvad der siges mindskes og at aktiviteter nedsættes.
Ved mani er det lidt anerledes, de sover mindre, er mere urolige, har forskellig seksualitets niveau, de spiser mindre og at social kontakt gennem mobil kan være spam lignende. 

Hvis døgnrytme kan detekteres ved hjælp af en mobil kan ændringer i denne detekteres, hvilket er vigtigt da hvis man ved at døgnrytmen er ved at ændrer sig ved man at lidelsen er ved at ændrer sig.

Såfremt det er muligt at måle social eller fysisk aktivitet på telefonen, kan dette bruges til at måle ændringer hvilket kan indikere en ændring i lidelsen.

På samme tid hvis man kan måle kropslig stress kan dette også indikere en forværring af lidelsen, dette kunne f.eks. være måling af rystelser, sved eller puls. 

Det der er mest værdifuldt er at registrere døgnrytme, altså hvornår man går i seng og hvornår man står op, og på samme tid også se på om man kan registrere forstyrret søvn hvor patienten vågner i REM søvn. 
Det er også vigtigt at registrere social aktivitet, og måske at kunne registrere kropslig stress.

Hvis man skal præsentere tilstanden for patient skal den være enkel og være på patientens egne præmisser. Det er en god idé at præsentere hvordan patienten har det ud fra den registrering applikationen har gjort. Dette skal så kunne bruges som et hjælpe middel der kan give objektiv information til patienten om deres tilstand.

For et fuld referat af mødet med Jørgen Aagaard, se \cref{app:moede-med-joergen-referat}.