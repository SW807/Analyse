Der kan konkluderes at en platform er blevet udviklet, der understøtter udviklingen af dataindsamlings- og analyse-moduler.
Et bevis derpå kan findes i \citet{misc:soevnrapp} og \citet{misc:surveyrapp}, hvor udviklingen af søvnestimeringsmoduler, aktivitetsmoduler, surveymoduler, opkald og sms/mms moduler er beskrevet.
Dette hænger fint i tråd med F-16 flyet som en metaforisk vision, se \cref{vision::fly}.
Tankegangen om at have en række moduler man kan hægte på platformen fungerer i praksis, men der er stadig en række problemstillinger, som bør arbejdes videre på som reflekteret i \cref{sec:refleksion}.

Derudover lød udfordringen på at udvikle en applikation til at overvåge adfærd for unipolare og bipolare patienter og gøre dem opmærksomme på adfærdsændringer.
Denne udfordring er ikke løst, men der er påbegyndt en løsning af problemet.
Fokus har gået på dataindsamlings- og analyse-aspektet af platformen, hvor det er blevet gjort muligt at udvikle moduler dertil, jævnfør tidligere nævnte udviklede moduler.
Vi kan konkludere at platformen har været tilstrækkelig modulær og fleksibel til de udviklede moduler.
Platformen danner dermed grundlaget for videre arbejde, hvor der skal fokuseres på moduler til den efterfølgende registrering af ændring i adfærd og visualisering af analyser, baseret på analyseresultaterne af tidligere nævnte udviklede moduler.

Da fokus har været at facilitere dataindsamlings- og analyseaspektet af platformen mangler der at blive udforsket hvorledes visualiserings-moduler skal faciliteres.
Vi er klar over denne mangel, men det blev bedømt vigtigere at fokusere på at gøre det nemt at indsamle relevant data først, inden at der blev lagt fokus på hvordan man skulle visualisere data.

Følgende kriterier blev opstillet i \cref{sec:process} og blev holdt i mente til udviklingen af platformen, beskrevet i \cref{arkitekturkrav}.
Det vil her blive beskrevet hvorvidt kriterierne er blevet opfyldt tilfredsstillende.
\begin{description}[style=nextline]
	\item[Modulær]
	Platformen er udviklet med primært fokus på at være modulær, og det kan konkluderes at den er det.
	Hele arkitekturen er opbygget af en manager applikation og en række separate moduler, som kan bruges i manageren.
	Dette understøttes af den fælles datagrænseflade i form af DBAccess og moduldefinitionerne.  
	Det er muligt at specificere at man er afhængig af blot et modul af en mængde af disse.
	Eksempelvis er det muligt for et analysemodul at afhænge af et accelerationsmodul.
	
	Som et eksisterende eksempel på at moduler nemt kan tilføjes og bruges af manageren er de nævnte udviklede moduler, beskrevet i \citet{misc:soevnrapp} og \citet{misc:surveyrapp}.
	Disse eksempler viser også hvorledes det er muligt at udvikle moduler, der afhænger af data fra andre moduler.
	
	\item[Fleksibilitet]
	Til at sikre at platformen er fleksibel i forhold til patienten, hvor de har kontrol over hvilke moduler, der kan køre, tilbydes dette i form af en indstillingsmenu.
	I den sammenhæng kan det konkluderes at platformen er fleksibel.
	Endvidere har patienten også kontrol over hvilke moduler, der er installeret på deres smartphone, idét hvert modul er en separat applikation på deres smartphone, som de sagtens kan afinstallere hvis ønsket.
	
	\item[Kombinerbar]
	Vi kan også konkludere at platformen sikrer kombinerbarhed.
	Med dette tænkes at data indsamlet fra diverse moduler kan benyttes af andre moduler, eksempelvis analyse moduler.
	Et klart eksempel på dette kan læses i \citet{misc:soevnrapp}, hvor data fra et søvnestimeringsmodul for acceleration og et modul for amplitude kombineres i et samlet søvnestimeringsmodul.
	
	\item[Kommunikativ]
	Det kommunikative kriterie er understøttet i den grad at data nemt kan kommunikeres til relevante moduler.
	Dog er sikkerhedskriteriet for denne kommunikation ikke på plads.
	Derudover er der ikke opsat skriverettighedsrestriktioner, og således er der altså et kriterie der ikke er blevet opfyldt og skal arbejdes videre med før man kan konkludere at en tilpas færdig platform er udviklet.
\end{description}


% Platform facilliterer udviklingen af moduler der kan benyttes
% Mangler stadig udvikling, såsom visualiserings perspektivet
% på trods af det svarer den til vision som metafor
% evaluering - integrationstest
% når stefan og mikael har taget kriterier, skriv om hvorvidt platformen holder det