\section{Enheder}\label{sec:kilder-til-sensorer}
Dette afsnit beskriver de forskellige typer af enheder udstyret med sensorer, således disse enheder kan agere kilder til dataindsamling.

De overordnede typer af enheder er \textit{smartphones}, \textit{smart wristband} og \textit{smartwatches}. 

\paragraph{Smartphone}
er en generel platform til et meget bredt brug og har derfor typisk et væld af sensorer, som fx. GPS, accelerometer og gyroskop.
Disse bruges til forskellige funktioner i telefonen, både internt og eksternt, fx. kan gyroskopet bruges til at bestemme orienteringen af smartphonen, hvor GPS kan bruges til navigation.

\paragraph{Smart wristband} % Skal det hedde dette?
er et stykke udstyr som bruges til aktivitetssporing, primært af fitness og helbredsgrunde, hvor de bl.a. bruges til at måle aktivitetsniveau og søvnmønster.
Denne aktivitetssporing påkræver en mængde sensorer, disse kan være de sensorer der findes i smartphones, men også sensorer der måler direkte på kroppen, fx. sensorer der kan måle puls eller galvanisk hud respons. 
Dog varierer det meget hvilke sensorer der findes i de forskellige smart wristbands, afhængigt af prisklasse.
Et eksempel på dette kunne være en JawBone UP3 hvor man finder sensorer der måler temperaturen af omgivelserne og kroppen, puls og galvanisk hud respons \citep{misc:jawboneup3sensors}. 
Dette er til kontrast af andre smart wristbands hvor der f.eks. ikke kan findes en sensor til at måle galvanisk hud respons.

\paragraph{Smartwatch}
er en intelligent version af normale armbåndsure, i den forstand at et smartwatch giver ekstra funktionalitet som minder om det der findes i en smartphone \citep{msic:smartwatchstate}. 
Som eksempel på dette kan man spille spil på dem, læse SMS og bruge den som medie-fjernbetjening, dog er smartwatches mindre kraftfulde end smartphones. 
Smartwatches og wristbands har fordelen at det er noget man går rundt med og derfor har direkte kontakt til kroppen det meste af tiden.
Endvidere har smartwatches mange af de samme sensorer som en smartphone og et smart wristband, ofte har de ikke sensorer til kropslige målinger, hvor de så primært fungerer som en hybrid mellem en smartphone og et smart wristbands.

\subsection{Opsummering}
Smartphones er alsidige platforme der kan bruges til at køre applikationer.
De indbyggede sensorer gør dem til en attraktiv platform at udvikle på, men smart wristbands og smartwatches har mere specialiserede sensorer som bedre understøtter logning af relevant helbredsdata.
På baggrund af dette ses det at smartphones kan bruges til meget, men hvis man vil have kropslige målinger er det en god idé at enten bruge et smart wristband eller et smartwatch med de relevante sensorer.

I \cref{tab:sensorsInDevices} er der et overblik over hvilke sensorer der findes i de forskellige slags udstyr. * indikerer at de kan findes i den slags udstyr, men det er ikke særlig tit at man finder det.

\begin{table}[h]
\centering
\begin{tabular}{|c|c|c|c|}
\hline  			 & Smartphone 	& Smart wristbands 	& Smartwatch	 	\\ 
\hline Accelerometer &  \checkmark 	& \checkmark		& \checkmark  		\\ 
\hline Gyroskop		 &	\checkmark	& \checkmark		& \checkmark		\\
\hline Kompas		 &  \checkmark	&					& \checkmark		\\
\hline GPS			 &	\checkmark	&					& \checkmark*		\\
\hline Barometer	 &	\checkmark	&					& \checkmark		\\
\hline Rumtemperatur &				& \checkmark*		&					\\
\hline Hudtemperatur &				& \checkmark*		& \checkmark		\\
\hline Lyd			 &	\checkmark	&					& \checkmark		\\
\hline Lys			 &	\checkmark	& \checkmark*		&					\\
\hline Kamera		 &	\checkmark	&					& \checkmark*		\\
\hline Puls			 &				& \checkmark		& \checkmark*		\\
\hline GHR			 &				& \checkmark*		& \checkmark*		\\ \hline
\end{tabular}
\caption{Overblik over sensorer i forskellige platforme.}\label{tab:sensorsInDevices}
\end{table}
