\section{Sensorer}
Dette afsnit beskriver de forskellige sensorer der findes og som kunne være interessant for projektet. 

\paragraph{Kamera}
Kameraet kan tage billeder og billedesekvenser i form af video optagelse.

\paragraph{Accelerometer}
Accelerometeret måler accelerationen i x,y,z retningerne i et koordinatsystem der er lagt på telefonen som vist på \cref{koord}.

\begin{figure}[h]
	\centering
	\begin{subfigure}[b]{width=0.4\textwidth}
		\includegraphics[width=0.1\textwidth]{accelerometer-telefon}
		\caption{Koordinatsystem lagt på en telefon}
		\label{koord}
	\end{subfigure}

	\begin{subfigure}[b]{width=0.4\textwidth}
		\includegraphics[width=0.1\textwidth]{accelerometer}
		\caption{sup}
		\label{acc}
	\end{subfigure}
	\caption{bla bla}
	\label{accelerometer}
\end{figure} 

\paragraph{GPS}
GPS sensoren giver en lokation som koordinater bestående af: breddegrad, længegrad og en pejling.

\paragraph{Mikrofon}
Mikrofonen kan optage lyd.

\paragraph{Lyssensor}
Lyssensoren kan måle lysniveauet i lux.

\paragraph{Pulsmåler}
Pulsmåleren måler pulsen i hjerteslag per minut.

\paragraph{Galvanisk Hud Respons}
Galvanisk hud respons sensoren giver adgang til data omkring hvor god hud er til at lede strøm, huden leder strøm bedre jo mere den sveder, og derfor giver det også data omkring hvor meget den sveder.