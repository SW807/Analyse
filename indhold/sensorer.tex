\section{Sensorer}
Dette afsnit beskriver de forskellige sensorer der findes og som kunne være interessant for projektet. 

\paragraph{Kamera}
Kameraet kan tage billeder og billedesekvenser i form af video optagelse.

\paragraph{Accelerometer}
Accelerometeret måler accelerationen i x,y,z retningerne i et koordinatsystem der er lagt på telefonen som vist på \cref{analyse:accelerometer:koo}.
Accelerometret kan konceptuelt forstås som en kugle der ruller rundt i et rum hvor væggene kan måle den kraft de bliver påvirket med.
På \cref{analyse:accelerometer:kraft} ses dette konceptuelle rum med påvirkning fra tyngdekraften. 
Sensoren vil i dette tilfælde rapportere en negativ kraft i z-aksens retning.

\begin{figure}[h]
	\centering
	\begin{subfigure}[b]{0.47\textwidth}
		\centering
		\includegraphics[width=.6\textwidth]{accelerometer-telefon}
		\caption{Koordinatsystem lagt på en telefon}
		\label{analyse:accelerometer:koo}
	\end{subfigure}
	~
	\begin{subfigure}[b]{0.47\textwidth}
		\centering
		\includegraphics[width=\textwidth]{accelerometer}
		\caption{Konceptuel tegning af et accelerometers virkemåde. Illustration fra \citet{accelerometer}}
		\label{analyse:accelerometer:kraft}
	\end{subfigure}
	\caption{}
	\label{accelerometer}
\end{figure} 

\paragraph{GPS}
GPS sensoren giver en lokation som koordinater bestående af: breddegrad, længdegrad og en pejling.

\paragraph{Mikrofon}
Mikrofonen kan optage omgivelserne ved at konvertere akustisk lyd til elektriske signaler.

Lyd kan beskrives som bølger, der kan karakteriseres ved bølgens amplitude og frekvens.
Amplituden svarer til hvor høj en lyd opfattes og måles i decibel.
Frekvensen af en bølge bestemmer hvilken tone lyden er og måles i hertz. \cite{sound}


\paragraph{Lyssensor}
Lyssensoren kan måle lysniveauet i lux.

\paragraph{Pulsmåler}
Pulsmåleren måler pulsen i hjerteslag per minut.

\paragraph{Galvanisk Hud Respons}
Galvanisk hud respons sensoren giver adgang til data omkring hvor god hud er til at lede strøm, huden leder strøm bedre jo mere den sveder, og derfor giver det også data omkring hvor meget den sveder.