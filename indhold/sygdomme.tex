\section{Afgrænsning af problemområde}
Vi skelner i vores arbejde mellem to kategorier af sygdomme, disse værende \textit{somatiske} og \textit{psykiske} sygdomme.

\textit{Somatiske} sygdomme grunder i det kropslige, som for eksempel ved Parkinsons sygdom.
\textit{Psykiske} sygdomme dækker over de psykotiske sygdomme, dvs. sygdomme med realitetssænkning og de affektive lidelser, som omhandler humørændringer og ændringer i aktivitetsniveau hos en person, eksempelvis depression \citep{misc:netpsykpsykose}. 

Valget står så på hvilken sygdoms-gren vi primært målretter vores produkt til.
Indenfor det somatiske felt er der en lang række sygdomme, der ikke umiddelbart kan hjælpes gennem en mobil løsning.
Dog kan der eksempelvis udvikles programmer til at kende forskel på patienter med Parkinsons sygdom og raske patienter, vha. stemmeanalyse \citep{6168572}.
Dette er til detektering af sygdomme, hvor dette projekt vil fokusere på tilfælde hvor sygdommen er kendt, og man i stedet vil behjælpe patienten så forværring i tilstand forhindres.
Imidlertid kan somatiske sygdomme gøre folk mere sårbare overfor affektive lidelser.

Per anbefaling af en kontaktperson, \citet{misc:janne-rasmussen}, arbejdes der med et system til affektive patienter frem for psykotiske patienter, da affektive patienter er nemmere at arbejde med og kan være interesseret i at tage vare om egen behandling, se \cref{sec:patientempowerment}.

På baggrund af dette undersøger vi affektive lidelser, som følger herefter.