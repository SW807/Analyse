\section{Sygdomme}
Vi skelner i vores arbejde mellem tre kategorier af sygdomme, disse værende somatisk, psykisk og affektiv.

Somatiske sygdomme er kendetegnende ved at være sygdomme der grunder i det kropslige, eksempelvis Parkinsons syge.
Hvis man modsat er psykotisk betyder det at på grund af sygdom har man nedsat realitetstestning \citep{misc:netpsykpsykose}.
Dette skal ikke forveksles med at man har en psykose, da det dækker over en række psykiske sygdomme såsom de affektive\citep{misc:netpsykpsykose}.
Affektive lidelser omhandler humørændringer hos en person, eksempelvis depression.

Der står et valg om hvilken sygdomsgren vi primært målretter vores produkt til.
Indenfor det somatiske felt er der en lang række sygdomme, der ikke umiddelbart kan hjælpes gennem telefonen.
Dog kan der eksempelvis udvikles programmer til at kende forskel på patienter med Parkinsons syge og raske patienter, v.h.a. stemmeanalyse \citep{6168572}.
Dette er dog til detektering af sygdomme, hvor det fokus vi har valgt at ligge er på patienter hvor sygdommen er kendt, men hvor man vil forhindre yderligere tilfælde.
Imidlertid kan somatiske sygdomme gøre folk mere sårbare overfor affektiv lidelse, men derudover kunne det være interessant at udvide systemet til somatiske patienter med mere tid.

Per anbefaling af kontaktperson \citet{misc:janne-rasmussen} arbejdes der med et system til affektive patienter frem for psykotiske patienter, da affektive patienter er nemmere at arbejde med og kan være interesseret i at tage vare om egen behandling, se \secref{sec:patientempowerment}.\fxnote{opdater til label for patient empowerment}