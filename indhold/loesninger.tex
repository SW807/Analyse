\section{Registrering af symptomer}
Dette afsnit bruger symptomer/kriterier for mani og depression beskrevet i \cref{sec:affektivelidelser} og ser på hvilke sensorer eller hvilke mobile lognings metoder der kunne bruges til at løse problemet.
Sensorerne og logningsmetoderne er beskrevet i henholdsvis \cref{sensorer} og \cref{logning}.

\subsection{Depression}
Under nuværende behandling bruges et skema der hedder stemningsregistering (\cref{sec:moede-med-psykolog}). Dette skema kunne bruges til at interagere med patienten, for på den måde at finde ud af hvordan vedkommende har det. Disse data kan dog manipuleres af patienten. Dog kan en hyppigere eller mindre brug af stemningsregistrerings funktion indikere mani eller depression.

Kernesymptomerne kunne alternativt opdages på følgende måder:

\paragraph{Man er i dårligt humør og er nedtrykt og trist}\label{darligthumor}
Her er det en mulighed at bruge kameraet til at tage billede af patienten. Disse analyseres derefter og der kigges på hvilket humør en person er i.

\paragraph{Man har nedsat lyst til at foretage sig noget, og man har mere eller mindre mistet interessen for ting man plejer at interessere sig for}
Hvis ting man plejer at have lyst til har en anden lokation end ens hjem, kan man vha. GPS'en se om interesse i de aktiviteter ophører. Der kunne indkodes nogle nøglelokationer, hvor nøglelokationer er lokationer hvor man foretager sig ting man har lyst til.
Hvis disse nøglelokationer så ikke bliver besøgt kan dette være et tegn på en depression.

Hvis man normalt laver ting man har lyst til i hjemmet er der en anden måde at afsløre en adfærdsændring. Man kunne evt. identificere forskellige aktiviteter i hjemmet vha. accelerometeret og gyroskopet. For hvis der er en ændring i data, kunne dette vise en adfærdsændring. Det kræver dog sandsynligvis meget data.

For at hjælpe behandlingen kunne man implementere en mobil løsning for huskekortet som Janne nævnte under mødet, se \cref{sec:moede-med-psykolog}. Man kan bruge notifikationer til at huske patienten på at vedkommende skal huske at lave nogle ting de engang have lyst til.

\paragraph{Man bliver hurtigt træt og har ikke så meget energi som man plejer}
Her er det muligt at bruge accelerometeret til at afgøre hvor meget man bevæger sig, hvis man antager at patienten ikke bevæger sig særlig meget grundet træthed. Dette har dog flere implikationer. For det første er det ikke sikkert man altid har smartphonen på sig. For det andet er det ikke sikkert man ikke bevæger sig selvom man er træt.
Det første problem kunne afhjælpes ved at læse et baseline aktivitetsniveau og bruge den til at sammenligne ny data med. Baseret på dette kunne man få en idé om at aktivitetsniveauet er faldet eller steget og baseret på dette ved man om man bliver hurtig træt eller ikke har så meget energi som man plejer at have. 
Desuden kunne man kigge på data fra mikrofonen, meget støj kunne indikere at patienten ikke hviler sig.

Man kunne også overvåge patientens applikationsbrug. 
En stigning kunne indikere at man bruger sin smartphone mere. 
Når man bruger sin smartphone forholder man sig for det meste roligt.
Samtidig kunne man også her se på hvor længe de forskellige applikationer bruges i forhold til hvad man plejer.
Længere applikationsbrug kunne også indikere manglende energi, da dette sandsynligvis får ting til at gå langsommere end normalt.

Her er det en mulighed at bruge kameraet til at tage billede af patienten. Disse analyseres derefter og der kigges på hvor træt en person er. 
Denne metode er en udvidning på metoden der kunne analysere humøret ved hjælp af kameraet, selve metoden er at smartphonen selv tager et billede uden at patienten ved det ved hjælp af front kameraet og derefter kan dette analyseres på.

\subsection{Ledsagesymptomer}\label{depr_ledsage}
Idéer til opdagelse af ledsagesymptomer:
\paragraph{Man har nedsat selvtillid eller selvværdsfølelse}
Tekst fra SMS beskeder og måske andre applikationer kan analyseres. Disse kan indikere om man har lav selvtillid eller lav selvværdsfølelser. Et eksempel er at hvis man bliver rost opfatter man det som skamros og prøver at bortforklare det eller snakke uden om \citep{selvtillid}.

\paragraph{Man lider af skyldfølelse og bebrejder sig selv urimeligt}
Som ovenstående kan SMS beskeder analysere for ord og sætning der tyder på skyldfølelse eller selvbebrejdelse.

\paragraph{Man har tanker om at det ville være bedre hvis man var død, eller man tænker på at begå selvmord}
Hvis man pludselig begynder at nævne selvmord eller død i sine SMS beskeder kan dette opdages, men dette påkræver at de patienter, som har selvmords tanker også deler dem hvilket gør denne idé tvivlsom.

\paragraph{Man har svært ved at koncentrere sig eller oplever at man ikke kan tænke klart}
Tekst fra SMS beskeder og måske andre applikationer kan analyseres. Her kan der ses på grammatiske fejl og om sætninger giver mening. Da man må antage hvis man ikke kan koncentrere sig eller kan tænke klart så er det svært at skrive noget sammenhængende der giver mening.

Dette symptom lægger desuden op til koncentrationsbesvær, som kan have mange kilder. Blandt dem der giver mening i denne kontekst er; arbejdsbyrde, søvnbesvær og alkohol\citep{koncentration}.
Søvnbesvær er dækket af ledsagesymptomet: '\textit{Man sover enten mere eller mindre end man plejer}'. Derfor kigger vi her på arbejdsbyrde og alkohol.

Til at påvise arbejdsbyrden kan man bruge GPS sensoren til at se hvor meget man opholder sig på arbejdet. Dette kan dog være misvisende hvis man arbejder hjemme.

Til at forsøge at vise alkoholforbrug er der muligt at indkode GPS lokationer på barer og værtshuse for at se om patienten opholder sig der mere end normalt.
Dette fanger dog kun alkoholforbrug der ikke foregår i hjemmet.

\paragraph{Man er enten urolig og hvileløs, eller også er ens bevægelser nærmest gået i stå}
Accelerometret kan bruges til at se om patienten bevæger sig rastløst rundt eller om bevægelserne er gået i stå.
Her er det vigtigt at kigge over en længere periode for at kunne sammenligne adfærdsmønstret for patienten.


\paragraph{Man sover enten mere eller mindre end man plejer}
For at genkende patientens søvnmønster er det nødvendigt at smartphonen ligger i eller tæt på sengen, eller at patienten har et armbånd eller ur med sensorer i.
Det er desuden nødvendigt for et armbånd eller ur at pulsen kan måles kontinuerligt gennem natten.

Det vil da være muligt at identificere bevægelser og puls i løbet af natten hvilket vil kunne give et billede af hvornår patienten er gået i seng og stået op, samt om patienten vågner i løbet af natten.

Man kunne også gå ud fra at patienten slukker skærmen på deres smartphone når de går i seng, og de tænder den når de står op. Hvis dette er sandt, kan dette bruges til at måske detektere søvn.

\subsection{Mani}

\paragraph{Have været opstemt, eksalteret og irritabel}
Som nævnt under \cref{darligthumor} kan et billede af patienten bruges til at vurdere humøret. 
Det kan derved undersøges på ansigtet om man er opstemt eller irritabel.

Her kan også skemaet nævnt i \cref{darligthumor} bruges til at finde patientens egen opfattelse af sit stemningsleje.

Patientens stemme kan muligvis benyttes til at detektere om man er i en maniperiode, som det diskuteres i \cref{janne_ideer} under ``lyd''. 
Her nævnes det at man taler hurtigere og snakker mere når man er manisk.

Ifølge Jørgen Aagaard medbringer forværring af affektive lidelser som regel mere kropslig stress for patienten, og derfor kan man muligvis bruge puls eller GHR til at detektere ændringer i stemningsleje \cref{sec:moede-med-joergen}

\paragraph{Man er hyperaktiv, rastløs og urolig}
Som nævnt under \cref{depr_ledsage} kan man bruge accelerometret til at sammenligne patientens adfærdsmønster.

\paragraph{Man føler et indre pres for at tale uafbrudt}
Her kan man bruge lydoptagelse af patientens samtaler for at analysere om der er færre pauser end der plejer.

\paragraph{Man har tankeflugt, hvor tankerne springer fra emne til emne}
Hvis patienten har et basis niveau for deres applikations brug på deres smartphone og dette pludselig ændrer sig til at være mere flygtig, altså at patienten skifter tit og hurtigt mellem applikationer, kan dette antyde at personen har tankeflugt.  

\paragraph{Man har nedsat behov for søvn}
Her kan metoden som blev nævnt under søvn i \cref{depr_ledsage} benyttes til at se om søvnen bliver kortere end normalen.

\paragraph{Man har forhøjet selvfølelse, grandiositet}
Samme metode som omtalt i \cref{depr_ledsage} under ``Man har nedsat selvtillid eller selvværdsfølelse'' kan man analysere indholdet af SMS beskeder, og i dette tilfælde kan man lede efter ord der tyder på at patienten har en forhøjet selvfølelse.

\paragraph{Man er usamlet eller bliver konstant distraheret}
Dette kunne måske detekteres på en lignende måde som ved tankeflugt.

\paragraph{Man har større seksualdrift end normalt}
Hvis man som under andre metoder kan analysere indholdet af SMS beskeder, kan man analysere om sex bliver omtalt mere ofte.



