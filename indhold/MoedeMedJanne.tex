\section{Møde med Psykolog Janne Rasmussen}\label{sec:moede-med-psykolog}
Vi aftalte et møde med en psykolog, Janne Rasmussen, på Psykiatrisk Sygehus også kalder Psykiatrien i Aalborg.
Hun har arbejdet på Psykiatrien i 10 år, hvor hun lige nu sidder på ambulatoriet for patienter med psykose og forsøger at diagnosere patienterne. 
Hun er primært involveret med affektive patienter, som unipolar depression eller bipolar depression, hun har også været involveret med patienter som lider af angst, OCD og andre affektive lidelser.

Da hun primært er involveret med affektive lidelser bliver det mest lagt vægt på denne og vi vil kort introducere dette emne, se \secref{sec:affektivelidelser} for flere detaljer om denne slags lidelser.
Derudover, er de patienter med affektive lidelser mere tilbøjelige til at benytte en mobil applikation som behandlingsmetode.
Derfor virker det til at patienter med affektive lidelser er en god målgruppe for denne slags applikation.

Hvad der følger er beskrivelser af de ting Janne nævnte til mødet, suppleret med information fra andre kilder. 
Dette skal så bruges til inspiration til selve produktet.

\subsection{Affektive Lidelser}
Affektive lidelser er tilbagevendende, hvor det kan være svært at holde en hverdag.
Eksempler på affektive lidelser er f.eks. unipolar depression eller bipolar depression.
Hvad vores fokus kunne være at se på advarsels tegn af tilbagevendende maniske eller depressive perioder, hvor man f.eks. kunne se på døgnrytme hvor unipolare sover for meget og bipolare sover for lidt i maniske perioder men kan også opleve depressive perioder hvor de sover for meget.

Det er vigtigt at identificere søvn da det siger meget om deres situation, og ud fra søvn baseret på ændringer kunne man identificere om de er på vej ind i en manisk eller depressiv periode og om de er på vej ud af den periode. 
Specifikt hvis man kan se om patienterne begynder at blive op længere eller sover længere kunne man måske hjælpe dem tidligere.

For at diagnostisere unipolar depression i patienterne bruges der et værktøj kaldet Hamilton skalaen, dette er et spørgeskema af en slags hvor en speciallæge objektivt vurderer patienten baseret på en række spørgsmål, som f.eks. kunne spørge ind til aktivitetsniveau, døgnrytme, energiniveau og andet.
Når denne er blevet udfyldt får man en score som ligger en i én af 4 kategorier: Ikke, let, moderat eller svært deprimeret. 
Denne kan så bruges som en måling af patienternes sindstilstand og hvordan behandlingsforløbet går.

Når patienterne er indlagt er det vigtigt at de har en meget struktureret hverdag da de ikke kan overskue når der sker noget uventet, derudover er det også vigtigt fordi ellers bliver de depressive bare liggende i deres seng hele dagen.
Når patienten skal udskrives er det vigtigt for dem at de har nogle værktøjer så de kan komme tilbage til en regulær hverdag, da risikoen for tilbagefald eller selvmord er meget stor på dette tidspunkt. 
For at finde disse værktøjer er det vigtigt for behandleren at identificere hvad der gør at patienten er på vej ind i en mani eller depressions periode.
Dette kunne f.eks. være ting som dårlig økonomi, fødselsdage, eller andre stressede situationer.
Endvidere er det også vigtigt for behandleren at identificere hvad det kan få patienten ud af sådan en periode, hvilket er de lystbetonede aktiviteter, motion eller et regulært søvnmønster.

I hverdagen giver psykologen patienter andre værktøj, som f.eks. huske kort, skema lægning over deres søvn samt ovennævnte lystbetonede aktiviteter, nedskrivning af tanker, stemningsregistrering hvor patienten hen over en måned sætter kryds i et skema over hvordan de har det og få dem til at udføre lystbetonede aktiviteter og ikke kun pligt opgaver som de gør for andre. 
Dette bruges til at give patienterne struktur og overblik over deres behandlingsgang, samt at give psykologen et indtryk over hvordan tingene går.  

Ved behandling af patienter der falder i det affektive spektrum er en af fordelene at nogle af dem er meget autoritets tro og pligt overfyldende.
Dette betyder så at hvis patienterne bliver fortalt at de skal forsøge at gøre et eller andet vil de ofte forsøge at gøre det.
Hvilket gør at de er mere tilbøjelige for at acceptere nye behandlings metoder, som f.eks. brug af en mobil applikation som kan identificere mønstre i deres opførsel som kan føre til tilbagefald. 
Dette er til kontrast med f.eks. psykotiske paranoide, som ikke vil blive overvåget eller fortælles hvad de skal gøre. 

\subsection{Præsentation af Idéer}
Denne del af mødet byggede på en brochure med idéer til hvordan forskellige sensorer i telefon eller ekstern hardware kan bruges til at hjælpe med at behandle sindslidelser. 
Denne brochure kan ses i \appref{app:brochure}.
Ikke ethvert emne i brochuren blev dækket, tastatur og tekstanalyse blev ikke dækket.

\begin{description}[style=nextline]
\item[Billed analyse]
	En idé her var at tage et billed af patienten og analysere ansigtsmimik(humør), automatisk så patient ikke kan snyde og tage et billed hvor de smiler.
	Dette virkede til at være en interessant emne som kunne blive bruges i produktet, dog ville nogle patienter være utilpasse med dette og derfor er det vigtigt at gøre det valgfrit og bare tilbyde det som et værktøj til dem som kan håndtere det.
	
	En anden idé til billed analyse er at tage en video sekvens, hvor blitzen bliver udløst og derefter analysere pupil reaktion.
	Dette virkede ikke til at være en god idé fordi stoffer, medicin og andre adfærdsændrende materialer samt andre lidelser kan påvirke pupil reaktion så det er begrænset hvad der kan reelt analyseres her.
\item[Accelerometer]
	En idé til accelerometer er at se på patientens gangart samt aktivitetsniveau. Disse skulle så benyttes til at se om en person er i mani eller depressions periode.
	
	Selve gangarten menes ikke at kunne blive benyttet da gangarten ikke ændres for patienten, dog kan aktivitetsniveauet bruges, fordi når patienten har en depression bevæger de sig ikke ret meget, og når patienten er i mani perioden bevæger de sig rigtig meget. 
	Her skal man dog være opmærksom på hvad patientens habituelle aktivitetsniveau, så man kan observere ændringer i det.
	Dog kan det være svært at skelne mellem lidelsens symptomer og når patienten dyrker motion, her vil det være en idé at benytte andre sensorer til at identificer dette.
\item[Lokation]
	Idéen til at bruge GPS til at lokalisere patienter er at man kan enten bruge dette til at observere hvor meget de bevæger sig eller hvor de opholder sig mest. 
	
	For mange patienter ville dette ikke sige særlig meget, da de ofte enten bare er derhjemme, et værested eller på arbejde og at de som regel ikke har særlig store sociale bevægelsesmønstre. 
	Dog kan det nok godt bruges til bipolare patienter, da når de er i en manisk periode vil de være mere omkringfarende. 
	Her skal man igen se på det habituelle niveau og så se om opførslen ændrer sig. 
	Man kan muligvis bruge det til at se om patienten er i bedring ved at se på deres bevægelsesmønster og dette kan informere behandleren om at de gør det de bliver bedt om at gøre. 
	Som et eksempel på hvordan behandleren kunne blive informeret om dette ville være at hvis de har fortalt patienten at de skal opsøge sociale situationer mere ofte, og lokationen bare viser dem hjemme.
\item[Lyd]
	Idéen her er at bruge en baggrunds optager i telefonen og derefter analysere personens tale, som f.eks. stemmeleje. 
	
	For bipolare patienter observeres det tit at de taler hurtigt, at de laver flere jokes, at de er småsyngende og opsnapper ord. Det er ikke rigtig stemmeleje, men mere stemningsleje der betyder noget. 
	Dog vides det ikke om patienternes stemmeleje ændre sig når de er i en mani eller depressions periode, og skal derfor undersøges.
	Det kan godt være der ligger for meget i det til at det kan effektivt analyseres. 
\item[Lys]
	Idéen her at at detektere om patienten opholder sig i mørke eller i lys, dog skal man kunne finde en måde at identificere om mobilen ligger i lommen før man kan bruge det til noget.
	
	Sollys hjælper mod depression, men det åbenbart ikke primært affektive hvor dette betyder det store, det er mere ved f.eks. psykotiske paranoide at man observere at de ruller gardinerne ned da de ikke vil have at nogen kigger ind. 
\item[Opkaldsoversigt]
	Idéen her er at analysere på hvor meget patienten har snakket, hvor lang tid ens opkald varer og hvor mange opkalder man misser.
	
	Dette ville kunne bruges til bipolare lidelser da når de er i mani perioder vil de være meget sociale og derfor snakke meget. 
	Dog kan det nok ikke bruges ved unipolare depressive da de ikke er særlig sociale i deres habituelle perioder alligevel.
\item[Applikation brug]
	Idéen her er at analysere på hvilke applikationer der oftest bliver brugt på telefonen. Her ville man kunne se på trends og forholde brug til det habituelle niveau og på samme tid ville der nok være god idé om patientens brug af telefonen ændrer sig.
	
	Der menes dog ikke her at dette kan bruges til noget.
\item[Puls]
	Idéen her er at man måler puls med ekstern hardware som en JawBone eller lignende. Ud fra dette kan man se om patienten er fysisk aktiv.
	
	Eftersom mange af patienterne har somatiske, altså fysiske, problemer som stress eller højt blodtryk er det derfor vigtigt at kunne skelne disse ting fra den fysiske aktivitet.
	Hvis dette kan lade sig gøre, kan det godt bruges. 
	
	Forbundet med brug af hardware som JawBone ville der være en positiv side effekt idet at patientens kendskab til at man bliver overvåget af en JawBone el. andet vil være en motiverende faktor for mange patienter. 
\item[Søvnmønstre]
	Idéen her er at man analysere patienternes søvnmønstre ved at måle når skærmen tændes, da de fleste slukker skærmen på siden telefon når man går i seng og tænder den når man står op. 
	
	Specielt det med at analysere patienters søvnmønster var en god idé, men man skal være opmærksom på at en patient kan ligge vågen selvom telefonen har slukket skærm, og at en patient kan falde i søvn igen efter en mobil alarm er blevet slukket.
	Dette kan også være meget afvigende hvis patienten ikke kigger på klokken.
\item[Galvanisk hud respons]
	Idéen her er at man måler hvor stresset en person er ud fra hvor meget de sveder. 
	Endvidere er idéen også at måle patientens søvnmønster for at se hvor godt patienten sover.
	
	Det med om en person er stresset er måske ikke en mulighed da der kan være flere ting der spiller ind når man sveder meget.
	Derimod vil måling af søvnmønstre være meget brugbart da dette ofte siger meget omkring en patients tilstand, da hvis de sover meget kort og har svært ved at falde i søvn kan dette indikere at de er i mani periode.
	Omvendt hvis patienten sover meget længe kan det indikere at patienten er i depressions perioden.
	Endvidere, alt efter hvordan man sover i løbet af natten kan også være indikatorer på om man er i en af perioderne, og dette kan derfor også bruges. 
\item[Andre]
	Ud fra de ting der blev snakket om, blev samtalen også vendt mod nogle ting der ikke var tænkt på, hvilket var at segmentere patienterne på hvordan de bruger deres telefon, 
	Derudover kunne det være en god idé hvis man kunne give påmindelser til patienterne, specielt dem med kognitive problemer(f.eks. hukommelses problemer), da dette ville kunne hjælpe dem i deres hverdag.
\end{description}

For et fuld referat af mødet med Janne Rasmussen, se \appref{app:moede-med-janne-referat}.