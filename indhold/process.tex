\section{Process}\label{sec:process}
\emph{Process view} har til formål at kunne evaluere om projektet er på vej i den rigtige retning. 
Til dette er det nødvendigt at holde fokus samt at have nogle kriterier at evaluere op imod.

\subsection{Facilitation}
\emph{Facilitation} opstiller et fokus som skal bruges i forbindelse med evaluering af løsningen.
Da denne del af projektet har en platform i fokus er det centralt at sikre at denne platform kan bruges til de formål den er tiltænkt når specifikke datamilder skal tilføjes.
Projektets fokus kan derfor beskrives således:
Der er fokus på at platformen er ekspandérbar, således at systemet kan udvides som information opnås i forhold til viden omkring lidelser og symptomer, samt tilgængelighed af nye datakilder.
\stefan{tror måske det er for løst}

\subsection{Kriterier}\label{firstsubseckriterier}
Her vil de vigtigste kriterier, som skal opfyldes for at kunne kalde projektet en succes, blive præsenteret.
Disse kriterier dækker over to overordnede områder: i forhold til bruger, og i forhold til platformen i sig selv.

\begin{description}[style=nextline]
	\item[Modulær] 
	Da forskellige individer har meget forskellige symptomer skal det være muligt at tilføje og fjerne datakilder samt analyser af disse, så den enkelte patient får den bedst mulige behandling.
	\item[Fleksibilitet]
	Det skal være nemt at modificere funktionalitet til platformen, da platformen skal kunne tilpasses til forskellige individer.
	\item[Kombinerbar] Eftersom vi prioriterer modulærbarhed skal ansvarsområder være lette at separere så indhentede data kan bruges på tværs af systemet.
	\item[Kommunikativ] Data skal være tilgængeligt på tværs af systemet, men skal samtidig være beskyttet mod at blive redigeret af uvedkommende.
\end{description}