\section{Patient Empowerment}\label{sec:patientempowerment}
Patient empowerment omhandler en inddragelse af patienten i egen behandling.
Dette inkluderer at give redskaber til patienten, der gør det nemmere for patienten at vurdere sin egen sygdomssituation og foretage informerede behandlingsvalg udfra den vurderede sygdomssituation.
Det er en tanke der er gøres en indsats for at implementere af sygehusvæsenet \citep{misc:patientpowerhovedstaden}.
Dog er det vigtigt at patient empowerment udføres ordenligt.
Steder hvor strategien ikke er at foretrække er hvis patienten føler sig utryg ved at skulle have et sådant ansvar for egen behandling, hvor patienten helst ikke vil indrages yderligere i behandlingen.
Derudover kræver det at patienten bliver velinformeret om hvordan han skal monitorere egen sygdom og handle udfra det.

Til at understøtte denne informering af patienten om egen sygdom kan diverse empiri kilder være fordelagtige, såsom søvnændringer og aktivitetsændringer\citep{misc:jorgen-aagaard}, og er derfor undersøgt nærmere i dette projekt.
Måder man kan understøtte denne selvbehandling er ved at se på diverse mønstre og handlinger, hvilket inkluderer trends, cykler og interventioner.

\section{Trends}
Trends beskriver tendenser i en persons adfærd, eksempelvis at man går i seng kl. 22 og står op kl. 7 eller går en tur hver aften.
Det interessante med trends er at se på ændringer, da disse kan antyde en ændret sindstilstand.
En ændring kunne være at personen begynder at gå i seng kl 4 eller sover til kl 12 i en periode.

\section{Cykler}
Cykler omhandler fænomenet at en persons sindstilstand går i cykler.
Dette er interessant da ens adfærd bliver påvirket af ens sindstilstand.
Ved at kunne detektere den nuværende sindstilstand og har kortlagt cyklus for den pågældende person, kan dette bruges til at forudsige det næste stadie, der kunne være en depression.

\section{Interventioner}
Interventioner omhandler handlinger der kan afbryde en sindstilstand eller påbegyndende sindstilstand.
Eksempelvis ved en påbegyndende depression kan man foretage sig lystbetonede aktiviteter.
Dette kan være farligt at overlade til et program at diktere, da forkerte vurderinger kan have voldsomme konsekvenser.
En måde dette kunne være et problem er ved en manisk patient der bliver vurderet depressiv, der ville en lystbetonet aktivitet, såsom at gå en tur, ikke gavne og kunne gøre situationen værre.