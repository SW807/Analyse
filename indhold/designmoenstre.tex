\section{Design Mønstre}
For at få lavet en fint struktureret platform har vi udforsket forskellige design mønstre vi finder nyttige til at sikre en generel og fleksibel platform.
Dermed ikke sagt at alle design mønstre beskrevet benyttes, men blot at de har været værd at overveje.

\subsection{Facade}
%Beskrivelse af formålet med facade
Vi tager inspiration i det strukturelle facade design mønster.
Facade er et design mønter der tilbyder et simpelt interface til et komplekst subsystem \citep{DATGANGOFFOUR}. 
Det er et middel til at få en løsere kobling mellem et system og klienterne der bruger systemet.
Dette gøres ved at opstille en facade mellem klienterne og subsystemet.
Facaden sikrer et ensartet interface til brug af subsystemerne, men hvor man abstraherer over disse.
At implementere dette sikrer også større modularitet idét at ændringer i subsystemerne ikke kræver ændring i klienterne, da de blot bruger facaden der kan justere for disse ændringer.

%Hvordan kan det være nyttigt i vores tilfælde
Da vi arbejder på en mobil platform er det værd at tage højde for lagerstyring og abstraktion derover.
Da der er begrænset plads på en smartphone kan det blive relevant at lagre noget af det indsamlede data i skyen.
Men det kan være nyttigt at abstrahere over hvor data indsamles og lagres, og er fint i tråd med de krav om generelitet og fleksibilitelt for platformen.

Af denne årsag vælges der at bruge facade designmønstret til at abstrahere over lagringen af data.

\als{Nævn hvordan det implementeres -- DBAccess og ContentProvider}



%Henvis til DBAccess?
\subsection{Observer}
For at moduler kan tilgå hinandens data uden at have adgang til at tilføje eller ændre i det har vi brug for en model til at dele data.

En overvejelse har været at bruge observer mønstret til dette.
Med observer mønstret kan man hver gang der er nyt data sende opdateringen ud til alle de moduler der har brug for dataen \cite[p.~244]{gamma1994design}.

Ved brug af dette mønster vil hvert modul der abonnerer på en datakilde kunne udføre sine beregninger med det samme når dataen er tilgængelig.
Det betyder på den anden side at det ikke er muligt for modulet selv at bestemme hvornår tunge beregninger skal udføres.
Hvis et modul skal udføre tunge beregninger vil det være fordelagtigt at kunne udføre disse når brugeren ikke bruger telefonen, eventuelt om natten når telefonen alligevel er sat til strøm.
Vi har derfor valgt ikke at bruge observer mønstret til deling af data.




Vores løsning til dette problem er i stedet at bruge adgangsbeskyttelse på databasen.
Alle moduler kan skrive til sine egne tabeller, og kan læse fra alle tabeller.
På denne måde kan ingen ændre i tabeller som det ikke er meningen der skal ændres i, og alle har adgang til det data de har brug for, når de har brug for det.
\stefan{adgangskontrol er så vidt jeg husker ikke implementeret, hvordan nævner vi dette?}



\subsection{factory pattern}
Til at strukturere den modulære arkitektur der er tiltænkt, kan et Factory Pattern overvejes.
Dette design mønster kan tages i brug når man ikke er helt sikker på hvilken klasse de objekter man skal lave er, hvilket man kunne forstille sig kunne ske i en arkitektur hvor komponenter tilføjes og fjernes efter forgodtbefindende.
Måden dette gøres på er, at der laves en enten abstract eller virtual create metode, som kaldes når objekter skal laves.
Create metoden laves som virtual hvis der kan defineres et fornuftigt standard tilfælde, og abstract hvis der ikke kan.
Metoden kan så overrides i de tilfælde hvor det er nødvendigt.
Når metoden kaldes, laver den det objekt der passer til den givne situation.

Et problem der kan være med dette design mønster er at den kræver nedarvning fra det generelle objekt til det specifikke.
Dette er problematisk hvis man har brug for at nedarve fra et andet objekt at det sprog man skriver i ikke understøtter multiple nedarvning, hvilket Java som vi arbejder med ikke gør.
Dette er dog ikke noget problem hvis man ikke havde brug for nedarvning ellers.

%vores vald
Dog vælger vi ikke at bruge dette mønster i sin helhed.
Dette valg grunder i, at vi ikke nødvendigvis behøver at lave et objekt af hver moduls type, men kun behøver at sætte den metode i gang der definerer modulets funktionalitet.
De dele af mønsteret vi tænker stadig er brugbart er definitionen af en abstrakt klasse, der har en metode der kan kaldes fra andre steder, subklasserne står for at overskrive så den funktionalitet det giver er den der passer.

Den modulære del der omhandler at vise resultater til brugeren gøre god brug af factory pattern.
Specifikt her vurderes det at være meget brugbart, da man kunne forstille sig disse har gavn af bare at kalde en metode der for eksempel siger createGraphView og så får man det objekt der passer bedst til den data man har.
