\section{Vision}
For at få et 'Vision' af projektet overvejes der følgende koncepter fra \citet{art:essence}, derudover er der enkelte oversatte citater derfra.
\textit{Metafor}, \textit{Ikon}, \textit{Prototype} og \textit{Proposition} er de forskellige muligheder som bogen præsenterer som kan bruges til at repræsenterer en vision.

Ifølge bogen er Metafor \textit{et ord eller sætning der repræsenterer eller figurativt beskriver en løsning uden af være bogstavelig anvendelig.}
Bogen beskriver Ikon som \textit{et symbol der eksemplificerer en eller flere nøgle kvaliteter i en idé.} Det går på de visuelle, hvor øvelsen er hvordan man konkretiserer en eller flere nøglekvaliteter af en idé. Dette er eksempelvis hvor den teknologiske del er nemt udførlig, men hvordan det æstetiske i en løsning ikke er umiddelbar.
Derudover beskrives Prototype som \textit{en kompleks og konkret repræsentation af et vision. Prototyper er ufærdiggjort software som traditionelt bruges til at bekræfte udviklernes opfattelse af krav.} Denne bruges til at få feedback på en ufuldstændig software løsning, og er en fysisk repræsentation af løsning. 
Til sidst Proposition er \textit{er et udsagn som er velbegrundet og har positive egenskaber.} Denne svarer basalt set til en traditionel problemformulering. 

Vores vision for projektet repræsenteres ved hjælp af metafor repræsentationen, specifikt formuleres det ved hjælp af tre metaforer.
Vi bruger metafor på baggrund af at det giver et løst overblik over selve løsningen som giver plads til at ændre på løsningen uden at miste overblikket. 

De tre metaforer er \textit{Objektiv dagbog}, \textit{Fitness tracker} og \textit{F16 fly}.
At formulere vores vision som en metafor lader os koncist specificere hvad der er nøgleaspekterne af produktet.

Den objektive dagbog danner tanken om en dagbog baseret på objektive datakilder, hvilket svarer til sensor data, brugsdata etc.
Alt sammen data der kan indsamles uden brugerinteraktion.

Fitness trackeren som metafor planter tanken om en applikation der løbende evaluerer ens præstationsevne, hvilket kan oversættes til mentalt helbred.

F16 fly metaforen henvender sig til platforms designet, der er tiltænkt at være en meget modulær og kraftig platform, ligesom det er tilfældet med F16 flyet hvor man kan hægte en lang række komponenter på alt efter hvad der er brug for i den pågældende situation.