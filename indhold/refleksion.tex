\section{Notifikationer}\mikael{Det er fra vores individuelle rapport, passer bedre ind her, skal nok rettes lidt til}
En naturlig udvikling af det udviklede survey modul ville være, i stedet for at indhente oplysninger fra brugeren, at præsentere dem.
Altså at kunne notificere brugeren med information.
Dette kunne enten ske på faste tidspunkter eller ved at et analyse modul registrerer en ændring i et af de monitorerede adfærdsmønstre.
Ligeledes kan patienten modtage en status-notifikation, der fx kan beskrive en opsummering over hans sociale aktivitet målt over den seneste uge.

\paragraph{Det udviklede modul}
Survey modulet tillader at indsamle information fra brugeren via en række forskellige typer spørgsmål (se \cref{survey:spg}).
Modulet fungerer herved som et sensor modul der anvender brugerens respons som dets input.
Modulet adskiller sig dog fra andre sensor moduler idet det anvender en grafisk brugergrænseflade til at indsamle information.
Givet relationen mellem de forskellige moduler (sensor-, analyse- og visningsmoduler) introducerer ovenstående en ny problemstilling.

Da information kun kan overføres fra sensor til analyse og videre til visning, og ikke i den modsatte retning, vil notifikationer skulle implementeres som et visningsmodul.
Kun på denne måde kan en notifikation præsentere information fra underliggende lag.
Det betyder at notifikationer og spørgsmål vil skulle udvikles som to separate moduler.
Modulerne kan defineres som en del af samme applikation for derved at kunne anvende samme kode-base, men vil i sidste ende skulle fungere som to moduler.

Skulle man yderligere ønske at præsentere information i et spørgsmål ville det kræve en sammensmeltning af de to typer moduler.
En sådan kombination af moduler er problematisk under den nuværende arkitektur, da den vil kræve både input og output i samme modul.