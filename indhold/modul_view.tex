\paragraph{Visualiserings Modul}\label{sec:visningsmodul}
\mikael{Synes dette er en blanding af design overvejelser og refleksion, især eftersom dette på ingen måde er implementeret. Desuden er det også uendeligt langt ift. at der ingen dybere beskrivelse er af data (sensor) eller analyse moduler.}
Som udgangspunkt er \textit{visualiseringer} forskellige visualiseringer af analyse modulernes output.

De eneste restriktioner der stilles for data der kan læses fra analysemoduler, er de formater der kan gemmes i databasen.
Det betyder at det mulige output fra et analysemodel er meget fleksibelt, hvilket kræver en tilsvarende fleksibilitet i visualiserings-modulerne.

En simpel løsning vil være, at der for hvert analysemodul skal defineres et visualiserings-modul.
Herved sikres det at alle analysemoduler har en visualisering der præcist repræsenterer genererede data.
Det er dog nemt at finde problemstillingerne i denne løsningsmodel, da det giver meget lav genanvendelse af eksisterende moduler.
Eksempelvis kan man nemt forestille sig forskellige datasæt der kan repræsenteres ved hjælp af en to-dimensional graf.
En løsning der ville kræve en ny implementering af en sådan visualisering, for hver analyse modul er ikke fleksibel nok.

Alternativt kan visualiseringer beskrive en liste over de analyser de kan anvendes på.
På den måde kan visualiseringer anvendes på mere end et analyse modul.
Omend dette løser problemet til en vis grad vil problemet stadig eksistere når der tilføjes nye analyse moduler.
Her vil visualiseringerne enten ikke have kendskab til de nye moduler eller de vil skulle opdateres løbende.
Altså er fleksibiliteten af denne løsning heller ikke god nok.
Man kan naturligvis også lade alle analyse moduler beskrive en liste over visualiserings moduler.
Men denne opbygning vil give samme problemstilling som beskrevet herover.

En mere fleksibel metode til håndtering knytningen mellem visualiseringer og analyser er at deducere ud fra analysens tabel/kolonne signatur, hvilke slags data der kan aflæses.
På denne måde kan der laves visualiseringer der repræsenterer bestemte signaturer, og der opstår en implicit binding mellem visualiseringer og analyser for de hvor de to beskrivelser stemmer overens.

I mange tilfælde vil denne implicitte binding være nok, dog kan det forestilles at analyserne vil være forskellige og til tider komplekse, hvilket kan gøre det nødvendigt at nærmere specificere hvordan den tilgængelige data kan vises.
Der er også den mulighed at analysemoduler kunne have ekstra database tabeller, der ikke skal vises, som så kunne virke forstyrrende for den implicite binding.
Her kunne der tilføjes noget meta-data på analysens tabeller/kolonner, der beskriver den på sådan en måde, at det ville kunne bruges til visualiseringer der ikke nødvendigvis implicit kunne knyttes til det.
Det kunne også være muligt at tilføje information om kriterier for visualiseringer i analysemodulers JSON filer, og så overlade det til Manager komponenten at matche et analysemoduls kriterier med de kriterier visualiseringsmoduler opfylder.

\subparagraph{Eksempel}
Her bruges light, som er en simpel sensor og analyse, der fra sensoren indhentes løbende lys-niveau (i lux).
Analysen tager den senest indsamlet data og giver et gennemsnit.
Det antages at alle tabeller har en tids-kolonne, der beskriver enten hvornår data blev indsat i databasen, eller i nogle tilfælde af analyser overføres tiden fra sensor-data.

\begin{lstlisting}
{
  "name": "lightAvg",
  "_version": 1.0,
  "tables": [
    {
      "name": "lightAvg",
      "columns":[
        { "name": "lightAvg", "dataType": "REAL", "_unit": "lux" }
      ]
    }
  ],
  "dependencies": [
    [{ "name": "light" }]
  ]
}
\end{lstlisting}

Et eksempel på et visualiserings-modul som passer på denne type analyse kunne være en simpel 2D graf, som viser den gennemsnitlige belysning over tid.

\begin{lstlisting}
{
  "name": "2dgraph",
  "_version": 1.0,
  "_type": "view",
  "view": {
    "layout": "2dgraph.xml",
    "data": [
      { "name": "x", "dataTypes": ["INTEGER"], "fromTimestamp": true },
      { "name": "y", "dataTypes": ["INTEGER", "REAL"] }
    ]
  }
}
\end{lstlisting}
 
Denne visualisering beskriver en 2D graf, som bruger layoutet i \texttt{2dgraph.xml}.
På x-aksen vises tiden (der står INTEGER fordi SQLite ikke har datorepræsentation ud over unix-time).
Her bruges det tidsstempel som forventes på alle tabeller.
Til y-aksen kan bruges alle analyser, som blot har en enkelt kolonne af heltal eller decimal-tal værdi.

\paragraph{Administrering af Visualiseringer}
Ligesom sensorer og analyser, skal visualiseringer også administreres af brugeren.

For at holde det så simpelt som muligt, kunne man, ligesom ved sensorer/analyser, have to niveauer af administration.
Som udgangspunkt vil alle installerede visualiseringer automatisk knytte sig til de aktiverede analyser.
Derudover vil man i de avancerede indstillinger kunne aktivere/deaktivere visualiseringer for de enkelte analyser.

På denne måde vil der automatisk blive genereret en liste af visualisering/analyse kombinationer, men med mulighed for at slå nogle af kombinationerne fra.
Det vurderes at muligheden for at slå visualiseringer fra skal være under avancerede indstillinger, da ikke teknisk erfarne brugere sagtens kunne forvirres af alle de potentielle visualisering/analyse kombinationer de ville præsenteret for.
Grunden til at det stadig er med som en mulig customization er, at der er stor fokus på at brugerne skal have mulighed for at styre hvad programmet skal gøre. 
