\subsection{Visnings Modul}

Som udgangspunkt er \textit{visninger} forskellige visualiseringer af analyse modulernes output.

De eneste restriktioner der stilles for det data der kan læses fra et analyse moduler er de formater der kan gemmes i databasen.
Det betyder at det mulige output fra et analyse model er meget fleksibelt, hvilket kræver en tilsvarende fleksibilitet i visningsmodulerne.

En simpel løsning vil være at der for hvert analysemodul skal defineres et visnings modul.
Herved sikres det at alle analysemoduler har en visning der præcist repræsenterer det genererede data.
Det er dog nemt at problemstillingerne i denne løsningsmodel, da det giver meget lav genanvendelse af eksisterende moduler.
Eksempelvis kan man nemt forestille sig forskellige datasæt der kan repræsenteres ved hjælp af en to-dimensional graf.
En løsning der ville kræve en ny implementation af en sådan visning for hver analyse modul er ikke fleksibel nok.

Alternativt kan visninger beskrive en liste over de analyser de kan anvendes på.
På den måde kan visninger anvendes på mere end et analyse modul.
Omend dette løser problemet til en vis grad vil problemet stadig eksistere når der tilføjes nye analyse moduler.
Her vil visningerne enten blot ikke have kendskab til de nye moduler eller de vil skulle opdateres løbende.
Altså er fleksibiliteten af denne løsning heller ikke god nok.
Man kan naturligvis også lade alle analyse moduler beskrive en liste over visnings moduler.
Men denne opbygning vil give samme problemstilling som beskrevet herover.

En mere fleksibel metode til håndtering knytningen mellem visninger og analyser er at deducere ud fra analysens tabel/kolonne signatur, hvilken slags data der kan aflæses.
På denne måde kan der laves visninger der repræsenterer bestemte signaturer, og der opstår en implicit binding mellem visninger og analyser for de hvor de to beskrivelser stemmer overens.

I mange tilfælde vil denne implicitte binding være nok, dog kan det forestilles at analyserne vil være forskellige og til tider komplekse, hvilket kan gøre det nødvendigt at nærmere specificere hvordan den tilgængelige data kan vises.
Her kunne der tilføjes noget meta-data på analysens tabeller/kolonner, der beskriver den på sådan en måde, at det ville kunne bruges til visninger der ikke nødvendigvis implicit kunne knyttes til det.

\subsection{Eksempel}
Her bruges light, som er en simpel sensor og analyse, der fra sensoren indhentes løbende lys-niveau (i lux).
Analysen tager den senest indsamlet data og giver et gennemsnit.
Det antages at alle tabeller har en tids-kolonne, der beskriver enten hvornår data blev indsat i databasen, eller i nogle tilfælde af analyser overføres tiden fra sensor-data.

\begin{lstlisting}
{
  "name": "lightAvg",
  "_version": 1.0,
  "tables": [
    {
      "name": "lightAvg",
      "columns":[
        { "name": "lightAvg", "dataType": "REAL", "_unit": "lux" }
      ]
    }
  ],
  "dependencies": [
    [{ name": "light" }]
  ]
}
\end{lstlisting}

Et eksempel på et visnigs-modul som passer på denne type analyse kunne være en simpel 2D graf, som viser den gennemsnitlige belysning over tid.

\begin{lstlisting}
{
  "name": "2dgraph",
  "_version": 1.0,
  "_type": "view",
  "view": {
    "layout": "2dgraph.xml",
    "data": [
      { "name": "x", "dataTypes": ["INTEGER"], "fromTimestamp": true },
      { "name": "y", "dataTypes": ["INTEGER", "REAL"] }
    ]
  }
}
\end{lstlisting}
 
Denne visning beskriver en 2D graf, som bruger layoutet i \texttt{2dgraph.xml}.
På x-aksen vises tiden (der står INTEGER fordi SQLite ikke har datorepræsentation ud over unix-time).
Her bruges det tidsstempel som forventes på alle tabeller.
Til y-aksen kan bruges alle analyser, som blot har en enkelt kolonne af heltal eller decimal-tal værdi.

\subsection{Administrering af visninger}
Ligesom sensorer og analyser, skal visninger også administreres af brugeren.

For at holde det så simpelt som muligt, kunne man, ligesom ved sensorer/analyser, have to niveauer af administration.
Som udgangspunkt vil alle installeret visninger automatisk knytte sig til de aktiverede analyser.
Derudover vil man i de avancerede indstillinger kunne aktivere/deaktivere visninger for de enkelte analyser.

På denne måde vil der automatisk blive genereret en liste af visning/analyse kombinationer, men med mulighed for at slå nogle af kombinationerne fra.
Det vurderes at muligheden for at slå visninger fra skal være under avancerede indstillinger, da ikke teknisk begavede brugere sagtens kunne forvirres af alle de potentielle visning/analyse kombinationer de ville præsenteret for.
Grunden til at det stadig er med som en mulig customization er, at der er stor fokus på at brugerne skal have mulighed for at styre hvad programmet skal gøre. 
