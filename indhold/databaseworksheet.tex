\section{Database Arbejdsblad}
%Database
Nødvendigt at lagre data fra sensorer så analyse moduler kan få adgang til dem.
I Android er det standard at alt data kun er tilgængeligt til den specifikke app der har lagret det, dog er der mulighed for selv at gøre det tilgængeligt for andre.
%samlet i manager vs i hver app
To muligheder for denne tilgængelighed, man kan samle alt dataen et sted, sandsynligvis i manageren, da denne står for den overordnede kontrol af systemet.
Den anden mulighed er at hvert modul lagre sin egen data, og så gør den tilgængelig for de andre moduler der vil bruge den.
Grundet den dynamik vi gerne vil have når det kommer til at tilføje nye moduler er den bedste fremgangsmåde at have det hele i manageren.
Dette er grundet at moduler så kun skal kontakte manageren for at få data, i stedet for først at skulle identificere hvilket modul man skal kontakte for at få sin data og derefter kontakte den for dataen.
%Database helper
For at lave databasen i manageren er det nødvendigt at have en SQLiteHelper klasse som skal stå for at implementere de metoder der skal til får at lave DB tables, læse fra DB, skrive til DB og hvad der ellers er brug for.
For selve table creation er hvert modul nød til at specificere hvordan deres data skal struktureres, dette gøres i en af deres filer i XML/JSON (eller hvad det lige blev til), som så bliver passet til SQLite helper der laver de requestede database tables
%Contentprovider
Til sidst er det nødvendigt at have en content provider før det rent faktisk er muligt at give adgang til sin data for andre applikationer.
Denne ContentProvider skal så override 5 forskellige metoder:  getType, querry, insert, delete og update.
Insert laver writes til databasen og giver reads.
Disse 2 metoder er de eneste der er implementeret, da det er vurderet at de andre ikke er nødvendige.
Vi kan ikke se hvad getType skal bruges til.
Delete vælges ikke at implementere funktionalitet til, da vi sandsynligvis ikke er interesseret i at slette vores data da den skal bruges til statistikker og lignende.
Update er af næsten lignende grunde heller ikke nødvendig, da vi sandsynligvis ikke vil ændre i vores data efter det er blevet sat ind, i hvert fald ikke sensor dataen.
Analyse moduler kunne måske have brug for at ændre deres data, hvis nu det bare er et modul der via sin analyse bare outputter en enkelt værdi, ville det måske ikke være nødvendig at lagre alle de results den nogensinde har haft, men bare lagre det nyeste.
Delete kunne også overvejes implementeret i en form, dog ikke en form der kan kaldes af andre moduler, mere som en oprydning der måske fjerne for gammel data. Da vi fokuserer på at se mønstre kunne det måske være sådan noget med at sensor data der var over 2 måneder gammel blev slettet, da det måske ikke længere var nødvendigt at se på, da de elementer af det der var vigtigt på nuværende tidspunkt ville være lagret af et analyse modul.