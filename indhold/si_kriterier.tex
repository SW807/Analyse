\section{Kriterier}
Her vil de vigtigste kriterier, som skal opfyldes for at kunne kalde projektet en succes, blive præsenteret.
Disse kriterier dækker over to overordnede områder: i forhold til bruger, og i forhold til platformen i sig selv.

\subsection{Bruger}
\begin{description}[style=nextline]
	\item[Sikkerhed] 
		Eftersom systemet kan komme til at arbejde med person-følsomme data, er det vigtigt at denne data opbevares sikkert, så applikationer der ikke har lov til at se ens data ikke har mulighed for dette.
	\item[Stabilitet]
		For at sikre der ikke opstår for mange huller i den indsamlede data, skal systemet køre stabilt, så der kontinuert kan indsamles data.
	Hvis der opstår for mange huller, kan dette give et ufuldstændigt billede af patientens sindstilstand, som måske kan fejl-fortolkes.
	\item[Præcision]
		Ligesom ved huller i indsamlingen af data over tid, er det lige så vigtigt at data der kommer ind er præcis.
	Dette grundes at analyse på upræcis data vil give et upræcist billede af patientens sindstilstand, hvilket kan føre til fejlagtige vurderinger af denne.	
\end{description}

\subsection{Platform}
\begin{description}[style=nextline]
	\item[Fleksibilitet]
	Det skal være nemt at modificere funktionalitet til platformen, da platformen skal kunne tilpasses til forskellige individer.
	\item[Mulighed for udvidelse]
	Det skal være muligt at udvide platformens funktionalitet, uden at skulle ændre på selve platformen.
	Det skal gøres på sådan en måde at personer der ikke er en del af systemet, kan lave og tilføje egne dele til platformen.
	Dette giver naturligvis yderligere overvejelser ift. sikkerhed.
\end{description}

\subsection{Evaluering af kriterier}
For at evaluere de forskellige kriterier skal


\mikael{Hvordan skal vi evaluere på de kriterier? Hvad kunne være godt at gøre? Hvad har vi egentlig tænkt os at gøre?}
\bruno{Igen +1 til Mikael - damn it..}
\lasse{Mikael skriver dette, det sagde han at han ville d. 2015-05-19 kl 12:45}