\section{Kriterier}
Her vil de vigtigste kriterier, som skal opfyldes for at kunne kalde projektet en succes, blive præsenteret.
Disse kriterier dækker over to overordnede områder: i forhold til bruger, og i forhold til platformen i sig selv.

\subsection{Bruger}

\subsubsection{Sikkerhed}
Da systemet, afhængigt af valgte moduler, kan indeholde person-følsomme data er det vigtigt at denne data er opbevaret sikkert og ikke kan tilgås af udefrakommende processer der ønsker at bruge den data til andet end helbredsestimering.

\subsubsection{Stabilitet}
For at sikre der ikke opstår for mange huller i den indsamlede data, skal systemet køre stabilt, så der kontinuert kan indsamles data.
Hvis der opstår for mange huller, kan dette give et ufuldstændigt billede af patientens sindstilstand, som måske kan fejl-fortolkes.

\subsubsection{Præcision}
Ligesom ved huller i indsamlingen af data over tid, er det lige så vigtigt at den data der kommer ind er præcis.
Dette er især også gældende for den måde som analyse-moduler fortolker den rå data, da fejl i analyser vil kunne give et forkert billede af patientens sindstilstand.

\subsection{Platform}

\subsubsection{Fleksibilitet}
Det skal være nemt at modificere selve platformen (ikke kun tilføjelse af nye moduler), da det stadig er et usikkert område der opereres indenfor.
Disse usikkerheder gælder både problemområde og den platform der implementeres på (Android).

\subsubsection{Mulighed for udvidelse}
Det skal være muligt at udvide platformen, uden at modificere platformens kodebase.
Dette skal gøres med moduler.
Det skal gøres på sådan en måde at personer der ikke er en del af systemet, kan lave og tilføje egne moduler.
Dette giver naturligvis yderligere overvejelser ift. sikkerhed.

\subsection{Evaluering af kriterier}
\mikael{Hvordan skal vi evaluere på de kriterier? Hvad kunne være godt at gøre? Hvad har vi egentlig tænkt os at gøre?}
