\section{Eksisterende Mønstre}
Ved implementation af platformen vil det være fordelagtigt at anvende et eksisterende mønster til at beskrive systemets arkitektur.
Ved at anvende et eksisterende mønster vil mange strukturelle udfordringer kunne undgås.
Herunder beskrives eksisterende arkitekturmønstre og der foretages en vurdering af hvorvidt mønstrene kan anvendes, givet behovene beskrevet i \cref{arkitekturkrav}.
\mikkel{Hvor har vi netop de her mønstre fra og hvordan ved vi at der ikke findes andre mønstre der passer bedre?}

\subsection{Client-Server}
En af standard arkitekturerne i mobil udvikling er \textbf{Client-Server}, hvilket kunne være hvor web sider kan bruges til at f.eks. visualisere data, men dette har et problem idet at sensor data indsamling vil nødvendigvis påkræve implementering adskilt da det er data som er interessant.
Denne arkitektur har dette problem at det ikke er nemt for udefrakommende udviklere at tilføje funktionalitet, det kan desværre også være svært at implementere dele som indsamler, analysere og viser data idet at alt data ville ligge lokalt og skal nødvendigvis sendes til serveren før den kan bruge denne data til noget. 
Det har dog fordelen at det kan køre på Android gennem en almindelig webbrowser eller en applikation som bruger WebViews.
Baseret på disse ville denne arkitektur nok ikke fungere.
\mikkel{Vi mangler en beskrivelse af selve arkitekturen.}
\stefan{kilde?}
\subsection{Standalone}
En anden arkitektur ville være en \textbf{standalone} applikation som indeholder alt sensor indsamling, analyse og visualisering i samme applikation.
Dette har fordelen at det er nemmere at programmere end 15 forskellige applikationer, men kommer med ulempen at applikationen kunne ende med at blive meget stor.
Det er selvfølgelig muligt for udefrakommende udviklere at implementere ny funktionalitet til denne hvis de har fri adgang til kildekoden, men dette vil nok danne med problemer med Google Play Store idet at hvis 30 forskellige udviklere tilføjer ny funktionalitet til applikationen ville dette påkræve at hvis denne skal bruges af flere ville der være flere versioner af den samme applikation uploadet af forskellige udviklere og dette gør det på samme tid svære at udvikle videre på andres arbejde da alting er samlet i en applikation og ikke over flere.
På grund af dette ville denne arkitektur ikke gøre det let at implementere og bruge ny funktionalitet.
Denne arkitektur gør det muligt at implementere forskellige dele til applikationen som indsamler, analysere og viser data.
Da det ikke er let at tilføje ny funktionalitet afvises denne arkitektur. 
\mikkel{Det her afsnit skal skrives om - det er noget rod. Har vi en kilde på det eller er det bare fri fantasi? :P}

Baseret på disse antages det at arkitekturen der er blevet valgt er den mest åbenbare og at hvis der er andre arkitekturer kender vi ikke til dem.
\mikkel{Det giver ingen mening. Arkitekturen er den åbenbare?
Og 'hvis der er andre arkitekturer kender vi ikke til dem'?? Det svarer til 'Hvis det vi har lavet er skidt er det fordi vi er uvidende.'}
\stefan{kilde?}