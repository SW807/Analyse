\section{Arkitektur Behov}\label{arkitekturkrav}
Ved analyse af problemområdet udledes der i følgende afsnit en beskrivelse af behov til arkitekturen for den udviklede platform.
Behovene beskrives herunder i form af følgende nøgleord: \textit{modulær}, \textit{fleksibel}, \textit{kombinerbar} og \textit{kommunikation}.

\subsection{Modulær}\label{arkitekturkrav::modulaer}
I \cref{sec:affektivelidelser} beskrives det at symptomerne på affektive lidelser varierer meget.
Derfor skal arkitekturen understøtte muligheden for at detektere alle de forskellige symptomer.
Det er dog ikke nødvendigt altid at detektere alle symptomer, da de konkrete symptomer varierer fra person til person.
Denne varians gør at en modulær opbygning er nødvendig, da dette vil gøre det muligt nemt at modificere funktionaliteten så den kun går efter at identificere den konkrete brugers symptomer.

Ydermere vil en modulær arkitektur også gøre det nemt at udskifte enkelte dele af funktionalitet, hvis der nye og bedre teorier inden for de forskellige områder.

\subsection{Fleksibel}\label{arkitekturkrav::fleksibel}
Arkitekturen skal være fleksibel så den kan tilpasses til patienten.
Dette kunne fx være en mulighed for patienten at styre hvilke moduler der aktive.

Arkitekturen skal også være fleksibel i forbindelse med fx valg af accelerometer.
Accelerometret i en smartphone er muligvis mere præcist end det i et smart watch, men da man har sit smart watch på håndleddet vil denne data i en given kontekst være mere relevant.
Det skal derfor være muligt at vælge denne.

\subsection{Kombinerbar}\label{arkitekturkrav::kombinerbar}
Arkitekturens moduler skal være kombinerbar således at det er muligt at kombinere flere moduler som fx siger noget om patientens sociale aktivitet eller søvn.

Data fra en sensor skal altså kunne bearbejdes af andre moduler der kan udnytte den information i forbindelse med en analyse.

Kombinerbare moduler sørger for at indhentning af data og analyse af data let kan separeres.
Derved er det ikke er nødvendigt for de enkelte moduler at udføre indhentningen af data.

\subsection{Kommunikation}\label{arkitekturkrav::kommunikation}
Platformen skal desuden understøtte kommunikation mellem moduler så data fra et modul nemt og sikkert kan deles med andre moduler.
Det skal ikke være muligt at skrive til andre modulers databasetabeller.
Ved på denne måde at kontrollere adgangsrettighederne til data, forenkles processen med udvikling af nye moduler.
Det sker fordi man som udvikler af et modul er garanteret, at data kun modificeres af modulet selv.
Man skal altså således ikke bruge ressourcer på at verificere data i selve modulet.
Dette er dog ikke blevet implementeret, men bliver diskuteret yderligere i \cref{databaserettigheder}.


\subsection{Opsummering}
For at opsummere ønsker vi en arkitektur der er opbygget af moduler.
Det skal være muligt at udvikle moduler uafhængigt af hinanden.
Arkitekturen skal være fleksibel så det er muligt at til- og fra-vælge moduler og sensorer for den individuelle patient.
Desuden skal modulerne være kombinerbar i en sådan forstand så det er muligt at fx bruge moduler andre har udviklet sammen med ens egne moduler.
Til sidst skal arkitekturen understøtte nem og sikker kommunikation mellem modulerne.


