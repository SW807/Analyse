\section{Møde med Psykolog Janne Rasmussen}\label{sec:moede-med-psykolog}
Vi holdte d. 18. februar 2015 et møde med Janne Rasmussen, på Psykiatrisk Sygehus, også kaldet Psykiatrien, i Aalborg.

Janne Rasmussen er psykolog og har arbejdet på Psykiatrien i 10 år, hvor hun lige nu sidder på ambulatoriet for patienter med psykose og  at diagnosticere patienterne. 
Hun er primært involveret med affektive patienter, som unipolar depression eller bipolar depression, hun har også været involveret med patienter som lider af angst, OCD og andre affektive lidelser.

Formålet med mødet var som følger:

\begin{enumerate}
\item At opnå viden om lidelserne.
\item At diskutere vores foreløbige idéer om at detektere symptomer via sensorer
\end{enumerate}

Da hun primært er involveret med affektive lidelser bliver det mest lagt vægt på denne og vi vil kort introducere dette emne, se \secref{sec:affektivelidelser} for flere detaljer om denne slags lidelser.
Derudover, er de patienter med affektive lidelser mere tilbøjelige til at benytte en mobil applikation som behandlingsmetode.
Derfor virker det til at patienter med affektive lidelser er en god målgruppe for denne slags applikation.

\subsection{Opsummering af interview}
Hvad der følger er de vigtigste ting vi kom frem til i diskussionen med Janne til mødet.
Det fulde referat af mødet med Janne Rasmussen findes i \appref{app:moede-med-janne-referat}.

\paragraph{Søvn}
Det er vigtigt at identificere søvn da det siger meget om deres situation, og ud fra søvn baseret på ændringer kunne man identificere om de er på vej ind i en manisk eller depressiv periode og om de er på vej ud af den periode. 
Specifikt hvis man kan se om patienterne begynder at blive op længere eller sover længere kunne man måske hjælpe dem tidligere.

\paragraph{Diagnosticering}

\subparagraph{Hamilton skalaen}
Hamilton skalaen er et spørgeskema der udføres af en behandler der vurderer patienten baseret på en række spørgsmål, som f.eks. kunne spørge ind til aktivitetsniveau, døgnrytme, energiniveau og andet.
Når denne er blevet udfyldt får man en score som ligger en i én af 4 kategorier: Ikke, let, moderat eller svært deprimeret. 
Denne kan så bruges som en måling af patienternes sindstilstand og hvordan behandlingsforløbet går.
Et online Hamilton skala findes på \citet{hamilton}


\paragraph{Værktøjer}
Når patienterne er indlagt er det vigtigt at de har en meget struktureret hverdag da de ikke kan overskue når der sker noget uventet, derudover er det også vigtigt fordi ellers bliver de depressive bare liggende i deres seng hele dagen.
Det er desuden vigtigt for behandleren at identificere hvad der gør at patienten er på vej ind i en mani eller depressions periode.
Dette kunne f.eks. være ting som dårlig økonomi, fødselsdage, eller andre stressede situationer.
Endvidere er det også vigtigt for behandleren at identificere hvad det kan få patienten ud af sådan en periode, hvilket er de lystbetonede aktiviteter, motion eller et regulært søvnmønster.

Når patienten skal udskrives er det vigtigt for dem at de har nogle værktøjer så de kan komme tilbage til en regulær hverdag, da risikoen for tilbagefald eller selvmord er meget stor på dette tidspunkt. 
Det følgende beskriver de værktøjer som Janne beskrev at hun bruger i forbindelse med sin behandling af maniodepressive patienter.

\subparagraph{Huskekort}
I samarbejde med patienten udarbejder behandleren en række individualiserede huskekort som patienten skal have på sig, og bruge i ``pressede'' situationer. 
Hvis en mani patient har en tendens til at lave forhastede beslutninger kunne et huskekort være: ``ring til din kollega og hør ham inden du træffer en beslutning''.

\subparagraph{Lystbetonede aktiviteter}


\subparagraph{Stemningsregistrering}
hvor patienten hen over en måned sætter kryds i et skema over hvordan de har det og få dem til at udføre lystbetonede aktiviteter og ikke kun pligt opgaver som de gør for andre. 




