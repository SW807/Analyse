\section{Møde med psykolog Janne Vedel Rasmussen}\label{sec:moede-med-psykolog}
Der blev den 18. februar 2015 holdt et møde med Janne Vedel Rasmussen på Psykiatrisk Sygehus, også kaldet Psykiatrien, i Aalborg.

Janne er psykolog og har arbejdet på Psykiatrien i 10 år, hvor hun er på ambulatoriet for patienter med psykose og arbejder med at diagnosticere disse. 
Hun arbejder primært med affektive patienter, som for eksempel lider af unipolar og bipolar lidelse.
Hun har også tidligere arbejdet med patienter som lider af andre affektive lidelser, herunder angst og tvangslidelse.

\paragraph{}
Formålet med mødet var som følger:

\begin{enumerate}
\item At opnå viden om lidelserne.
\item At diskutere vores foreløbige idéer om at detektere symptomer via sensorer
\end{enumerate}

Da hun primært arbejder med affektive lidelser (se evt. \cref{sec:affektivelidelser}) er dette hovedemnet af mødet.
Ifølge Janne vil patienter med affektive lidelser være mere tilbøjelige til at kunne bruge en mobil applikation, som vil overvåge dem, i forhold til patienter med en psykose diagnose, da disse overhovedet ikke vil overvåges.
Derfor virker det til at patienter med affektive lidelser er en god målgruppe for denne slags applikation, som benævnt i \cref{afgraensning_af_problemområde}.

\subsection{Opsummering af mødet}
Her vil de mest relevante ting, der blev fundet frem til i mødet med Janne blive præsenteret.
Det fulde referat af mødet med Janne findes i \cref{app:moede-med-janne-referat}.

\subsubsection{Søvn}
Det er vigtigt at identificere søvn, da det siger meget om personer med affektive lidelsers sindstilstand.
Ud fra ændringer i søvn vil man kunne identificere om de er på vej ind i en mani- eller depressiv-periode, eller om de er på vej ud af perioden. 
Hvis man kan se om patienterne begynder at blive oppe længere eller sover mere, kunne man måske hjælpe dem tidligere.

\subsubsection{Evaluering af stemningsleje}
Til at evaluere patienternes stadie af depression bliver der brugt en metode kaldet Hamilton skalaen, hvilket er et spørgeskema der udfyldes af en behandler ud fra patientens besvarelser. 
Behandleren vurderer patienten baseret på en række spørgsmål, eksempelvis spørges der ind til aktivitetsniveau, døgnrytme og energiniveau.
Når skemaet er blevet udfyldt får man en score, som ligger i én af 4 kategorier: Ikke-, let-, moderat- eller svært-deprimeret. 
Scoren bruges som et mål for patientens sindstilstand og for hvordan behandlingsforløbet forløber.
En online Hamilton skala kan findes på \citet{hamilton}.

\subsubsection{Værktøjer}
Når patienterne er indlagt er det vigtigt at de har en meget struktureret hverdag, da de ikke kan overskue når der sker noget uventet.
Hvis dette ikke lykkes kan det være at patienten bare bliver liggende i sin seng hele dagen.

Det er desuden vigtigt for behandleren at identificere hvad der gør at patienten er på vej ind i en mani eller depressions-periode.
Dette kunne f.eks. være ting som dårlig økonomi, fødselsdage, eller andre stressede situationer.
Endvidere er det også vigtigt for behandleren at identificere hvad der kan få patienten ud af sådan en periode, hvilket kunne være lystbetonede aktiviteter (se nedenunder her), motion eller et regulært søvnmønster.

Når patienten skal udskrives er det vigtigt for dem at de har nogle værktøjer så de kan komme tilbage til en regulær hverdag, da risikoen for tilbagefald eller selvmord er meget stor på dette tidspunkt. 
Det følgende beskriver de værktøjer som Janne bruger i forbindelse med sin behandling af bipolare patienter.
\begin{description}[style=nextline]
	\item[Huskekort] I samarbejde med patienten udarbejder behandleren en række individualiserede huskekort som patienten skal have på sig, og bruge i ``pressede'' situationer. 
	Hvis en mani patient har en tendens til at lave forhastede beslutninger kunne et huskekort være: ``ring til din kollega og hør ham ad inden du træffer en beslutning''.
	
	\item[Lystbetonede aktiviteter] Et andet værktøj, der kan bruges ved påbegyndende forværring er en liste af lystbetonede aktiviteter.
	Derved kan en af disse aktiviteter udføres, for at bryde det dårlige mønster.
	I samarbejde med behandler vælges der et antal aktiviteter ud, med udgangspunkt i en samlet liste over lystbetonede aktiviteter.
	
	\item[Stemningsregistrering] For at detektere ændringer i stemning, vil patienten hen over en måned sætter kryds i et skema, og derved indikere hvordan de har det.
	Folderen der bruges til stemningsregistrering kan ses i \cref{app:stemningsregistrering}.
	Hvis dette er gjort løbende vil det være nemmere for patient at snakke om sin stemning, og ændringerne i denne, under samtale.
	Dette kræver dog at patienten i forvejen er klar over sine individuelle symptomer og er i stand til at opdage og notere dem når de er aktive.
\end{description}

\subsubsection{Patienter med affektive lidelser}
Ved behandling af patienter der falder i det affektive spektrum er en af fordelene at nogle af dem er meget autoritetstro og pligtopfyldende.
Dette gør at de er mere tilbøjelige til at acceptere nye behandlingsmetoder.
Dette står i kontrast med f.eks. psykotiske paranoide, som ikke vil blive overvåget eller fortælles hvad de skal gøre.

\subsubsection{Præsentation af idéer}\label{janne_ideer}
Denne del af mødet byggede på en brochure med idéer til hvordan forskellige sensorer i smartphones eller ekstern hardware kan anvendes i forbindelse med sindslidelser. 
Denne brochure kan ses i \cref{app:brochure}.
Her er udtaget de forslag som Janne synes er brugbare.

\begin{description}[style=nextline]
\item[Billedanalyse]
En idé her var at tage et billede af patienten og automatisk analysere ansigtsmimik (humør).
Dette virker til at være en interessant emne som kan bruges i produktet, dog risikerer vi at nogle patienter er utilpasse med dette og derfor er det vigtigt at gøre det valgfrit og bare tilbyde det som et værktøj til dem som kan håndtere det.


\item[Lokation]
Ved bipolare patienter, i en mani-periode, vil de være mere omkring-farende.
Denne ændring i adfærd ville kunne detekteres via lokation.

\item[Lyd] 
For bipolare patienter, i en mani-periode, observeres det tit at de taler hurtigt, at de laver flere jokes, at de er småsyngende og opsnapper ord. Det er ikke rigtig \textbf{stemme}leje, men \textbf{stemnings}leje der er vigtigt her.

\item[Opkald]
Dette ville kunne bruges til bipolare lidelser, da når de er i mani-perioder vil de være meget sociale og derfor snakker mere. 

\item[Puls]
Eftersom mange af patienterne har somatiske, altså fysiske, problemer, som stress eller højt blodtryk, er det derfor vigtigt at kunne skelne disse ting fra den fysiske aktivitet.
Hvis dette kan lade sig gøre, kan det godt bruges.

\item[Søvn]
Specielt det med at analysere patienters søvnmønster var en god idé, men man skal være opmærksom på at en patient kan ligge vågen selvom han er gået i seng. 
En patient kan også falde i søvn igen efter en mobil alarm er blevet slukket.
\end{description}

\subsubsection{Segmentering}
En yderligere overvejelse var at segmentere patienterne på hvordan de bruger deres smartphone.
Dette er nødvendigt da brugerne er meget forskellige i hvordan de bruger deres smartphone, og det skal tages i forbehold ved de forskellige detektions-former (fx. er det ikke alle der har sin smartphone på sig derhjemme).
\mikael{Forstår slet ikke det her, kan det bare slettes?}

\subsubsection{Påmindelser}
Det kunne være en god idé hvis man kunne give påmindelser til patienterne, specielt dem med kognitive problemer (f.eks. hukommelses problemer), da dette ville kunne hjælpe dem i deres hverdag.
Disse påmindelser skal være knyttet til de værktøjer der bruges i forvejen, som fx. huskekort eller lystbetonede aktiviteter.

\paragraph{}
Da Janne er meget involveret i behandling af patienter ønskes der endvidere en mere teknisk viden, hvilket søges fra psykiatri professor Jørgen Aagaard, mødet med ham følger herefter.