\section{Møde med Psykolog Janne Rasmussen}\label{sec:moede-med-psykolog}
Vi holdte d. 18. februar 2015 et møde med Janne Rasmussen, på Psykiatrisk Sygehus, også kaldet Psykiatrien, i Aalborg.

Janne Rasmussen er psykolog og har arbejdet på Psykiatrien i 10 år, hvor hun lige nu sidder på ambulatoriet for patienter med psykose og  at diagnosticere patienterne. 
Hun er primært involveret med affektive patienter, som unipolar depression eller bipolar depression, hun har også været involveret med patienter som lider af angst, OCD og andre affektive lidelser.

Formålet med mødet var som følger:

\begin{enumerate}
\item At opnå viden om lidelserne.
\item At diskutere vores foreløbige idéer om at detektere symptomer via sensorer
\end{enumerate}

Da hun primært er involveret med affektive lidelser bliver det mest lagt vægt på denne og vi vil kort introducere dette emne, se \cref{sec:affektivelidelser} for flere detaljer om denne slags lidelser.
Derudover, er de patienter med affektive lidelser mere tilbøjelige til at benytte en mobil applikation som behandlingsmetode.
Derfor virker det til at patienter med affektive lidelser er en god målgruppe for denne slags applikation.

\subsection{Opsummering af interview}
Hvad der følger er de vigtigste ting vi kom frem til i diskussionen med Janne til mødet.
Det fulde referat af mødet med Janne Rasmussen findes i \cref{app:moede-med-janne-referat}.

\paragraph{Søvn}
Det er vigtigt at identificere søvn da det siger meget om deres situation, og ud fra søvn baseret på ændringer kunne man identificere om de er på vej ind i en manisk eller depressiv periode og om de er på vej ud af den periode. 
Specifikt hvis man kan se om patienterne begynder at blive op længere eller sover længere kunne man måske hjælpe dem tidligere.

\paragraph{Diagnosticering}

\subparagraph{Hamilton skalaen}
Hamilton skalaen er et spørgeskema der udføres af en behandler der vurderer patienten baseret på en række spørgsmål, som f.eks. kunne spørge ind til aktivitetsniveau, døgnrytme, energiniveau og andet.
Når denne er blevet udfyldt får man en score som ligger en i én af 4 kategorier: Ikke, let, moderat eller svært deprimeret. 
Denne kan så bruges som en måling af patienternes sindstilstand og hvordan behandlingsforløbet går.
Et online Hamilton skala findes på \citet{hamilton}


\paragraph{Værktøjer}
Når patienterne er indlagt er det vigtigt at de har en meget struktureret hverdag da de ikke kan overskue når der sker noget uventet, derudover er det også vigtigt fordi ellers bliver de depressive bare liggende i deres seng hele dagen.
Det er desuden vigtigt for behandleren at identificere hvad der gør at patienten er på vej ind i en mani eller depressions periode.
Dette kunne f.eks. være ting som dårlig økonomi, fødselsdage, eller andre stressede situationer.
Endvidere er det også vigtigt for behandleren at identificere hvad det kan få patienten ud af sådan en periode, hvilket er de lystbetonede aktiviteter, motion eller et regulært søvnmønster.

Når patienten skal udskrives er det vigtigt for dem at de har nogle værktøjer så de kan komme tilbage til en regulær hverdag, da risikoen for tilbagefald eller selvmord er meget stor på dette tidspunkt. 
Det følgende beskriver de værktøjer som Janne beskrev at hun bruger i forbindelse med sin behandling af maniodepressive patienter.

\subparagraph{Huskekort}
I samarbejde med patienten udarbejder behandleren en række individualiserede huskekort som patienten skal have på sig, og bruge i ``pressede'' situationer. 
Hvis en mani patient har en tendens til at lave forhastede beslutninger kunne et huskekort være: ``ring til din kollega og hør ham inden du træffer en beslutning''.

\subparagraph{Lystbetonede aktiviteter}
Et andet værktøj der kan bruges ved påbegyndende nedtur er en liste af lystbetonede aktiviteter.
Derved kan en af disse aktiviteter udføres, for at bryde det dårlige mønster.
I samarbejde med behandler vælges der et antal aktiviteter ud, med udgangspunkt i en samlet liste over lystbetonede aktiviteter.

\subparagraph{Stemningsregistrering}
For at detektere ændringer i stemning, vil patienten hen over en måned sætter kryds i et skema, og derved indikere hvordan de har det.
Folderen der bruges til stemningsregistrering kan ses i \cref{app:stemningsregistrering}.
Hvis dette er gjort løbende, vil det være nemmere for patient at snakke om sin stemning og ændringerne i denne under samtale.
Dette kræver dog at patienten i forvejen er klar over sine individuelle symptomer og er i stand til at detektere og notere dem når de er aktive.

\paragraph{Patienter med affektive lidelser}
Ved behandling af patienter der falder i det affektive spektrum er en af fordelene at nogle af dem er meget autoritetstro og pligtopfyldende.
Dette gør at de er mere tilbøjelige til at acceptere nye behandlingsmetoder, som f.eks. brug af en mobil-applikation som kan identificere mønstre i deres opførsel som kan føre til tilbagefald.
Dette står i kontrast med f.eks. psykotiske paranoide, som ikke vil blive overvåget eller fortælles hvad de skal gøre.

\paragraph{Præsentation af Idéer}\label{janne_ideer}
Denne del af mødet byggede på en brochure med idéer til hvordan forskellige sensorer i telefon eller ekstern hardware kan bruges til at hjælpe med at behandle sindslidelser. 
Denne brochure kan ses i \cref{app:brochure}.

Her er udtaget de forslag som Janne syntes var brugbare.

\subparagraph{Billedanalyse}
En idé her var at tage et billed af patienten og automatisk analysere ansigtsmimik (humør).
Dette virker til at være en interessant emne som kan bruges i produktet, dog risikerer vi at nogle patienter er utilpasse med dette og derfor er det vigtigt at gøre det valgfrit og bare tilbyde det som et værktøj til dem som kan håndtere det.

\subparagraph{Lokation}
Ved bipolare patienter, i en manisk periode, vil de være mere omkring-farende.
Denne ændring i adfærd ville kunne detekteres via lokation.

\subparagraph{Lyd}
For bipolare patienter, i en manisk periode, observeres det tit at de taler hurtigt, at de laver flere jokes, at de er småsyngende og opsnapper ord. Det er ikke rigtig stemmeleje, men mere stemningsleje der betyder noget.

\subparagraph{Opkald}
Dette ville kunne bruges til bipolare lidelser, da når de er i mani perioder vil de være meget sociale og derfor snakke meget. 

\subparagraph{Puls}
Eftersom mange af patienterne har somatiske, altså fysiske, problemer som stress eller højt blodtryk er det derfor vigtigt at kunne skelne disse ting fra den fysiske aktivitet.
Hvis dette kan lade sig gøre, kan det godt bruges. 

Forbundet med brug af hardware som JawBone ville der være en positiv side effekt idet at patientens kendskab til at man bliver overvåget af en JawBone el. andet vil være en motiverende faktor for mange patienter.

\subparagraph{Søvn}
Specielt det med at analysere patienters søvnmønster var en god idé, men man skal være opmærksom på at en patient kan ligge vågen selvom telefonen har slukket skærm, og at en patient kan falde i søvn igen efter en mobil alarm er blevet slukket.
Dette kan også være meget afvigende hvis patienten ikke kigger på klokken.

\paragraph{Segmentering}
En yderligere overvejelse var at segmentere patienterne på hvordan de bruger deres telefon.
Dette er nødvendigt da brugerne er meget forskellige i hvordan de bruger deres telefon, og det skal tages i forbehold ved de forskellige detektions-former (fx. er det ikke alle der har sin mobiltelefon på sig derhjemme).

\paragraph{Påmindelser}
Det kunne være en god idé hvis man kunne give påmindelser til patienterne, specielt dem med kognitive problemer (f.eks. hukommelses problemer), da dette ville kunne hjælpe dem i deres hverdag.
Disse påmindelser skal være knyttet til de værktøjer der bruges i forvejen, som fx. huskekort eller lystbetonede aktiviteter.
