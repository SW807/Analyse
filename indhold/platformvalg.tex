\section{Valg af Platform}\label{sec:valg_af_android}
Platformen for projektet skal vælges, her ses der på hvilke fordele og ulemper de forskellige platforme har og endeligt laves der en beslutning.

Der er forskellige platforme der kan vælges, specifikt Android, iPhone og Windows Phone.
Der kunne også vælges en hybrid platform fx. Apache Cordova \citep{misc:apachecordova}

Fordelene ved Android er som følgende:
\begin{itemize}
\item Det er en meget åben platform som giver adgang til meget sensor data
\item Android har flere brugere end iPhone og Windows Phone.
\item Der er ingen begrænsninger på hvad der kan udvikles
\item Der er mange udviklings ressourcer tilgængelige såsom guides, tutorials.
\item Android applikationer udvikles som regel i Java, hvilket er et udbredt programmeringssprog.
\end{itemize}

Android har dog også ulemper, såsom mange forskellige typer smartphones med forskellig hardware og forskellige versioner af operativ systemet.

Fordelene ved iPhone er primært at der er nemmere at udvikle til da det ikke er særlig fragmenteret i forhold til Android. 
Brugerfladen er meget standardiseret hvilket gør det nemmere at udvikle den del af applikationen. 
iPhone har ulemper, da det er et meget mere lukket system hvilket gør at data kan være umulige at få fat i. iPhone udvikling foregår kun på OS X, og kræver en licens. iPhone udvikling foregår i et sprog som ikke bruges bredt, Objective-C.

Fordelene ved Windows Phone er et meget modent/godt udviklingsmiljø og programmeringssprog, og at brugerflade design er meget nemt. 
Ulemper ved Windows Phone er så at der ikke er særlig mange udviklings ressourcer tilgængelig da Windows Phone markedet ikke er særlig bredt. Windows Phone er også en lukket platform.

Vores projekt er meget afhængig af åben adgang til datakilder, hvilket Android giver bedre adgang til. Vores situation er også at udvikling på Android virker nemmere da det ikke påkræver et specielt operativ system da ingen på udviklingsholdet har OS X computere, og at udvikling i Java er kendt blandt mange på udviklingsholdet.

Grunden til en hybridplatform fravælges, er da det vurderes at den tilføjer et unødvendigt lag af kompleksitet til udviklingen og vil derfor være noget der bliver brugt ressourcer på som ikke giver ret meget værdi til projektet.
Det vurderes også at det umiddelbart vil være en simpel opgave at konvertere det udviklede system fra den valgte platform til de andre. \lars{Er dette stykke fornuftigt?}