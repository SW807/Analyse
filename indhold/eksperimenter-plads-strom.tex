På mobile enheder er der begrænset lagerplads grundet enhedens størrelse, som eksempel har vores Samsung Galaxy S4 enheder kun 16 GB hvorimod mange computere nu om dage har mindst 1 TB (ca 64 gange mere).
Dette er nødvendigt at forholde sig til, da det sætter en begrænsning på hvad man kan tillade sig at gøre.
I vores tilfælde er det vi skal overveje hvor stor andel sensor data der kan lagres.
For at undersøge hvordan de forskellige sensorer bruger pladsen på telefonen opstilles to eksperimenter.
\ivan{Evt. omtale af backup af data til server}

Applikationen blev sat til at køre med følgende moduler tændt:
\begin{itemize}
	\item sound
	\item screen
	\item proximity
	\item location
	\item gyroscope
	\item accelerometer
\end{itemize}

Dette udvalg er de moduler der på eksperiment tidspunktet var kørbare.
Det er vigtigt at bemærke at de ikke er i deres endelige form og at listen er ikke er komplet.
Der vil i fremtiden komme flere sensorer og der vil komme analysemoduler der analyserer data og muligvis gemmer denne analyse.
Disse eksperimenter vil derfor kun kunne give et indblik i hvordan sensorer opfører sig og give et estimat for pladsforbruget.

Moduler som screen, proximity og location producerer kun data når der sker ændringer. 
Derfor blev det første eksperiment udført på en måde der forsøger at efterligne brugssituation. 
Telefonen blev bevæget og skærmen blev tændt og slukket flere gange under testperioden.
Det første eksperiment blev udført over 30 minutter.

Det andet eksperiment blev udført over en weekend, hvor telefonen lå stille.
Dette eksperiment burde vise et minimumsforbrug for sensorerne.

\paragraph{Forespørgselshastighed}
Pladsen som modulerne bruger vil naturligvis afhænge af hvor ofte modulerne forespørger sensorerne for data.

I android er der forskellige metoder til at angive forespørgselshastigheden på en sensor.
Der findes fire konstanter der kan benyttes til at angive hastigheder der passer i forskellige kontekster \cite{sensormonitor}.

\begin{itemize}
	\item SENSOR\_DELAY\_FASTEST forespørger hele tiden.
	\item SENSOR\_DELAY\_GAME forespørger hver 20. ms, og er beregnet til spil.
	\item SENSOR\_DELAY\_UI forespørger hver 60. ms og er tilstrækkeligt til brug i userinterfaces.
	\item SENSOR\_DELAY\_NORMAL forespørger hver 200. ms er tilstrækkeligt til at opfange ændringer i skærm orientering.
\end{itemize}
Det er muligt at angive opdateringen i ticks hvis man vil have en langsommere opdatering.
Det skal bemærkes at disse angivelser kun benyttes som et hint til telefonen, det kan ikke garanteres at sensoren bliver forespurgt med det angivne interval.
I dette eksperiment er SENSOR\_DELAY\_NORMAL benyttet.


\subsection{Resultater}
Til analyse af pladsforbruget blev programmet ``SQLite-analyzer'' benyttet \cite{sqliteanalyzer}.
SQLite-analyzer viser statistik for en SQLite database inklusive data for de enkelte tabeller.

\paragraph{30 minutters eksperiment}
Efter 30 minutter fyldte databasen 1156459 bytes ($\sim 1.15$ MB).
Under antagelsen at sensorerne vil fortsætte med at generere data i den samme hastighed vil det efter 24 timer fylde omkring 55 MB. 

De enkelte sensorers tabeller var fordelt på følgende måde

\begin{tabular}{|c|c|c|c|}
	\hline Modul 			& Plads forbrugt i bytes	& Antal datapunkter  & \% af databasen \\
	\hline Accelerometer 	& 609266 / $\sim 609$ KB	& 12434 			 & 51.1 \\ 
	\hline Gyroskop 		& 495018 / $\sim 495$ KB	& 10146 			 & 41.6\\ 
	\hline Lyd 				& 42648  / $\sim 42$ KB		& 1706 			 	 & 4.3 \\ 
	\hline Lokation 		& 4751 	 / $\sim 4$ KB		& 91 				 & 0.92 \\ 
	\hline Proximity 		& 3173 	 / $\sim 3$ KB		& 135 				 & 0.31 \\ 
	\hline Skærm 			& 1242 	 / $\sim 1$	KB		& 54				 & 0.31 \\ 
	\hline 
	\ivan{SKAL disse data lagres diskret form? Kan de aggregeres uden at miste vigtig viden}
\end{tabular} 

Som det ses af tabellen er de store pladssyndere accelerometret og gyroskopet.
Dette skyldes at de forespørger konstant og gemmer al data.

Lydmodulet var på eksperimenttidspunktet meget simpelt, og gemte kun amplituden af den optagne lyd i databasen. 
Af denne grund fylder lydmodulets data i denne test ikke mere end 4 \% af databasen.
Afhængig af hvad man vil analysere vil det være nødvendigt at gemme væsentligt mere end dette.

Lokation er fra android konstrueret så den kun forespørger når der er en tilstrækkelig ændring i lokationen.

Proximity sensoren ændrer sig kun når man sætter noget ind foran den, og der er derfor få datapunkter i dennes tabel.
Det samme gælder for skærmsensoren der kun registrerer en ændring når skærmen enten tændes eller slukkes.	

\paragraph{Weekend eksperiment}

Eksperimentet blev udført fra kl. 13:35 fredag eftermiddag til kl. 08:23 mandag morgen, hvilket giver en total varighed på 66 timer og 48 minutter.

I denne periode blev databasen fyldt med 136523139 bytes (136 MB) hvilket svarer til 1018829 bytes (1 MB) pr. halve time eller 49 MB i døgnet.

\begin{tabular}{|c|c|c|c|}
	\hline Modul 		 & Plads forbrugt i bytes	  	& Antal datapunkter & \% af databasen \\
	\hline Accelerometer & 65493057 / 65 MB   			& 1336593    		& 47.6 \\ 
	\hline Gyroskop 	 & 64930232 / 64 MB  			& 1335216   		& 47.6\\ 
	\hline Lyd 		  	 & 5476680  / 5  MB  			& 227991     		& 4.4 \\ 
	\hline Lokation 	 & 622494 	/ 622 KB			& 11971      		& 0.45 \\ 
	\hline Proximity 	 & 3173 	/ 3 KB				& 135 				& 0.003 \\ 
	\hline Skærm 		 & 138           				& 6          		& 0.003 \\ 
	\hline 
\end{tabular} 

Denne tabel ligner ikke overraskende den tabel der blev produceret af eksperimentet på 30 minutter.
Det er igen accelerometret og gyroskopet der fylder databasen efterfulgt af lydmodulet der bruger væsentligt mindre, og til sidst de resterende moduler der næsten ikke bruger noget plads.

\subsection{Muligheder for begrænsning af pladsforbrug}
For ikke at fylde mobil telefonen op med data fra modulerne kan det være en god idé at begrænse pladsforbruget, hvilket kan gøres på forskellige måder, nogle af måderne er beskrevet herefter.

\subsubsection{Client server}
Ved at sætte en server til rådighed for at holde på data kan man holde det aktuelle forbrug på selve telefonen nede.
Det vil kræve at man sætter en synkronisering op der sørger for at kun data der allerede er analyseret bliver slettet fra telefonen. 
Frekvensen af en sådan synkronisering afhænger derfor af hvor lang tids data analysemodulerne er afhængige af.
Det vil selvfølgelig altid være muligt at hente data tilbage fra serveren, eller hvis pladsen bliver et større problem, udføre analyser på serversiden.

\subsubsection{Oprydning i data}
Skærm tændt modulet er opbygget således at det kun indsættes data når skærmen bliver tændt eller slukket.
En udvidelse af de andre moduler kan gøre at de virker ligedan. 
For eksempel er accelerometerdata kun interessante når der sker en vis svingning i acceleration.
Der kan da sættes en tærskel for hvor store svingningerne i acceleration der skal ske for at dataen gemmes i databasen.
Her kunne algoritmer som Douglas-Peucker algoritme eller Sliding Window blive brugt. \ivan{Forklar gerne utralkort}

\subsubsection{Opdateringshastighed}
Som nævnt under opsætningen til forsøget bruges android konstanter til at angive hvor hurtigt en sensor skal opdatere.
Disse konstanter er \textbf{indbygget} og beregnet til opdateringshastigheder der er hurtige nok til at være responsive ved brug i applikationer.\ivan{RØD RETTELSE}
Vores kontekst er at logge brugerens færden, og 5 gange i sekundet er ikke nødvendigt.
Der kan derfor spares en del plads ved enten at opdatere langsommere generelt, eller ved at ændre opdateringshastigheden løbende i takt med at der kommer mere relevant data.
Det skal dog overvejes om den nedsatte præcision kommer til at have en effekt på de analyser der skal bruge dataen.
\ivan{Er det en ide at diskutere opdateringshastighed i forhold til forskellige typer data?}


\subsection{Konklusion}
Når telefonen ikke er i brug vil det valgte udsnit af moduler generere 49 MB data i døgnet mens aktiv brug af telefonen vil få dette forbrug op på 55 MB.

De store syndere er accelerometret og gyroskopet der begge er sat til at gemme data 5 gange i sekundet.
Det vil derfor være nødvendigt at udvikle en strategi for begrænsning af datamængde.
Her er foreslået enten oprydning af data som går ud på kun at gemme det data der er interessant for analyserne.
En anden mulighed er at nedsætte opdateringshastigheden så der ikke gemmes så ofte.
\ivan{En tredje mulighed er at aggregere (tælle \# skrift over tid istedet for diskrete værider)}

Implementation af enten den ene eller begge af disse strategier vil kunne reducere den gemte data, men det vil højst sandsynligt ikke stadig være for meget til at dataen kan holdes på telefonen i længere perioder.\ivan{OK. denne kombi af negativ/positiv kan jeg ikke parse. Jeg giver op :-)}
Derfor vil en udvidelse med server klient arkitektur være nødvendig, men der er ikke nok ressourcer eller tid til at implementere dette så derfor vil dette ikke blive implementeret.\ivan{Unødvendigt defensivt. Dette kan tilføjes en senere version}