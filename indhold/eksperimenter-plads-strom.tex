\subsection{Eksperiment}

På mobile enheder er der begrænset lagerplads grundet enhedens størrelse, som eksempel har vores Samsung Galaxy S4 enheder kun 16 GB hvorimod mange computere nu om dage har mindst 1 TB (ca 100 gange mere).
Dette er nødvendigt at forholde sig til, da det sætter en begrænsning på hvad man kan tillade sig at gøre.
i vores tilfælde er det vi skal overveje hvor meget sensor data vi rent faktisk kan lagre.
For at undersøge hvordan de forskellige sensorer bruger pladsen på telefonen opstilles et eksperiment.

Applikationen blev sat til at køre med følgende moduler tændt:
\begin{itemize}
	\item sound
	\item screen
	\item proximity
	\item location
	\item gyroscope
	\item accelerometer
\end{itemize}

Moduler som screen, proximity og location producerer kun data når der sker ændringer. 
Derfor blev testen udført på en måde der forsøger at efterligne brug. 
Telefonen blev bevæget og skærmen blev tændt og slukket flere gange under testperioden.

\paragraph{Forespørgselshastighed}
Pladsen som modulerne bruger vil naturligvis afhænge af hvor ofte modulerne forespørger sensorerne for data.

I android er der forskellige metoder til at angive forespørgselsraten på en sensor.
Der findes fire konstanter der kan benyttes til at angive hastigheder der passer i forskellige kontekster \cite{sensormonitor}.

\begin{itemize}
	\item SENSOR\_DELAY\_FASTEST forespørger hele tiden.
	\item SENSOR\_DELAY\_GAME forespørger hver 20. ms, og er beregnet til spil.
	\item SENSOR\_DELAY\_UI forespørger hver 60. ms og er tilstrækkeligt til brug i userinterfaces.
	\item SENSOR\_DELAY\_NORMAL forespørger hver 200. ms er tilstrækkeligt til at opfange ændringer i skærm orientering.
\end{itemize}
Det er også muligt at angive opdateringen i ticks hvis man vil have en langsommere opdatering.
I dette eksperiment er SENSOR\_DELAY\_NORMAL benyttet.


\subsection{Resultater}
Til analyse af pladsforbruget blev programmet ``SQLite-analyzer'' benyttet \cite{sqliteanalyzer}.
SQLite-analyzer viser statistik for en SQLite database inklusive data for de enkelte tabeller.

Efter 30 minutter fyldte databasen 1156459 bytes.
Under antagelsen at sensorerne vil fortsætte med at generere data i den samme hastighed vil det efter 24 timer fylde omkring 55 MB. 

De enkelte sensorers tabeller var fordelt på følgende måde

\begin{tabular}{|c|c|c|c|}
	\hline Modul & Plads forbrugt & Antal indgange & \% af hele databasen \\
	\hline Accelerometer & 609266 & 12434 & 51.1 \\ 
	\hline Gyroskop & 495018 & 10146 & 41.6\\ 
	\hline Lyd & 42648 & 1706 & 4.3 \\ 
	\hline Lokation & 4751 & 91 & 0.92 \\ 
	\hline Proximity & 3173 & 135 & 0.31 \\ 
	\hline Skærm & 1242 & 54 & 0.31 \\ 
	\hline 
\end{tabular} 

\subsection{Muligheder for begrænsning af pladsforbrug}

\paragraph{Client server}
Ved at sætte en server til rådighed for at holde på dataen kan man holde det aktuelle forbrug på selve telefonen nede.
Det vil da kræve at man sætter en synkronisering op der sørger for at kun data der allerede er analyseret bliver slettet fra telefonen. 
Frekvensen af en sådan synkronisering afhænger derfor af hvor lang tids data analysemodulerne er afhængige af.
Det vil selvfølgelig altid være muligt at hente data tilbage fra serveren, eller hvis pladsen bliver et større problem, udføre analyser på serversiden.

\paragraph{Oprydning i data}
Skærm tændt modulet er opbygget således at det kun indsættes data når skærmen bliver tændt eller slukket.
En udvidelse af de andre moduler kan gøre at de virker ligedan. 
For eksempel er accelerometerdata kun interessante når der sker en vis acceleration.
Der kan da sættes en tærskel for hvor meget acceleration der skal ske for at dataen gemmes i databasen.

\paragraph{Opdateringshastighed}
Som nævnt under opsætningen til forsøget bruger android konstanter til at angive hvor hurtigt en sensor skal opdatere.
Disse kontanter er indbygget og er beregnet til opdateringshastigheder der er hurtige nok til at være responsive ved brug i applikationer.
Sensorer der bruger disse konstanter til opdatering har i 
Vores kontekst er at logge brugerens færden, og 5 gange i sekundet er ikke nødvendigt.
Der kan derfor spares en del plads ved enten at opdatere langsommere generelt, eller ved at ændre opdateringshastigheden løbende i takt med at der kommer mere relevant data.
Det skal dog overvejes om den nedsatte præcision kommer til at have en effekt på de analyser der skal bruge dataen.