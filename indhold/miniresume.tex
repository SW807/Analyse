\chapter{Mini Resume til Fokusgruppemøde}
Dette dokument skal bruges til at sikre relevante spørgsmål og emner bliver diskuteret ved fokusgruppe mødet samt at give en struktur til mødet så mødelederne har noget af følge og give dem de værktøjer de skal bruge til at styre mødet.

\section{Formål med Interview}
Til fokusgruppemødet er formålet specifikt at få viden fra en relevant brugergruppe vedrørende en mobil applikation som kan give dem information om deres situation, i henhold til psykiske lidelser, ved at se på forskellige aspekter af deres adfærd, som f.eks. fysisk aktivitetsniveau, søvnmønster eller socialt aktivitetsniveau. 
Man vil se på disse da en af de bedste indikatorer for en forværrende depression eller optrappende mani er ændringer i adfærd.
Eksempler på dette kunne være at hvis brugeren pludselig begynder at sove meget eller stopper med motion kan dette være en indikation på en forværrende depression eller hvis brugeren pludselig begynder at sove meget lidt eller begynder at dyrke meget motion kan dette være en indikation for en optrappende mani.
 
Da patienterne i fokusgruppen er den målgruppe som systemet er rettet imod, er det vigtigt at vise dem produktet og se om de kan forstå det og bruge det.
På samme tid er det også vigtigt få deres holdning til at blive overvåget af et system og hvilke bekymringer de har.
Der vil også være mulighed for at diskutere deres tanker om eksisterende idéer til analyse af deres situation og om de overhovedet synes det kan bruges til noget.

\section{Fremgangsmetode og spørgsmål}
Dette afsnit dokumenterer fremgangsmåde til fokusgruppemødet og hvilke emner og spørgsmål der skal spørges ind til.

Fremgangsmåden er at der vil stilles generelle spørgsmål som naturligt fører emnet til det udarbejdede system, hvorefter idéen og systemet vil blive præsenteret for patienterne. 
Herefter vil der stilles mere specifikke spørgsmål om deres tanker.

\begin{itemize}
\item Hvad oplever i oftest er en indikator for forværring af jeres situation i henhold til affektive lidelser?
\item Hvilken slags information ville i ønske der blev gemt vedrørende jeres opførsel? Dette kunne f.eks. være information om jeres søvnmønstre.
\item Hvad er jeres holdning til at blive overvåget af et system som kan give jer indsigt i jeres situation i henhold til jeres affektive lidelser specifikt ved at se på jeres adfærd og undersøge om den ændrer sig.
\item \textbf{Præsenter systemet og idé bag det}
\item {Synes i systemet er forståeligt og brugbart?
	   \begin{itemize}
	   \item Hvis ja, hvad synes i den gør godt?
	   \item Hvis ikke, hvorfor synes i at det ikke er?
	   \item Hvis ikke, hvordan kan det gøres mere forståeligt og brugbart? \lasse{Skal denne være her, virker lidt usability test agtigt og tror ikke de kan sige så meget om det.}
	   \end{itemize}
\item {Hvilke bekymringer har i om systemet generelt?
	  \begin{itemize}
	  \item Hvordan kan de bekymringer lettes?
	  \end{itemize}}
\item {Tror i at de idéer præsenteret for jer, kan bruges til at forudse/give information om en forværring i jeres tilstand?
	   \begin{itemize}
	   \item Hvis ikke, hvorfor tror I ikke det og hvad synes I så der kan bruges?
	   \end{itemize}}
\item Har I idéer som I synes ville være ideelle at bruge, men mangler i systemet?

\end{itemize}

\lasse{Der skal flere spørgsmål på!}