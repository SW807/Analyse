\chapter{Miniresume af Interview}
Dette dokument skal bruges til at sikre ved fokusgruppe mødet med affektive patienter at relevante spørgsmål og emner bliver diskuteret samt at give en struktur til mødet så mødelederne kan lettere styre det.

\section{Formål med Interview}
Et fokusgruppemøde er en kvalitativ måde at få information vedrørende et domæne meget effektivt, specifikt at få information om den relevante gruppens holdninger og meninger vedrørende et eller andet emne.

Til vores fokusgruppe møde er formålet specifikt at få viden fra en relevant brugergruppe vedrørende en mobil applikation som kan give dem information om deres situation ved at kigge på deres fysiske aktivitetsniveau, deres søvnmønstre og deres sociale aktivitetsniveau. Man vil se på disse da en af de bedste indikatorer for en forværrende depression eller optrappende mani er ændringer i adfærd, f.eks. hvis brugeren pludselig begynder at sove meget eller stopper med motion kan dette være en indikation for en forværrende depression samt hvis brugeren pludselig begynder at sove meget lidt eller begynder at dyrke meget motion kan dette være en indikation for en optrappende mani.
 
Da patienterne i fokusgruppen er den relevante målgruppe som skal bruge systemet, er det vigtigt at vise dem arbejdet og se om de kan forstå og bruge systemet. På samme tid er det også vigtigt få deres holdning til at blive overvåget af en applikation og hvad de er bekymret om, f.eks. at systemet kan få adgang til deres SMSer. Til sidst vil der også være mulighed for at diskutere hvad de synes om eksisterende idéer til analyse af deres situation og om de overhovedet synes det kan bruges til noget.

\section{Fremgangsmetode og spørgsmål}
Dette afsnit dokumenterer fremgangsmåde til fokusgruppemødet og hvilke emner og spørgsmål der skal spørges ind til.

Fremgangsmåden er at der vil stilles generelle spørgsmål som naturligt fører emnet til det udarbejdede system, hvorefter idéen og systemet vil blive præsenteret for patienterne. Herefter vil der spørges mere specifikke spørgsmål om deres tanker.

\begin{itemize}
\item Hvad oplever i oftest er en indikator for forværring af jeres situation i henhold til affektive lidelser?
\item Hvilken slags information ville i ønske der blev gemt vedrørende jeres opførsel? Dette kunne f.eks. være jeres søvnmønstre.
\item Hvad er jeres holdning til at blive overvåget af et system som kan give jer indsigt i jeres situation i henhold til jeres affektive lidelser specifikt ved at se på jeres adfærd og undersøge om den ændrer sig.
\item \textbf{Præsentér systemet og idé bag det}
\item {Synes i systemet er forståeligt og brugbart?
	   \begin{itemize}
	   \item Hvis ikke, hvorfor synes i at det ikke er?
	   \item Hvis ikke, hvordan kan det gøres mere forståeligt og brugbart?
	   \end{itemize}}
\item {Hvilke bekymringer har i om systemet generelt?
	  \begin{itemize}
	  \item Hvordan kan de bekymringer lettes?
	  \end{itemize}}
\item {Tror i at de idéer præsenteret for jer, kan bruges til at forudse/give information om en forværring i jeres tilstand?
	   \begin{itemize}
	   \item Hvis ikke, hvorfor tror i ikke det og hvad synes i så der kan bruges?
	   \end{itemize}}
\item Har i idéer som i synes ville være idéelle at bruge, men mangler i systemet?

\end{itemize}