% Psykiske lidelser, hvor mange er påvirket?
% Forskellige slags, nogle er normalt fungerende

% Normalt fungerende = normale mennesker = forskellige behov
% Platform der kan ramme bredt
% Usikkert område, som stensikkert er her i lang tid = mulighed for at tilpasse

% Mulighed for at bruge mobilteknologi (initierende problem)
% Undersøge muligheder ift. sensorer/logning

I dagens Danmark lider mange danskere af en psykisk sygdom, ifølge Psykiatrifonden er det sådan at \textit{''I Danmark får hver 3. af os på et tidspunkt en psykisk sygdom – og mange flere er pårørende. 
Nu og her oplever næsten 700.000 danskere psykiske problemer, hvoraf stress, angst og depression udgør den største kilde til alt fra dårlig trivsel til svær sygdom.''}\citep{psykiatrifonden}.
Idet der er så mange der lider af psykiske sygdomme ville det være fordelagtigt at undersøge mulighederne for at lave et hjælperedskab der kan informere patienter om deres sindstilstand og dermed gøre sygdomsforløbet nemmere at håndtere.

Den nuværende praksis kræver at patienten bliver evalueret af en læge.
Hvis man kunne supplere med en metode, der kan aflaste lægen og assistere patienten i sin hverdag, vil dette være behjælpeligt. 

For at være brugbart i en sygdomssituation bør et sådan værktøj i størst muligt omfang emulere den praksis som læger følger i dag.

En mulig platform for et værktøj af denne beskrivelse er en smartphone.
En smartphone har en mængde sensorer som kan udnyttes til at opsamle information om patientens adfærd uden at forstyrre patienten.
Ydermere, kan en smartphone interagere med patienten i løbet af dagen. 
Det er altså ikke nødvendigt at vente til en konsultation, som hvis man skulle konsultere læge.

Til at understøtte sensorerne på smartphone vil det være muligt at benytte smartwatches og smartwristbands.
Ved sværere lidelser kunne det være en god tilføjelse, da det vil give et bedre billede.
Dog, idet at mange af denne slags enheder er meget nye, er der ikke særligt mange som har dem endnu.

Denne rapport vil omhandle udviklingen af et system der kan assistere patienter med psykiske lidelser på en mobil platform, således at man kan supplere den nuværende behandling hos en læge med behandling i hverdagen.