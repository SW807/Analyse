% Psykiske lidelser, hvor mange er påvirket?
% Forskellige slags, nogle er normalt fungerende

% Normalt fungerende = normale mennesker = forskellige behov
% Platform der kan ramme bredt
% Usikkert område, som stensikkert er her i lang tid = mulighed for at tilpasse

% Mulighed for at bruge mobilteknologi (initierende problem)
% Undersøge muligheder ift. sensorer/logning

I dagens Danmark lider mange danskere af en psykisk sygdom og ifølge Psykiatrifonden er det sådan at \textit{''hver 3. af os på et tidspunkt [får] en psykisk sygdom – og mange flere er pårørende. 
Nu og her oplever næsten 700.000 danskere psykiske problemer, hvoraf stress, angst og depression udgør den største kilde til alt fra dårlig trivsel til svær sygdom.''}\citep{psykiatrifonden}.
Idet der er så mange der lider af psykiske sygdomme vil det være fordelagtigt at undersøge mulighederne for at lave et hjælperedskab der kan informere patienter om deres sindstilstand og dermed gøre sygdomsforløbet nemmere at håndtere.

Den nuværende praksis kræver at patienten bliver diagnosticeret af en psykolog.
Hvis man kunne supplere med en metode, der kan aflaste psykologen og assistere patienten i sin hverdag, vil dette være behjælpeligt. 

En mulig platform for et værktøj af denne beskrivelse er en smartphone.
En smartphone har en mængde sensorer som kan udnyttes til at opsamle information om patientens adfærd uden at forstyrre patienten.
Ydermere kan en smartphone interagere med patienten i løbet af dagen. 

Til at understøtte sensorerne på smartphones vil det være muligt at benytte smartwatches og smartwristbands der også er udstyret med sensorer, men som har den fordel at de er tættere bundet til patienten.

Denne rapport omhandler udviklingen af en platform, der udgør et fundament for moduler der kan assistere patienter med affektive lidelser, på en mobil platform, således at man kan supplere den nuværende behandling hos en læge.