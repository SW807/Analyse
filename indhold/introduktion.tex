% Psykiske lidelser, hvor mange er påvirket?
% Forskellige slags, nogle er normalt fungerende

% Normalt fungerende = normale mennesker = forskellige behov
% Platform der kan ramme bredt
% Usikkert område, som stensikkert er her i lang tid = mulighed for at tilpasse

% Mulighed for at bruge mobilteknologi (initierende problem)
% Undersøge muligheder ift. sensorer/logning

I dagens Danmark lider mange danskere af en psykisk sygdom, ifølge Psykiatrifonden er det sådan at \textit{''I Danmark får hver 3. af os på et tidspunkt en psykisk sygdom – og mange flere er pårørende. 
Nu og her oplever næsten 700.000 danskere psykiske problemer, hvoraf stress, angst og depression udgør den største kilde til alt fra dårlig trivsel til svær sygdom.''}\cite{psykiatrifonden}.
Idet at der er så mange der lider af psykiske sygdomme ville det være idéelt at have en måde at informere patienter om deres sindstilstand og dermed gøre sygdomsforløbet nemmere at håndtere.

Hvis en metode der ikke kræver at folk tager til lægen uden grund kunne findes ville det klart gøre det bedre for patienter med psykiske sygdomme.
En måde at løse dette på er ved at give folk et redskab de kan have på sig uden det forstyrrer dem. 
Det kan derfor være en god idé at kigge på ting som næsten alle danskere bruger til hverdag som de også tæt på sig hele tiden.

Det redskab der skal bruges bør overholde nogle bestemte krav for at kunne opdage psykiske lidelser.
Kravene det bør overholdes er at redskabet skal kunne gøre de ting som lægen gør for at se om en patient har en psykisk lidelse, eller også skal redskabet kunne benytte en ny måde at opdage en psykisk lidelse på.
Det der tænkes på som redskab er en mobil telefon, som har en masse sensorer der muligvis kan opdage psykiske lidelser. 
Ydermere, er mobil telefonen også et redskab den største del af den danske befolkning bruger.

Til at understøtte mobil telefonens sensorer vil det også være muligt at benytte sig at smartwatches og smartwristbands.
Problemet med dem er dog at de ikke ejes at den største del af den danske befolkning, men kan godt være en tilføjelse til dem med svære lidelser.

\bruno{Jeg synes introduktionen skal skrives om. Der mangler kilder. Sætningerne roder og er ofte for lange. Er i enige?}
\lasse{Hvad med nu?}