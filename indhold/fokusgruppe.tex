\section{Fokusgruppeinterview}
For at supplere viden om den aktuelle målgruppe; patienter med affektive lidelser, specifikt uni- og bi-polar affektiv lidelse, blev der arrangeret et fokusgruppeinterview med indlagte patienter.
Herunder vil der blive beskrevet, samt reflekteret over, de vigtigste punkter fra interviewet.
For et komplet referat af fokusgruppeinterviewet, se \cref{app:fokusgruppe-interview-referat}.
Det bør her dog nævnes, at vi fik det indtryk at patienterne til fokusgruppeinterviewet kun var unipolare og ingen officielle oplysninger blev givet om deres tilstand.

\subsection{Søvn}
Alle deltagende var enige om at søvn er en vigtig faktor.
Ikke blot varigheden af den daglige søvn men også tidspunkt og kvalitet af søvn (fx. sammenhæng og rolig/urolig).
To af patienterne nævnte søvn som et symptom, der opstod en klar ændring i op til deres depressions-periode.
Andet interessant, der blev nævnt her, var at der i forsøget på at falde i søvn blev udført beroligende aktiviteter såsom at se fjernsyn, læse eller drikke te.

\subsection{Social tilbagetrækning}
Den sociale faktor blev der lagt mindre vægt på, dog var der enighed om at patienternes netværk, og kontakt til andre, mindskes under depressionen.
Social kontakt blev også benævnt som en god forebyggende handling, inden depression opstod.

\subsection[Early warning]{Early warning\footnote{Også benævnt som intervention.}}
Dette var oprindeligt antaget som en usikker og til dels farlig feature.
Der var dog enighed om at det kunne være godt at blive gjort opmærksom på, i god tid, at ens situation er under forværring.
Der bliver dog nødt til at gøres overvejelser omkring levering og formulering af sådanne notifikationer.
Det bør i højere grad knyttes til ændring i adfærd, frem for advarsel om at depression var på vej.
Der var også stor enighed om at det kunne være godt at blive gjort opmærksom på at søge hjælp, enten ved læge (forbundet med medicin) eller ved hjem og venner.

Ved tidlig advarsel kunne patienterne også gribe til deres forebyggende handlinger, her blev nævnt motion og strukturering af hverdag.

\subsection{Dagbog og status}
Ligesom interventioner, var det tidligere antaget at direkte input fra brugere var en dårlig ting.
Dette var dog ikke set som en dårlig ting fra patienterne, som var interesseret i hvad som helst, som kunne hjælpe dem med at undgå depression.
Dette kunne både være en dagbog, der skal skrives hver dag eller løbende status-opdatering (fx. ift. nuværende tilstand).

En vigtig ting er dog at disse ting foregår inden depression, da under depression vil sådanne ting synes uoverkommelige set fra patienten synspunkt.

\subsection{Tilpasning i forhold til symptomer}
Som forventet, så er der forskel på de individuelle symptomer.
Både hvilke ting, der skal holdes øje med (fx. søvn) og på hvilken måde disse fremtræder (fx. kvalitet og længde af søvn samt tidspunkt for søvn).

I forhold til tilstands-spørgsmål skulle disse kunne vælges til/fra, samt tilpasses.
