% A description of use context and selected use scenarios (see Essence-book Chapter 13).


\section{Paradigm view}
\textit{Paradigm} bruges her til at undersøge problemet; herunder problemområde, stakeholders og scenarier.

\subsection{Reflection}
For at forstå problemet bedre er det nødvendigt at sætte sig ind i problemområdet.
Ved at kombinere indsigt fra brugerne, domæne eksperter o teknisk indsigt fra udviklerne har \emph{reflection} til formål at udnytte problemområdet bedst muligt, så man ender med den bedste løsning.

Til dette opstilles en \emph{challenge} der beskriver den udfordring projektet skal forsøge at løse samt en \emph{use context} hvor udfordringen finder sted.

\paragraph{Challenge}\label{paradigm:challenge}
Patienter der er diagnosticeret med unipolar eller bipolar affektiv lidelse vil have tilbagevendende perioder af henholds mani og/eller depression, beskrevet i \cref{sec:affektivelidelser}.
Det kan være et problem hvis tilstanden for en patient er under begyndende forværring og dette ikke opdages i tide, da det kan føre til en manisk eller depressiv periode.
Det kan være svært for patienten selv at opdage denne ændring i adfærd, da det kan være en blød overgang eller svært at acceptere.
Desuden kan der gå flere år mellem perioder, hvilket kan gøre at det ikke er noget der overhovedet overvejes af patienten.
Dette ville også kunne gøre at patienten heller ikke overvejer lægehjælp, hvorfor en periode heller ikke vil detekteres på denne måde.
Ved at bruge en smartphone til passivt at indsamle data omkring patienten, kunne der holdes øje med ændringer i denne data, som vil kunne sige noget om adfærdsændringer.
Herefter ville patienten kunne gøres opmærksom på disse, og måske undgå større forværring.

Denne beskrivelse af problemet resulterer i følgende \emph{challenge}:
Kan vi bruge mobilen/wearables til at overvåge adfærd for unipolare og bipolare patienter og gøre dem opmærksomme på adfærdsændringer?

\paragraph{Use context}
Konteksten som applikationen skal kunne bruges i er meget bred, da den omfatter hele patientens hverdag.
Der skal derfor tages højde for at forbindelse til GPS, WIFI og mobilnetværk ikke altid er tilgængelige.
Da applikationen skal logge data om patientens færden skal der håndteres at smartphonen kan være i lommen, i hånden, på et bord eller i lommen på en jakke der hænger i entréen.
Konteksten kan altså variere fra ude i en skov uden dataforbindelse til patientens arbejde.

\subsection{Stakeholders}
I dette projekt er patienterne den vigtigste \textit{stakeholder}, da det er patienterne der skal bruge systemet i hverdagen.
Systemet skal derfor udvikles på patienternes præmisser.
I dette projekt er det specifikt patienter unipolar eller bipolar affektiv lidelse vi beskæftiger os med.

Der er et antal sponsorer tilknyttet projektet.
Morten Aagaard har både erfaring inden for datalogi og psykologi og kan agere bindeled mellem de to discipliner.
Janne Vedel Rasmussen og Jørgen Aagaard arbejder inden for psykologifaget og kan derfor bidrage med viden inden for faget.
De har derudover en interesse for at få udviklet værktøjer der eventuelt vil kunne bidrage til deres arbejde.

\subsection{Scenarios}
Her undersøges hvordan \textit{the Challenge} bliver set fra brugerens perspektiv.
Teknikker til denne undersøgelse inkluderer at udforske systemets problemdomæne ved hjælp af \textit{Use scenarios}.

\paragraph{Use scenarios}
\textit{Use scenarios} bruges til at udforske ideer og muligheder i forhold til brugerens brug af systemet.

\subparagraph{Scenarier:}
\begin{itemize}
	\item Patienten bevæger sig rundt i sin hverdag med smartphonen i lommen. 
	\begin{itemize}
		\item Data logges i systemet om gemmes til at kunne blive analyseret.
	\end{itemize}
	
	\item Patienten vil gerne have applikationen til at fortælle hvordan den vurderer hans tilstand.
	\begin{itemize}
		\item Applikationen viser at patienten udviser normal adfærd.
		\item Applikationen viser at patientens sindstilstand er lavere end normalen. Patienten konsulterer sin liste af lystbetonede aktiviteter og udfører en af disse.
		\item Applikationen viser at patientens sindstilstand er lavere end normalen. Patienten foretager sig intet og tilstanden fortsætter med at forværres.
	\end{itemize}	

\end{itemize}
