\section{Arkitektur}\label{sec:arkitektur}
Dette afsnit præsenterer arkitekturen som er udarbejdet på baggrund af de specificerede behov i \cref{arkitekturkrav}.
Først beskrives den overordnede opbygning.
Derefter beskrives komponenterne mere detaljeret.

\paragraph{Platform}
Android er blevet valgt som platform, se \cref{sec:valg_af_android}, hvilket også lægger op til nogle overvejelser ift. arkitektur.

Arkitekturen er lavet ud fra ideen om at applikationer på Android kan kontakte hinanden igennem Android systemet.
Det er muligt for en hovedapplikation at starte en service der ligger i en anden applikation \citep{android_service}.
Dette udnyttes ved at pakke alle moduler i hver deres selvstændige applikation.
Ved at anvende en centraliseret database vil det være muligt for moduler at få adgang til data fra andre moduler.
Samtidig kan adgangsrettigheder til de forskellige tabeller styres ét sted.
På den måde kan adgang kontrolleres på alle moduler, inklusiv de der måtte blive udviklet i fremtiden.
Hertil ønskes en central styrende enhed, der skal fungere som bindeled mellem de installerede moduler.

\subsection{Opbygning}\label{arkitektur:opbygning}
Den overordnede arkitektur er opbygget af fire komponenter: \textit{manager}, \textit{moduler}, \textit{DB\footnote{Database} access} og \textit{DB}.
For at opfylde de specificerede behov i \cref{arkitekturkrav}, nærmere nøgleordene \textit{modulær}(se \cref{arkitekturkrav::modulaer}) og \textit{fleksibel}(se \cref{arkitekturkrav::fleksibel}) er arkitekturen opbygget af moduler der opsamler data, analyserer data og viser det uafhængigt af hinanden.
Disse moduler kan vælges til eller fra i \textit{manager}'en.


Et diagram over arkitekturen kan ses på \cref{arkitektur_udkast_1}.
\begin{figure}[h]
	\centering						%  l   b   r	t
	\includegraphics[scale=0.5, trim = 1cm 17.5cm 1cm 1cm, clip]{ArkitekturLucidChart}
	\caption{Systemets arkitektur}
  \label{arkitektur_udkast_1}
\end{figure}

\lasse{Ivan vil have noget med patterns her, men troede vi fjernede det?}

Herefter beskrives komponenterne i rækkefølge af deres indbyrdes afhængighed, således at forståelsen af hver komponent kun afhænger af det læste.

\subsection{DB}
Denne komponent administrerer data for systemets forskellige moduler.
Data opbevares i en række tabeller i et relationelt database system.
Hvert modul har mulighed for at definere egne tabeller, der alle gemmes i \textit{DB} komponenten.

Til dette projekt er der valgt en SQLite database da denne er standard i Android \citep{android_database}.


\subsection{DBAccess}\label{subsec:DBACCESS}
Denne komponent styrer adgangen til \textit{DB} komponenten så det sker på en ensrettet måde.
Da vi arbejder på en mobil platform er det værd at tage højde for lagerstyring og abstraktion derover.
Da der er begrænset plads på en smartphone kan det blive relevant at lagre noget af det indsamlede data i skyen.
Grundet denne potentielle opdeling af hvor data er lagret, kan det være nyttigt at abstrahere over hvor data er lagret.
En sådan abstraktion vil gøre at udviklere af moduler ikke skal tage højde for den bagvedliggende lagring.
Om det er lokal lagring, cloud eller en kombination håndteres på samme måde fra udviklerens synspunkt.

For at opfylde nøglepunktet \textit{kommunikativ}, se \cref{arkitekturkrav::kommunikation}, er det nødvendigt at sørge for at moduler har adgang til kun at skrive til deres egen database og samtidig læse fra alle andres database.
På denne måde forhindres det at eksterne moduler modificerer andre modulers data.
Dette er dog ikke blevet implementeret, men bliver diskuteret yderligere i \cref{databaserettigheder}.

\paragraph{Design} 
DB access benytter implementerer en \textit{ContentProvider} \citep{contentprovider}, der er et interface som Android stiller tilrådighed.
Content Providers er en måde hvorpå man kan stille data til rådighed for andre apps i Android.

Brugen af content provideren gør også at interfacet til DB access er veldefineret.
Ændring af lagringsmetoden eller tilføjelsen af backup i skyen vil kunne gøres under dette interface, så det ikke ødelægger kompatibilitet til moduler.

\subsection{Moduler}
Denne komponent befinder sig i et lag for sig selv og indeholder tre typer moduler: \textit{data}, \textit{analyse} og \textit{visualisering}.
Modulkomponenten indeholder applikationens hovedfunktionalitet og er ansvarlig for indsamling af data, bearbejdelse af data og visualisering af data.

Dette afsnit beskriver hvordan modulerne er defineret og giver en beskrivelse af de tre typer moduler.
Men først følger en begrundelse for valget af dette lag på baggrund af behovene i \cref{arkitekturkrav}.

\paragraph{Begrundelse for valg}
For at opfylde nøgleordet \textit{modulær}(se \cref{arkitekturkrav::modulaer}) er det valgt lave et lag der indeholder moduler, så det er let at tilføje og fjerne moduler.
Dette kunne fx være i en situation hvor der kommer nye sensorer på markedet, eller hvis der skal laves nye former for visualiseringer til det allerede indsamlede data.
Desuden opfylder det også nøgleordet \textit{kombinerbar}, fordi fx moduler fra \textit{data} kan kombineres til at lave et \textit{analyse}-modul.

\section{Modul Definition}

%Måde at lave eksterne modul-apps uden at ændre hoved-app
%Definere output (tabeller/kolonner), samt input (afhængigheder)
%Evt. konfigurationsmuligheder for moduler afhængige af det
%Fleksibel måde at modtage data fra andre moduler, uden at skulle opdatere app(s). Dvs. ikke design-mønstre: observer, mediator, men gennem content provider.
% modularisering i Android; multiple apps under samme package

For at gøre systemet mere fleksibelt, vil vi udtænke en modul-baseret arkitektur.
Det skal være muligt at tilføje eksterne moduler, uden at have behov for at lave ændringer i hoved-applikationen.
Dette kan gøre sig gældende når der kommer nye sensorer på markedet, eller hvis der skal laves nye former for visninger til det allerede indsamlede data.
Til dette er der valgt at bruge JavaScript Object Notation (JSON) samt JSON Schema \cite{json_schema}.
Eksemplerne der bruges herefter vil derfor være i henholdsvis JSON eller JSON Schema.

\subsection{Typer af moduler}
Der vil findes i alt tre typer moduler; \textit{sensor}, \textit{analyse} og \textit{view}.\footnote{Der bruger engelske begreber her for at holde det implementerings-nært.}
Sensor-modulerne repræsenterer de fysiske sensorer til stede i telefon eller tilsluttet wearable.
De leverer data som analysis eller view modulerne skal bruge til at henholdsvis behandle eller vise data.
Baseret på en eller flere sensor- eller analysis-moduler, kan et analyse modul levere behandlet data, til brug af andre analysis-moduler, samt view-moduler.
View-modulerne bruges til visning af den rå sensor-data eller den behandlet analysis-data.

Som minimum har et modul et navn og en version.
Sensor- og analysis-moduler vil have data, analysis- og view-moduler vil have afhængigheder.

\subsection{Data}
Sensor- og analysis-moduler skal gøre data tilgængeligt for andre analysis- og view-moduler.
For at specificere hvordan den data skal gemmes, samt hvad der er tilgængeligt for andre, skal dette defineres for hvert modul af førstnævnte typer.
For hvert modul skal der defineres en eller flere tabeller, med navn, da det er muligt for ét modul at levere mere end én slags data.
For hver tabel defineres en eller flere kolonner, med et beskrivende navn, samt datatyper og evt. en måleenhed.

\subsection{Afhængigheder}
Et analysis- eller view-modul kan være afhængigt af andre sensor- eller analysis-moduler, da det kan være de skal bruge en bestemt slags data for at være anvendelige.
Derfor skal det defineres for hvert modul hvilke andre moduler det er afhængigt af.
Dette kan gøre på to måder; hard- eller soft-dependency.
En hard-dependency er ét andet modul, samt version, som det pågældende modul ikke kan fungere uden.
En soft-dependency er en liste af andre moduler, hvor mindst ét af de listede moduler skal være til stede på enheden.

\subsection{JSON og JSON Schema}
For at have en modul-beskrivelse der er læselig for både mennesker og maskiner, er JSON valgt.
Det burde også være muligt for ikke-tekniske personer at læse, skrive og forstå et JSON dokument.
For at sikre validiteten af eksternt leveret modul-beskrivelser, udarbejdes der et JSON Schema, som JSON-dokumenter kan holdes op imod og derved verificeres.
Det anvendte JSON Schema kan findes i \cref{app:json_schema}.

\paragraph{Eksempel} på en modul-beskrivelse.
Al meta-data er præfikset med \_ (underscore).
\begin{lstlisting}
{
  "name": "accelerometer",
  "_version": 1.0,
  "tables": [
    { "name": "accelerations",
      "columns": [
        { "name": "accX",
          "dataType": "REAL",
          "_unit": "g" },
        { "name": "accY",
          "dataType": "REAL",
          "_unit": "g" },
        { "name": "accZ",
          "dataType": "REAL",
          "_unit": "g" }
      ]}]}
\end{lstlisting}

\subsection{Implementering}
Som nævnt i \cref{valg_af_android}, implementeres der til Android telefoner.
Dette sætter nogle begrænsninger ift. valg af løsninger.

\subsubsection{JSON}
JSON blev valgt over XML, da der er bedre native support for JSON på Android.
Dette er ikke normalt et problem, men da vi gerne ville have automatisk generering af Java klasser ud fra vores schema, og det ikke kunne lade sig gøre med XML, blev JSON i stedet valgt.
Der er ingen begrænsninger ved JSON, frem for XML, dog en anden syntaks.

\subsubsection{Moduler som apps}
For at det skal være muligt at installere moduler uden at opdatere hoved-applikationen, skal der installeres apps via Google Play Store.
Alle modul-apps, samt hoved-applikationen, deler \textit{package}-navn.
Hver modul-app har sin JSON beskrivelse som en eksternt tilgængelig \textit{resource}, som hoved-applikationen eller andre moduler har adgang til.

Kommunikation mellem apps foregår med \textit{services}, \textit{intents} eller \textit{content provider}.
Services og intents skal defineres i applikationens \textit{manifest}, og ville derfor skulle ændres hver gang der kom nye moduler til.
I stedet for benytter vi content provider til både at gemme data opsamlet af moduler, samt at gøre tilgængelig til andre moduler.

\subsubsection{Data typer}
De tilgængelige data typer tilgængelig for tabel-kolonner, er begrænset til dem som er tilgængelig i SQLite, som er den database der bruges på Android telefoner.
Der er 5 typer: \textit{NULL}, \textit{INTEGER}, \textit{REAL}, \textit{TEXT} og \textit{BLOB}.


\subsection{Manager}\label{subsec:arkitektur-Manager}
Dette afsnit begrunder valget af denne komponent.
Derefter beskrives komponenten.

\paragraph{Begrundelse for valg}
For at opfylde nøgleordet \textit{fleksibel}(\cref{arkitekturkrav::fleksibel}), er der tilføjet en komponent kaldet manager.
Manager komponenten kan siges at være grænsefladen mellem bruger og moduler.
Den står for at administrere de installerede moduler ud fra de beskrivelser der er givet for de enkelte moduler.
Denne administration indebærer blandt andet oprettelse af de database tabeller hvert modul har bedt om i sin beskrivelse, samt start og stop af sensor- og analysemoduler.
Sidstnævnte sker ud fra definitioner givet i beskrivelserne af de enkelte moduler.
Desuden er det også gennem manageren at brugeren får vist information, såsom visualiseringer.

\subsubsection{Beskrivelse af manageren}
Manageren er den eneste komponent brugeren interagerer med.
Det er igennem manageren at brugeren vælger hvilke moduler, der skal køre.
Det er også den, som sørger for at de rigtige moduler kører på de rigtige tidspunkter.
Dette gøres ved hjælp af \texttt{TaskRunner}.
Derudover er det også manageren, der sørger for at skaffe moduldefinitionen fra alle de installerede moduler.
Ydermere, giver manageren et overblik over de visualiseringsmoduler, som er installeret samt have en måde at vise visualiseringsmodulerne på.
\paragraph{Udførsel af moduler}
Der er overordnet to måder hvorpå moduler køres; enten styrer de selv deres kørsel (kontinuerte kørende moduler) eller så administreres de af Managerens \texttt{TaskRunner}, som er en \texttt{Service} der startes når mobilen tændes.
Både sensor- og analyse-moduler kan køre på disse to forskellige måder.

\paragraph{Kontinuerligt kørende moduler}
Til moduler der skal køres kontinuert, startes deres \texttt{Service} blot lige så snart modulet aktiveres i indstillinger, hvorefter det selv administrerer hvornår og hvor ofte det udfører dets opgave.
Fordelen ved dette er at vi sparer kommunikations-overhead, da der ikke konstant skal kommunikeres mellem Manager og kontinuert-kørende moduler.

\paragraph{TaskRunner}
Derudover er der også moduler som kun skal køres med faste, større, intervaller, eller på bestemte tidspunkter.
Her er der ikke behov for at det enkelte modul har en kontinuert kørende \texttt{Service}, men kan derimod nøjes med at Manageren starter modulets opgave på det korrekte tidspunkt.
På denne måde spares der ressourcer, da modulet kun aktiveres i den tid hvor det skal udføre sin opgave. 

Når \texttt{TaskRunner}en startes, danner den en liste over de moduler der skal køres med interval eller på fast tidspunkt.
Derefter laver den en prioriteret kø, sorteret efter næste kørsels-tidspunkt.
\texttt{TaskRunner} tråden \texttt{sleep()}es så, indtil næste opgave skal udføres.
Efter opgaven er udført, udregnes næste kørsels-tidspunkt for den netop kørte opgave, hvorefter prioritets-køen sorteres.
Så \texttt{sleep()}es tråden igen, indtil næste kørsels-tidspunkt, og dette fortsætter så længe der er aktive moduler der skal køres på denne måde.

\paragraph{Indstillinger}\label{sec:settings}
\chapter{Indstillinger Arbejdsblad}
%guidelines
For at få en velkendt og standardiseret brugergrænseflade fulgtes android design guidelines.
Disse guidelines angiver hvornår man skal bruger diverse knapper, actionbars, settings etc.
\winde{Kilde til androids guidelines}

%Prototypes
Ved at følge disse guidelines blev en række prototyper for indstillinger lavet.
Disse byggede på samme princip om at udarbejde en indstillingsmenu.
Der var diskussion om hvordan disse skulle være, men over flere iterationer valgtes der at gå fra en "wizard" tilgang til en regulær settings menu,
Billeder af diverse prototyper kan ses i \cref{fig:prototype-manager}

\begin{figure}
	\centering
	\begin{subfigure}[b]{0.45\textwidth}
			\includegraphics[scale=0.3, page=1, trim = 1cm 5.5cm 1cm 0cm, clip]{prototype.pdf}
			\caption{Forside}
	\end{subfigure}
	\begin{subfigure}[b]{0.45\textwidth}
			\includegraphics[scale=0.3, page=2, trim = 1cm 5.5cm 1cm 0cm, clip]{prototype.pdf}
			\caption{Indstillinger}
	\end{subfigure}
	\caption{Prototype af Manager}
	\label{fig:prototype-manager}
\end{figure}


%Actionbar
Ud fra prototypen kan en actionbar blandt andet ses, tanken er at følge et standard design hvor man har en actionbar i toppen.
Denne muliggør navigation til indstillinger, men også at gå tilbage til hovedmenuen, nuværende er der dog problemer med at den ikke vises under settings\als{Skal have lavet mere arbejde med den}

%Indstillinger, checkbox
Til at angive om et givent modul skal være aktiveret eller ej bruges checkboxes.
Dette skyldes at det er et simpelt ja/nej valg. 
Tanken er så at de moduler man har valgt er dem der kører på telefonen.

%indstillinger, dependencies og events
For at scanne mobilen for de moduler der er installeret bruges JSONParser der tager vare af dette. \als{referer til JSONParser}
Dette giver udslag i en række moduler der har afhængigheder af andre moduler og skal takles.
JSONParseren giver som resultat en liste af moduler. Disse scannes så igennem for at finde deres afhængigheder.
Disse afhængigheder bruges så til at konstruere events (OnChange) til at fortælle de moduler der skal have besked når et givent modul aktiveres/deaktiveres.
Ved at lave en sådan række er der implicit konstrueret en dependency graph.
Og som resultat af dette kan man forestille sig et hierarki hvor et modul på det lavest liggende niveau medfører en kæde af deaktiveringer af moduler der eksplicit og implicit afhænger af dette modul.

%indstillinger, onPause start og stop sensorer
Efter afhængighederne er enkodet i programmet mangler der at takle hvordan sensorer skal startes og stoppes fra indstillingsmenuen.
Til at takle dette bruger vi den allerede udviklede ServiceHelper\als{referer til denne}.
Der vælges så at starte og stoppe de fornødne sensorer i onPause, da det typisk er når man forlader en indstillingsmenu at man gerne vil have at indstillingerne træder i kraft, og sikrer også at man ikke skal klikke på ekstra knapper for at indstillingerne træder i kraft.
\paragraph{JSON-Parser}\label{subsub:JSONparser}
JSON-parserens job er at finde JSON filen for hvert eneste modul installeret på smartphonen, hvorefter for hver modul bliver lavet et objekt ud fra den klasse lavet fra JSON-skemaet.
Måden dette er gjort på er ved brug af `jsonschema2pojo' \citep{jsonpojo}, der gør det muligt at få genereret klasserne som passer til JSON skemaerne, og som kan lave objekter baseret på de klasser og hvad der står i JSON filerne. 

