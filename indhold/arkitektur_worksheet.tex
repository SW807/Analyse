\section{Arkitektur}\label{sec:arkitektur}
Dette afsnit præsenterer arkitekturen som er udarbejdet på baggrund af de specificerede behov i \cref{arkitekturkrav}.
Først beskrives den overordnede opbygning.
Derefter beskrives komponenterne mere detaljeret.

\paragraph{Platform}
Projektet har valgt at bruge Android platformen, se \cref{sec:valg_af_android}, som også lægger op til nogle overvejelser ift. arkitektur.

Arkitekturen er lavet ud fra ideen om at applikationer på Android kan kontakte hinanden igennem Android systemet.
Det er muligt for en hovedapplikation at starte en service der ligger i en anden applikation \cite{android_service}.
Dette udnyttes ved at pakke alle moduler i hver deres selvstændige applikation.
Ved at anvende en centraliseret database vil det være muligt for moduler at få adgang til data fra andre moduler.
Samtidig kan adgangsrettigheder til de forskellige tabeller styres et sted.
På den måde kan adgang kontrolleres på alle moduler, inklusiv de der måtte blive udviklet i fremtiden.
Hertil ønskes en central styrende enhed, der skal fungere som bindeled mellem de installerede moduler.

\subsection{Opbygning}\label{arkitektur:opbygning}
Den overordnede arkitektur er opbygget af fire komponenter: \textit{manager}, \textit{moduler}, \textit{DB\footnote{Database} access} og \textit{DB}.
For at opfylde de specificerede behov i \cref{arkitekturkrav}, nærmere nøgleordene \textit{modulær}(se \cref{arkitekturkrav::modulaer}) og \textit{fleksibel}(se \cref{arkitekturkrav::fleksibel}) er arkitekturen opbygget af moduler der opsamler data, analyser denne data og viser den uafhængige af hinanden.
Disse moduler kan vælges til eller fra i \textit{manager}'en.


Et diagram over arkitekturen kan ses på \cref{arkitektur_udkast_1}.
\begin{figure}[h]
	\centering						%  l   b   r	t
	\includegraphics[scale=0.5, trim = 1cm 17.5cm 1cm 1cm, clip]{ArkitekturLucidChart}
	\caption{Systemets arkitektur}
  \label{arkitektur_udkast_1}
\end{figure}

Herefter beskrives komponenterne i rækkefølge af deres indbyrdes afhængighed, således at forståelsen af hver komponent kun afhænger af det læste.

\subsection{DB}
Denne komponent administrerer data for systemets forskellige moduler.
Data opbevares i en række tabeller i et relationelt database system.
Hvert modul har mulighed for at definere egne tabeller, der alle gemmes i \textit{DB} komponenten.

Til dette projekt er valgt en SQLite database da denne er standard i Android \cite{android_database}.


\subsection{DBAccess}\label{subsec:DBACCESS}
Denne komponent styrer adgangen til \textit{DB} komponenten så det sker på en ensrettet måde.
Da vi arbejder på en mobil platform er det værd at tage højde for lagerstyring og abstraktion derover.
Da der er begrænset plads på en smartphone kan det blive relevant at lagre noget af det indsamlede data i skyen.
Grundet denne potentielle opdeling af lager placeringer, kan de være nyttigt at abstrahere over hvor data lagres.

For at opfylde nøglepunktet \textit{kommunikation}, se \cref{arkitekturkrav::kommunikation}, er det nødvendigt at sørge for at moduler har adgang til kun at skrive til deres egen database og samtidig læse fra alle andres database.
På denne måde forhindres det at eksterne moduler modificerer andre modulers data.

\paragraph{Design} 
For at imødekomme kravene om både at give ensrettet adgang til moduler og stille garantier om adgangsrettigheder er der blevet valgt at benytte facade designmønster \citep[s.~185]{gamma1994design}.

Grunden til at facade er benyttet er fordi det giver en kobling mellem moduler og databasen.
Denne kobling sker gennem et interface der sørger for modulerne kun skal opfylde nogle få krav for at få deres data gemt i databasen, samt for at få data fra andre moduler.
Endvidere, sørger den kobling for at skabe en specifik og sikker kommunikation mellem moduler sker nemt og sikkert, hvilket er nødvendigt for at opfylde nøglepunktet \textit{kommunikation}, se \cref{arkitekturkrav::kommunikation}.

\subsection{Moduler}
Denne komponent befinder sig i et lag for sig selv og indeholder tre typer moduler: \textit{data}, \textit{analyse} og \textit{visualisering}.
Komponenten indeholder applikationens hovedfunktionalitet og er ansvarlig for indsamling af data, bearbejdelse af data og visualisering af data.

Dette afsnit beskriver hvordan modulerne er defineret og en beskrivelse af de tre typer moduler.
Men først en begrundelse for valget af dette lag på baggrund af behovene i \cref{arkitekturkrav}.

\paragraph{Begrundelse for valg}
For at opfylde nøgleordet \textit{modulær}(se \cref{arkitekturkrav::modulaer}) er det valgt lave et lag der indeholder moduler, så det er let at tilføje og fjerne moduler.
Dette kunne fx være i en situation hvor der kommer nye sensorer på markedet, eller hvis der skal laves nye former for visualiseringer til det allerede indsamlede data.

\section{Modul Definition}

%Måde at lave eksterne modul-apps uden at ændre hoved-app
%Definere output (tabeller/kolonner), samt input (afhængigheder)
%Evt. konfigurationsmuligheder for moduler afhængige af det
%Fleksibel måde at modtage data fra andre moduler, uden at skulle opdatere app(s). Dvs. ikke design-mønstre: observer, mediator, men gennem content provider.
% modularisering i Android; multiple apps under samme package

For at gøre systemet mere fleksibelt, vil vi udtænke en modul-baseret arkitektur.
Det skal være muligt at tilføje eksterne moduler, uden at have behov for at lave ændringer i hoved-applikationen.
Dette kan gøre sig gældende når der kommer nye sensorer på markedet, eller hvis der skal laves nye former for visninger til det allerede indsamlede data.
Til dette er der valgt at bruge JavaScript Object Notation (JSON) samt JSON Schema \cite{json_schema}.
Eksemplerne der bruges herefter vil derfor være i henholdsvis JSON eller JSON Schema.

\subsection{Typer af moduler}
Der vil findes i alt tre typer moduler; \textit{sensor}, \textit{analyse} og \textit{view}.\footnote{Der bruger engelske begreber her for at holde det implementerings-nært.}
Sensor-modulerne repræsenterer de fysiske sensorer til stede i telefon eller tilsluttet wearable.
De leverer data som analysis eller view modulerne skal bruge til at henholdsvis behandle eller vise data.
Baseret på en eller flere sensor- eller analysis-moduler, kan et analyse modul levere behandlet data, til brug af andre analysis-moduler, samt view-moduler.
View-modulerne bruges til visning af den rå sensor-data eller den behandlet analysis-data.

Som minimum har et modul et navn og en version.
Sensor- og analysis-moduler vil have data, analysis- og view-moduler vil have afhængigheder.

\subsection{Data}
Sensor- og analysis-moduler skal gøre data tilgængeligt for andre analysis- og view-moduler.
For at specificere hvordan den data skal gemmes, samt hvad der er tilgængeligt for andre, skal dette defineres for hvert modul af førstnævnte typer.
For hvert modul skal der defineres en eller flere tabeller, med navn, da det er muligt for ét modul at levere mere end én slags data.
For hver tabel defineres en eller flere kolonner, med et beskrivende navn, samt datatyper og evt. en måleenhed.

\subsection{Afhængigheder}
Et analysis- eller view-modul kan være afhængigt af andre sensor- eller analysis-moduler, da det kan være de skal bruge en bestemt slags data for at være anvendelige.
Derfor skal det defineres for hvert modul hvilke andre moduler det er afhængigt af.
Dette kan gøre på to måder; hard- eller soft-dependency.
En hard-dependency er ét andet modul, samt version, som det pågældende modul ikke kan fungere uden.
En soft-dependency er en liste af andre moduler, hvor mindst ét af de listede moduler skal være til stede på enheden.

\subsection{JSON og JSON Schema}
For at have en modul-beskrivelse der er læselig for både mennesker og maskiner, er JSON valgt.
Det burde også være muligt for ikke-tekniske personer at læse, skrive og forstå et JSON dokument.
For at sikre validiteten af eksternt leveret modul-beskrivelser, udarbejdes der et JSON Schema, som JSON-dokumenter kan holdes op imod og derved verificeres.
Det anvendte JSON Schema kan findes i \cref{app:json_schema}.

\paragraph{Eksempel} på en modul-beskrivelse.
Al meta-data er præfikset med \_ (underscore).
\begin{lstlisting}
{
  "name": "accelerometer",
  "_version": 1.0,
  "tables": [
    { "name": "accelerations",
      "columns": [
        { "name": "accX",
          "dataType": "REAL",
          "_unit": "g" },
        { "name": "accY",
          "dataType": "REAL",
          "_unit": "g" },
        { "name": "accZ",
          "dataType": "REAL",
          "_unit": "g" }
      ]}]}
\end{lstlisting}

\subsection{Implementering}
Som nævnt i \cref{valg_af_android}, implementeres der til Android telefoner.
Dette sætter nogle begrænsninger ift. valg af løsninger.

\subsubsection{JSON}
JSON blev valgt over XML, da der er bedre native support for JSON på Android.
Dette er ikke normalt et problem, men da vi gerne ville have automatisk generering af Java klasser ud fra vores schema, og det ikke kunne lade sig gøre med XML, blev JSON i stedet valgt.
Der er ingen begrænsninger ved JSON, frem for XML, dog en anden syntaks.

\subsubsection{Moduler som apps}
For at det skal være muligt at installere moduler uden at opdatere hoved-applikationen, skal der installeres apps via Google Play Store.
Alle modul-apps, samt hoved-applikationen, deler \textit{package}-navn.
Hver modul-app har sin JSON beskrivelse som en eksternt tilgængelig \textit{resource}, som hoved-applikationen eller andre moduler har adgang til.

Kommunikation mellem apps foregår med \textit{services}, \textit{intents} eller \textit{content provider}.
Services og intents skal defineres i applikationens \textit{manifest}, og ville derfor skulle ændres hver gang der kom nye moduler til.
I stedet for benytter vi content provider til både at gemme data opsamlet af moduler, samt at gøre tilgængelig til andre moduler.

\subsubsection{Data typer}
De tilgængelige data typer tilgængelig for tabel-kolonner, er begrænset til dem som er tilgængelig i SQLite, som er den database der bruges på Android telefoner.
Der er 5 typer: \textit{NULL}, \textit{INTEGER}, \textit{REAL}, \textit{TEXT} og \textit{BLOB}.

\section{Visnings Modul}

Som udgangspunkt er \textit{visninger} forskellige visualiseringer af analyse modulernes output.

Det antages at analyse moduler kan og vil være meget forskellige, og at de enkelte visninger ikke nødvendigvis vil passe på alle analyser.
Derfor skal der være en måde at knytte de enkelte visninger med de passende analyser.
En måde at gøre dette på er at lave en baglæns afhængighed for en visning til en eller flere analyser, som beskriver hvilke analyser den pågældende visning kan repræsentere.
Problemet med dette er at denne liste, dvs. selve modul app'en, skal opdateres hver gang der oprettes nye analyser der passer sammen med visningen.
En anden måde er at have bindingen på analyse-modulerne, så der i stedet beskrives hvilke visninger der gør sig gældende for de enkelte analyser.
Her er problemet dog stadig det samme, at analyserne skal opdateres hver gang der kommer nye visninger.

En mere fleksibel måde at håndtere denne knytning mellem visninger og analyser er at deducere ud fra analysens tabel/kolonne signatur, hvilken slags data den outputter.
På denne måde kan der laves visninger der kan repræsentere bestemte signaturer, og der i stedet laves en implicit binding mellem visninger og analyser.

I mange tilfælde vil denne implicitte binding være nok, dog kan det forestilles at analyserne vil være forskellige og til tider komplekse, hvilket kan gøre det nødvendigt at nærmere specificere hvordan den tilgængelige data kan vises.
Her kunne der tilføjes noget meta-data på analysens tabeller/kolonner, der beskriver den på sådan en måde, at det ville kunne bruges til visninger der ikke nødvendigvis implicit kunne knyttes til det.

\subsection{Eksempel}
Her bruges light, som er en simpel sensor og analyse, der fra sensoren indhentes løbende lys-niveau (i lux).
Analysen tager den senest indsamlet data og giver et gennemsnit.
Det antages at alle tabeller har en tids-kolonne, der beskriver enten hvornår data blev indsat i databasen, eller i nogle tilfælde af analyser overføres tiden fra sensor-data.

\begin{lstlisting}
{
  "name": "lightAvg",
  "_version": 1.0,
  "tables": [
    {
      "name": "lightAvg",
      "columns":[
        { "name": "lightAvg", "dataType": "REAL", "_unit": "lux" }
      ]
    }
  ],
  "dependencies": [
    [{ name": "light" }]
  ]
}
\end{lstlisting}

Et eksempel på et visnigs-modul som passer på denne type analyse kunne være en simpel 2D graf, som viser den gennemsnitlige belysning over tid.

\begin{lstlisting}
{
  "name": "2dgraph",
  "_version": 1.0,
  "_type": "view",
  "view": {
    "layout": "2dgraph.xml",
    "data": [
      { "name": "x", "dataTypes": ["INTEGER"], "fromTimestamp": true },
      { "name": "y", "dataTypes": ["INTEGER", "REAL"] }
    ]
  }
}
\end{lstlisting}

Denne visning beskriver en 2D graf, som bruger layoutet i \texttt{2dgraph.xml}.
På x-aksen vises tiden (der står INTEGER fordi SQLite ikke har datorepræsentation ud over unix-time).
Her bruges det tidsstempel som forventes på alle tabeller.
Til y-aksen kan bruges alle analyser, som blot har en enkelt kolonne af heltal eller decimal-tal værdi.

\subsection{Administrering af visninger}
Ligesom sensorer og analyser, skal visninger også administreres af brugeren.

For at holde det så simpelt som muligt, kunne man, ligesom ved sensorer/analyser, have to niveauer af administration.
Som udgangspunkt vil alle installeret visninger automatisk knytte sig til de aktiverede analyser.
Derudover vil man i de avancerede indstillinger kunne aktivere/deaktivere visninger for de enkelte analyser.

På denne måde vil der automatisk blive genereret en liste af visning/analyse kombinationer, men med mulighed for at slå nogle af kombinationerne fra.
Det vurderes at muligheden for at slå visninger fra skal være under avancerede indstillinger, da ikke teknisk begavede brugere sagtens kunne forvirres af alle de potentielle visning/analyse kombinationer de ville præsenteret for.
Grunden til at det stadig er med som en mulig customization er, at der er stor fokus på at brugerne skal have mulighed for at styre hvad programmet skal gøre. 


\subparagraph{Data}
\textit{Data}-laget indeholder moduler der indsamler data fra smartphonens (eller tilbehør dertil) forskellige sensorer og applikationer.
Der påføres kun et minimum af behandling på indsamlede data (eksempelvis komprimering) således at data kan indsamles kontinuert uden stort energi-behov.
Derudover for at have mulighed for at lave meningsfulde analyser skal alt logget data gemmes i \textit{DB} i rå eller komprimeret form.
Det vil sige at komprimering skal være tabsfri og at eventuelle fejlmålinger bør markeres som sådan i den pågældende data tabel.

Et eksempel på et sensormodul er indsamling af data fra accelerometret.
Modulet får data fra accelerometret og gemmer det i sin database. 
For ikke at gemme unødvendige mængder data vil modulet kun gemme et datapunkt når det er forskelligt fra det forrige målte punkt.

\subparagraph{Analyse}
\textit{Analyse}laget indeholder moduler der bruger data fra et antal sensormoduler samt eventuelle andre analysemoduler.
Herefter udføres en analyse af det indsamlede data med \textit{''forståelig information''} som resultat.
I denne process vil der kunne forekomme tab af data.
Herved opnås en opsummering af det indsamlede data, der skaber værdifuld information for brugeren.
Som en del af et analysemoduls beskrivelse findes en beskrivelse af hvilken information man kan få fra modulet.
Denne information anvendes af visualiseringsmodulerne.

Der kan ved analyse af sensordata anvendes flere ressourcer, da analysen typisk vil kunne udføres på større mængder indsamlede data få gange dagligt.
Eksempelvis kan analysen foretages om natten hvor smartphonen kan sættes til opladning.

Et eksempel på et analysemodul kan være et søvnanalyse modul der bruger førnævnte accelerometer data til at undersøge en sovendes søvn.
Dette modul vil bruge flere sensormoduler til sin analyse og vil muligvis bruge accelerometerdata fra både en smartphone og et smart watch.

\subparagraph{Visualisering}
\textit{Visualiserings}laget indeholder moduler der visualiserer analyserede data.
Som en del af et visualiseringsmoduls beskrivelse findes en beskrivelse af hvilken information modulet accepterer.
Hvis denne beskrivelse stemmer overens med et analysemodul, kan den enkelte visualisering anvendes på den enkelte analyse.

Et eksempel på et visualiseringsmodul er et modul der visualisere de resultater som føromtalte søvnanalyse har fundet frem til.
Dette kunne være en graf der fortæller hvor dyb søvn man har været i, eller en simpel farvet indikator hvis farve viser hvor godt man har sovet de sidste par dage.

Eksempelvis vil et accelerometer \textit{sensor}modul kunne anvendes af et søvn \textit{analyse}modul der viser resultatet i et graf \textit{visualiserings}modul.

Skulle man her ønske en anden fortolkning af accelerometerets data kan et andet analysemodul anvendes.
Sammensætningen af analyse- og visualiseringsmoduler sker ved en beskrivelse af deres grænseflade.
Moduler med fælles grænseflade vil kunne kombineres.
Det vil sige at der kan anvendes flere forskellige visualiseringsmoduler til det samme analysemodul, givet at de er kompatible.
Tilsvarende kan det samme visualiseringsmoduler anvendes til flere forskellige analysemoduler.

\subsection{Manager}\label{subsec:arkitektur-Manager}
Dette afsnit begrunder valget af denne komponent.
Derefter beskrives komponenten.

\paragraph{Begrundelse for valg}
For at opfylde nøgleordet \textit{fleksibel}(\cref{arkitekturkrav::fleksibel}), er der tilføjet en komponent kaldet manager.
Manager komponenten kan siges at være grænsefladen mellem bruger og moduler.
Den står for at administrere de installerede moduler ud fra de beskrivelser der er givet for de enkelte moduler.
Denne administration indebærer blandt andet oprettelse af de database tabeller hvert modul har bedt om i sin beskrivelse, samt start og stop af sensor- og analysemoduler.
Sidstnævnte sker ud fra definitioner givet i beskrivelserne af de enkelte moduler.
Desuden er det også gennem manageren brugeren får vist den information de leder efter.

\paragraph{Design}
Ved at sammenholde visualiseringer og analyser kan manageren beskrive for brugeren hvilke data der kan fremvises og med hvilke visualiseringer det kan ske.
Manageren har desuden en prædefineret brugerflade der anvender ovenstående kombinationer til at vise brugeren de relevante informationer.

Manager komponenten indeholder desuden et JSON skema for hver modul-type i moduler komponenten.
Disse definitioner beskriver formatet for eventuelle nye moduler man måtte ønske at føje til systemet.
\stefan{et eller flere skemaer? Muligvis har views anderledes skema, men analyse og sensorer har det samme}

