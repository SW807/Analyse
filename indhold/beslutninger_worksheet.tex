\chapter{Beslutninger Worksheet}
Der var forskellige valg at træffe til projektet, dette arbejdsark detaljerer dem.
 
Problematikken ved at tage beslutninger ligger i at de forskellige muligheder kommer med forskellige fordele og ulemper.
Disse muligheder skal redegøres for og evalueres hvorefter der skal tages en beslutning.

\section{Udviklingsplatform Valg}
Platformen for projektet skal vælges, her ses der på hvilke fordele og ulemper de forskellige udviklingsplatforme har og endeligt laves der en beslutning.

Der er forskellige udviklingsplatform der kan vælges, specifikt Android, iPhone, Windows Phone. 

Fordelene ved Android at det er en meget åben udviklingsplatform som giver adgang til meget data uden at gøre noget specielt, bredere befolkning som bruger Android, der er ingen begrænsninger på hvad der kan udvikles og der er mange ressourcer tilgængelig til at udvikling såsom guides, tutorials og generel viden om hvordan man udvikler på Android. Android applikationer udvikles i Java, hvilket er et meget bredt brugt programmeringssprog.
Android har dog forskellige ulemper, som at der er mange forskellige smartphones med forskellig hardware og forskellig version af operativ systemet. 

Fordelene ved iPhone er primært at der er nemmere at udvikle til da det ikke er særlig fragmenteret i forhold til Android. Brugerfladen er meget standardiseret hvilket gør det nemmere at udvikle den del af applikationen. 
iPhone har ulemper, da det er et meget mere lukket system og ikke er så åben som Android idet meget data er aflukket og kan være umulig at få fat i. iPhone udvikling foregår kun på OS X, og kræver en licens. iPhone udvikling foregår i et sprog som ikke bruges bredt, Objective-C.

Fordelene ved Windows Phone er et meget modent/godt udviklingsmiljø og programmeringssprog og at brugerflade design er meget nemt. 
Ulemper ved Windows Phone er så at der ikke er særlig mange udviklings ressourcer tilgængelig da Windows Phone markedet ikke er særlig bredt. Windows Phone er også en lukket udviklingsplatform.

Vores projekt er meget afhængig af åben adgang til datakilder, hvilket Android giver bedre adgang til. Vores situation er også at udvikling på Android virker nemmere da det ikke påkræver da ingen på udviklingsholdet har OS X computere, og at udvikling i Java er allerede kendt blandt mange på udviklingsholdet. Grunden til at Windows Phone ikke blev valgt er at meget få som bruger udviklingsplatformen i forhold til Android.

\section{Data gemmes i lang eller kort tid?}
Data kan opbevares i lang tid eller i kort tid, hvilket medfører nogle fordele og ulemper. Dette afsnit evaluerer begge muligheder.

Data opbevaring i lang tid kommer med visse fordele. Det betyder at man har mere data at analysere og præsentere til brugeren af systemet.  
Der har selvfølgelig også nogle ulemper, et af at dataen vil fylde mere og mere jo længere de bliver opbevaret, specielt hvis frekvensen af dataindsamling ikke reguleres. Det kan også være en sikkerheds risiko, da hvis f.eks. lokations data opbevares i lang tid kan en angriber vide hvor ejeren af smartphonen har opbefundet sig, dette er ikke et så stort problem hvis mindre data bliver opbevaret. Det vil også betyde at data tager længere tid at analysere og præsentere for brugeren, da mere data logisk medfører en længere processerings tid.

Data opbevaring i kort tid kommer også med visse fordele, idet man har mindre data kræver det mindre plads og man kan optage data punkter med højere frekvens uden store bekymringer. % mer?
Det har selvfølgelig også nogle ulemper, idet man har data i kortere tid vil de være mindre sigende og have mindre detalje. % mere?

En mulig løsning ville være at optage detaljeret data og opbevare denne, hvorefter man forsøger at opsummere denne data så den fylder mindre og kan opbevares permanent. % Mere?

\section{Data opbevares lokalt eller fjernt?}
En af beslutningerne der skal tages er om data skal opbevares lokalt eller fjernt. 

Hvis data opbevares lokalt, skal man svare på problemerne præsenteret i ovenstående afsnit hvilket selvfølgelig er en ulempe, men det kommer også med nogle fordele, da hvis dataen ligger lokalt behøver man ikke at hente det ned over internettet og man kan tilgå det selvom internettet ikke fungerer.

Hvis man på den anden hånd opbevarer data fjernt, kommer det med nogle fordele som man ikke får når det ligger lokalt. Idet man opbevarer data fjernt, undgår man plads problemer på smartphonen, det er nemmere at dele data, evaluering af egen data i forhold til andres data gøres muligt hvilket kan give et bedre perspektiv på sin situation, det vil gøre det muligt at analysere data på serveren i stedet for på smartphonen. 
Det kommer dog med nogle ulemper. Hvis man opbevarer data fra hver eneste bruger vil det flytte pladsproblemet til serveren, hvilket dog er nemmere at takle end på smartphonen. Det vil også påkræve at systemet har adgang til internettet, og at brugeren kan sende meget data meget hurtigt og ikke har båndbredde problemer. Det vil også muligvis gøre systemet langsommere, da data skal hentes ned fra internettet.

Der er mange fordele ved at opbevare data fjernt, men det er der nok ikke mulighed for at gøre det da der ikke er nok tid til at undersøge det.

% Fjernt lyder godt, men det kan vi ikke nå!


