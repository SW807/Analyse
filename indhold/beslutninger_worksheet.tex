\chapter{Beslutninger Worksheet}
Der var der forskellige valg at træffe til dette projekt, dette arbejdsark detaljerer dem.
 
Problematikken ved at tage beslutninger ligger i at forskellige valg kommer med fordele og ulemper.
Disse beslutninger skal redegøres for og evalueres, og fastsættes.

\section{Platform Valg}
Hvilken platform skal vælges, udfra de muligheder der nu er, hvilke fordele og ulemper har de forskellige platformer?

Der er forskellige platforme, dog begrænset til smartphones, der kan vælges specifikt Android og iPhone. 

Fordelen ved Android at det er en meget åben platform som giver adgang til meget data uden at du behøver at gøre noget specielt, bredere befolkning som bruger Android, der er ingen begrænsninger på hvad der kan udvikles. Android applikationer udvikles i Java, hvilket er et meget bredt brugt programmeringssprog.
Android har dog forskellige ulemper, som at der er mange forskellige smartphones med forskellig hardware og forskellig version af operativ systemet. 

Fordelen ved iPhone er primært at der er nemmere at udvikle til da det ikke er særlig fragmenteret i forhold til Android. Brugerfladen er meget standardiseret hvilket gør det nemmere at udvikle den del af applikationen. 
iPhone har ulemper, da det er et meget mere lukket system og ikke er så åben som Android idet at meget data er aflukket og kan være umulig at få fat i. iPhone udvikling foregår kun på OS X, og kræver en licens. iPhone udvikling foregår i et sprog som ikke bruges bredt, Objective-C.

Vores projekt er meget afhængig af åben adgang til datakilder, hvilket Android giver bedre adgang til. Vores situation er også at udvikling på Android virker nemmere da det ikke påkræver da ingen på udviklingsholdet har OS X computere, og at udvikling i Java er allerede kendt blandt mange på udviklingsholdet.

\section{Data gemmes i lang eller kort tid?}
En af beslutningerne at tage er hvor lang tid data skal opbevares, og på samme tid skal fordele og ulemper evalueres. 

Data opbevaring i lang tid kommer med visse fordele, specifikt at det tillader at data kan gå længere tilbage, og han man har mere data at analysere og præsentere til brugeren af systemet. 
Der har selvfølgelig også nogle ulemper, et af disse er at man løber ind i det problem at dataen kan komme til at fylde meget hvis frekvensen af data indsamling ikke reguleres. 
Hvis dataen bliver opbevaret i lang tid, kan dette være en større sikkerheds risiko end hvis mindre data bliver opbevaret. % sikkerhed
Ved analysering af data tager en stor mængde historisk data længere tid at analysere end en mindre mængde nyere data. % Langsom at traversere igennem meget data

Data opbevaring i kort tid kommer også med visse fordele, idet at man har mindre data kræver det mindre plads og man kan optage data punkter med højere frekvens uden store bekymringer. % mer?
Det har selvfølgelig også nogle ulemper, idet at man har data i kortere tid vil de være mindre sigende og have mindre detalje. % mere?

En mulig løsning ville være at optage detaljeret data og opbevare denne, og så på en måde opsummere denne data så den fylder mindre og kan opbevares permanent. % Mere?

Dette lægger op til næste spørgsmål som skal besvares, vedrørende om data skal opbevares lokalt eller fjernt.

\section{Data opbevares lokalt eller fjernt?}
En af beslutningerne der skal tages er om data skal opbevares lokalt eller fjernt. 

Hvis data opbevares lokalt, skal man svare på problemerne præsenteret i ovenstående afsnit hvilket selvfølgelig er en ulempe, men det kommer også med nogle fordele, da hvis dataen ligger lokalt behøver man ikke at hente det ned over internettet og man kan tilgå det selvom internettet ikke fungerer.

Hvis man på den anden hånd opbevarer data fjernt, kommer det med nogle fordele som man ikke får når det ligger lokalt. Idet at man opbevarer data fjernt, undgår man plads problemer på telefonen, det er nemmere at dele data, det gør det nemmere at evaluere egen data i forhold til andres data hvilket kan give et bedre perspektiv på sin situation, det vil gøre det muligt at analysere data på serveren i stedet for på telefonen. 
Det kommer dog med nogle ulemper. Hvis man opbevarer data fra hver eneste bruger vil det flytte pladsproblemet til serveren, hvilket nok er nemmere at takle end på telefonen. Det vil også påkræve at systemet har adgang til internettet, og at brugeren kan sende meget data meget hurtigt og ikke har båndbredde problemer. 

% Fjernt lyder godt, men det kan vi ikke nå!


