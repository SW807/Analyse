\chapter*{Mini Protokol til Fokusgruppemøde}
Dette dokument skal bruges til at sikre, at relevante spørgsmål og emner bliver diskuteret ved fokusgruppe mødet.
Desuden skal det give en struktur til mødet så mødelederne har noget af følge og give dem værktøjer de kan bruge til at styre mødet.

\section*{Formål}
Til fokusgruppemødet er formålet specifikt at få viden fra en relevant brugergruppe vedrørende en mobil applikation som kan give dem information om deres situation, i henhold til psykiske lidelser, ved at se på forskellige aspekter af deres adfærd, som f.eks. fysisk aktivitetsniveau, søvnmønster eller socialt aktivitetsniveau. 
Man vil se på disse da en af de bedste indikatorer for en forværrende depression eller optrappende mani er ændringer i adfærd.
Eksempler herpå kan være at brugeren sover mere end normalt eller stopper med motion.
Dette kan være en indikation på en forværrende depression.
Hvis brugeren sover mindre end normalt eller begynder at dyrke meget motion kan dette være en indikation for en optrappende mani.
Normalt anvendes i denne sammenhæng til beskrivelse af den enkelte brugers vanlige adfærd.
Man ønsker altså at sammenholde egen nuværende adfærd med egen tidligere adfærd.
 
Da patienterne i fokusgruppen er den målgruppe som systemet er rettet imod, er det vigtigt at vise dem produktet og se om de kan forstå det og bruge det.
På samme tid er det også vigtigt at udensøge deres holdning til den overvågning systemet vil foretage af dem og hvilke bekymringer de måtte have dertil.
Der vil også være mulighed for at diskutere deres tanker om eksisterende idéer til analyse af deres situation og om de overhovedet mener at de vil kunne få et positivt udbytte af systemet.

\section*{Fremgangsmåde og Spørgsmål}
Dette afsnit dokumenterer fremgangsmåde til fokusgruppemødet og hvilke emner og spørgsmål der skal spørges ind til.

\subsection*{Fremgangsmåde}
Der vil blive stillet generelle spørgsmål som naturligt fører emnet til det udarbejdede system, hvorefter idéen og systemet vil blive præsenteret for patienterne. 
Herefter vil der blive stillet mere specifikke spørgsmål om patienternes tanker, samt ligges op til en åben diskussion.

\subsection*{Spørgsmål}
Herunder gives en liste af spørgsmål der følger ovenstående fremgangsmåde.

\begin{itemize}
\item Hvad er oftest en indikator for forværring af jeres situation?
\item Hvilken slags information ville I ønske der blev gemt vedrørende jeres adfærd? Dette kunne f.eks. være information om jeres søvnmønstre eller jeres sociale aktivitet.
\item Hvad er jeres holdning til at blive overvåget af et mobilt system som kan give jer indsigt i jeres situation?
\item \textbf{Præsenter systemet og idé bag det}
\item Synes I systemet er forståeligt og brugbart?
	   \begin{itemize}
	   \item Hvis ja, hvad synes I det gør godt?
	   \item Hvis nej, hvorfor synes I at det ikke er?
	   \item Hvis nej, hvordan kan det gøres mere forståeligt og brugbart?
	   \end{itemize}
\item Hvilke bekymringer har I om systemet generelt?
	  \begin{itemize}
	  \item Hvordan kan de bekymringer lettes?
	  \end{itemize}
\item Kan de idéer der er blevet præsenteret, bruges til at forudse/give information om en forværring I jeres tilstand?
	   \begin{itemize}
	   \item Hvis ikke, hvorfor tror I ikke det og hvad tænker I der kan anvendes istedet?
	   \end{itemize}
\item Har I idéer som I synes ville være ideelle at bruge, men mangler i systemet?
\end{itemize}